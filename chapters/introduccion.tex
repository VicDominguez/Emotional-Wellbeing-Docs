\chapter{Introducción}
\label{ch:introduccion}

\section{Contexto}

\section{Objetivos}

\subsection{Objetivo general}

\subsection{Objetivos específicos}


El objetivo de un \gls{pfg}, \gls{pfm} y  \gls{td} es una de las piezas clave a plantear, y a su vez una de las más complicadas. Se considera \textbf{la finalidad} del proyecto en cuestión a realizar y suele encajar dentro de una de las siguientes categorías:

\begin{itemize}
    \item \textbf{Contraste} o validación de una hipótesis. Este es típico de \glspl{td}, aunque algunos \glspl{pfm} y (muy raramente) \glspl{pfg} pueden caer dentro de esta categoría.
    \item \textbf{Desarrollo} o diseño de algo (e.g.~Software, hardware, sistema, edificio). Suele ser el más común en la rama de la ingeniería, tanto \glspl{pfm} como \glspl{pfg}.
    \item \textbf{Estudio} de un tema que deduce o descubre nuevo conocimiento. Éste suele ser más común en las ramas de las ciencias puras y humanidades, tanto \glspl{pfm} como \glspl{pfg}.
\end{itemize}

Decimos que es una pieza clave porque sirve como primer indicador de la consecución del proyecto. Si nos planteamos un objetivo, en las conclusiones podemos indicar si se ha cumplido o no el objetivo planteado. Por eso es necesario que el objetivo esté bien definido, porque si se acepta como objetivo válido en un proyecto, y éste se concluye como cumplido, el proyecto habrá sido ejecutado correctamente.

Ahora bien, ¿cómo determinamos que el objetivo se ha cumplido? pues intentando definirlo para que se pueda cumplir, es decir, intentando que sea:

\begin{itemize}
    \item \textbf{Acotado en el tiempo}, así es más fácil establecer un marco temporal para su realización y programar temporalmente las partes de las que se compone.
    \item \textbf{Medible}, para saber cómo de lejos estamos de llegar a un resultado aceptable.
    \item \textbf{Específico}, de manera que esté bien acotado y sea difícil embarcarse en tareas que no nos acerquen a su consecución.
    \item \textbf{Alcanzable}, porque si no lo es, por mucha intención y esfuerzo que le pongamos no se va a terminar.
    \item \textbf{Relevante}, porque si, en un \gls{pfg} para Ingeniería del Software, desarrollamos un producto mecánico para sexar pollos, pues por muy importante que sea, poco tiene que ver con lo que se ha estudiado durante todos estos años.
\end{itemize}

Y sí, para acordarnos de cuáles son estas características podemos usar el acrónimo %\textit{AMEAR}.
\section{Motivación}

\section{Justificación}
\section{Estructura del documento}

