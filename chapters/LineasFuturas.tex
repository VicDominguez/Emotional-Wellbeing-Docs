\chapter{Líneas futuras}
\label{chapter:lineas}

\chapquote{La victoria más bella es siempre la próxima.}{Enzo Ferrari}

\todo[inline]{Comprobar si hemos hablado del sistema como prototipo}

Al ser el sistema de este proyecto un prototipo, existen numerosas vías para ampliarlo y refinarlo. Algunas de dichas
mejoras son:

\begin{enumerate}
    \item Cifrado de las comunicaciones entre la app y el servidor mediante un certificado SSL en el propio servidor.
    \item Establecimiento de una política de privacidad detallada. Este punto es imprescindibile para poder publicar la
    aplicación (en tiendas virtuales como Play Store) y recoger datos con legalidad.
    \item Desplegar la aplicación en el ámbito universitario. Realizándolo se podría obtener una retroalimentación más
    amplia y profunda de la aplicación, y además se podrían recolectar datos para entrenar
    modelos que predigan cada una de las variables de Salud Mental.
    \item Comprobar la recolección de datos en otras pulseras, tanto de Fitbit como de otros fabricantes.
    \item Adaptar la aplicación para soportar Android 14. Como hemos comentado en la sección 
    \ref{section:salud_conectada}, Android 14 vendrá preinstalado, y Google ha decidido introducir cambios notables, 
    pasando Salud Conectada de ser un APK a un \textit{framework} \cite{noauthor_como_nodate-1}, por lo que se necesita
    ajustar la aplicación para cubrir este caso.
    \item Elaboración de un conjunto de pruebas exahustivo para la aplicación móvil, incluyendo reportes de 
    \textit{coverage} del código.
    \item Establecer que al instalar la aplicación se realice un cuestionario diario. Según la hora del día se elegiría
    si realizar el de mañana o el de noche, pero en cualquier momento del día uno de esos dos.
    \item Dar más presencia a los cuestionarios puntuales, ya sea con gráficas, notificaciones...
    \item Estudiar la imputación de datos nulos con otras técnicas, tales como: regresión, \textit{Last Observation 
    Carried Forward}, \textit{Next Observation Carried Backward}... \cite{gupta_null_nodate}
    \item Realizar agregación de datos para ciertas variables, como las pulsaciones, para aligerar el volumen de datos.
    \item Diseñar e implementar una interfaz gráfica centrada en las pantallas \textit{expandidas}\footnote{Pantallas 
    que disponen de más de 840dp de ancho o 900dp de alto}.
    \item Localizar la aplicación a más idiomas.
    \item Disponer de al menos dos pautas de recomendación a los usuarios para cada variable y estado.
    \item Segmentar la comunidad de usuarios para mejorar la representatividad del conjunto.
\end{enumerate}
