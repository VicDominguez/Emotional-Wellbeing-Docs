\chapter{Marco teórico y contexto tecnológico}
\label{chapter:marco}

\chapquote{La ciencia amigo, está compuesta de errores; pero son errores que es útil cometer ya que nos acercan poco a poco hacia la verdad.}{Julio Verne}

\section{Marco Teórico}


%En este capítulo se introducen los conceptos teóricos sobre los que se asienta el desarrollo de este proyecto. Con el contenido de este capítulo se espera
%crear un base de conocimiento sobre la que desarrollar el resto de este documento. Además, se realiza un análisis de las herramientas que se van a emplear para desarrollar el sistema y cómo estas se encuadran en el contexto tecnológico actual.
%
%\section{Sistemas IoT}
%
%Tal y como se introdujo de manera informal en el capítulo anterior, \textit{IoT} se define como un sistema global e inteligente
%sistema con conciencia global, transmisión fiable
%y procesamiento inteligente de datos %\cite{noauthor_8_2020}.
%
%En este sentido una arquitectura de alto nivel, basada en el conjunto de capas de todos estos objetos interconectados puede verse representada gráficamente en la Figura \ref{fig:IoT_General}.
%
%\begin{figure}[H]
%  \centering
%  \includegraphics[width=0.7\linewidth]{figures/IoT-Architecture-Layers-and-Components.png}
%  \caption{Diagrama general de un sistema IoT}
%  \label{fig:IoT_General}
%\end{figure}


\section{Contexto tecnológico}

    En esta sección se describirán las tecnologías y herramientas más relevantes que se utilizarán a lo
    largo del desarrollo proyecto. El uso de las mismas se describirá en la sección \ref{chapter:desarrollo}.
    \todo{referenciar seccion correspondiente}

    \subsection{Gestión del proyecto}
        \subsubsection{Notion}
        \subsubsection{Microsoft Teams}
        \subsubsection{Zotero}
        \subsubsection{\LaTeX y Overleaf}
        

    \subsection{Ingeniería del Software}
        \subsubsection{Inyección de dependencias}
        \subsubsection{Integración continua}
        \subsubsection{Control de versiones}

    \subsection{Aplicación móvil}

        \subsubsection{Android}

            A grandes rasgos, Android es un Sistema Operativo orientado a dispositivos móviles basado en el núcleo 
            Linux, diseñado para ser independente de la arquitectura hardware de dichos dispositivos. 
            Si bien originalmente fue planteado para teléfonos móviles, con el avance de la industria ha adopado 
            un enfoque más amplio y es compatible con más dispositivos: tabletas, relojes inteligentes, televisores, 
            pantallas de automóviles... aunque, excepto en el caso de las tabletas, se trata de versiones basadas en
            Android con su propia idionsincrasia. \newline

            \begin{figure}[h]
                \centering
                \includegraphics[width=0.25\textwidth]{figures/Android logo.png}
                \caption[Logo actual de Android.]
                {Logo actual de Android. Imagen extraída de \cite{vulcansphere_english_2019}}
                \label{figure:android:logo}
            \end{figure}

            Normalmente cuando nos referimos a Android, no nos referimos únicamente al sistema operativo, sino a la
            plataforma creada entorno al mismo; como haremos a lo largo de este proyecto. Dicha plataforma o 
            \textit{framework} consta de numerosas capas, siendo el sistema operativo una parte de ellas. El sistema 
            operativo como tal es denominado AOSP o \textit{Android Open Source Project}, siendo su código fuente 
            público. Cualquier persona puede acceder a él, descargarlo y modificiarlo \cite{collado_que_2022}.

            \begin{figure}[h]
                \centering
                \includegraphics[width=0.66\textwidth]{figures/Android capas.jpg}
                \caption[Capas de Android.]
                {Capas de Android. Imagen extraída de \cite{perez_aosp_2019}}
                \label{figure:android:capas}
            \end{figure}

            No obstante, en la inmensa mayoría de los teléfonos móviles el sistema operativo es complementado con,
            entre otros, los GMS (\textit{Google Mobile Services}, o servicios de Google), los cuales solo están 
            disponibles bajo licencia; otorgada a los fabricantes que cumplen con una serie de requisitos. Los GMS 
            se utilizan para tareas como la gestión de notificaciones, servicios de geolocalización... además de para 
            acceder a las herramientas de Google, como la tienda de aplicaciones Play Store. Los fabricantes también 
            pueden personalizar y añadir funciones al sistema operativo, lo que explica que dos terminales con la misma 
            versión puedan verse tan diferentes entre sí. \newline
            

            Por otra parte, Android fue inicialmente desarrollado por la empresa homónima, si bien fue comprada en 2005
            por Google por 50 millones de dólares. La salida del sistema operativo se produciría dos años después, el 5 
            de noviembre de 2007, si bien el primer terminal que lo utilizaba (HTC Dream, también conocido como 
            T-Mobile G1) fue comercializado el 23 de septiembre de 2008 \cite{adeva_android_2023} \cite{marquez_asi_2022}.

            \begin{figure}[h]
                \centering
                \includegraphics[width=0.5\textwidth]{figures/HTC Dream.jpg}
                \caption[HTC Dream en funcionamiento.]{HTC Dream en funcionamiento. Imagen extraída de \cite{oryl_t-mobile_2008}}
                \label{figure:android:htc_dream}
            \end{figure}

            Desde entonces, numerosas versiones de Android han sido lanzadas, siendo la última versión estable Android 
            13; estableciéndose por parte de Google la \textit{costumbre} de lanzar cada año una nueva versión principal. 
            En cada una de ellas se introducen nuevas características, pero esto no significa que todos los dispositivos 
            puedan actualizar. Los fabricantes no están obligados a actualizar sus terminales, lo que en la práctica 
            supone que las nuevas versiones no son utilizadas masivamente y que los programadores deben de tener en 
            cuenta las versiones antiguas en sus aplicaciones. 
            \newline

            Debido a que Google dejó de publicar oficialmente las estadísticas de uso de su sistema operativo, no es
            posible conocer con plena exactitud dichas cifras. La comunidad se ha encargado de estimar dicha 
            información \cite{belinski_android_nodate}; relevando que a fecha de mayo de 2023 sólo el 20\% de los 
            dispositivos tienen la última versión, mientras que las versiones 12,11 y 10 están presentes en el 
            20,8\%, 21,1\% y 16,6\% respectivamente. \newline

            \begin{figure}[H]
                \centering
                \includegraphics[width=1\textwidth]{figures/Android usage.PNG}
                \caption[Estadísiticas acumulativas de las versiones de Android]
                {Estadísiticas acumulativas de las versiones de Android. Imagen extraída de \cite{belinski_android_nodate}}
                \label{figure:android:usage}
            \end{figure}

            Por último, a fecha de marzo de 2023, Android dispone de una cuota de mercado del 71\% en el segmento de sistemas 
            operativos para dispositivos móviles, teniendo su mayor rival,el sistema operativo iOS (propiedad de Apple) 
            un 28\%. Entre ambos acaparan el mercado, con un 99\% de cuota de mercado. En cuanto a España, el 
            porcentaje de Android asciende hasta el 77,73\% por el 21,81 de iOS \cite{press_asi_2023}.

        \subsubsection{Kotlin}

            Durante el diseño de Android se estableció que el lenguaje principal para desarrollar aplicaciones sería 
            Java, si bien incorpora soporte para utilizar código C y C++ \cite{android_developers_como_nodate}. No
            obstante, al ser Java un lenguaje interpretrado sobre una máquina virtual (JVM o \textit{Java Virtual
            Machine}) se abrió la puerta para  utilizar otros lenguajes que utilizasen la JVM. En la conferencia de 
            Google \textit{I/O} de 2017 fue anunciado el soporte oficial y completo 
            de Kotlin dentro de Android. \newline

            Kotlin es un lenguaje de programación desarrollado por JetBrains\footnote{JetBrains es una
            empresa muy reconocida dentro de la industria por crear una serie de entornos de desarrollo muy populares,
            como PHPStorm, CLion o Intellij IDEA. Sobre este último está construido Android Studio, el entorno de 
            desarrollo oficial dentro de Android.} y publicada su primera versión estable el 15 de febrero de 2016.
            El objetivo de este lenguaje fue tan sencillo como ambicioso: crear un lenguaje conciso (permitiendo reducir
            la cantidad de código inútil), con soporte de nuevas funcionalidades; pero sin renunciar a la rapidez de
            compilación de Java ni al todo el código escrito en él 
            \cite{rao_k_history_nodate}. \newline

            Sus principales características son las siguientes \cite{noauthor_kotlin_nodate} \cite{noauthor_enfoque_nodate}:
            \begin{itemize}
                \item Interoperable al 100\% con Java, lo que facilita la reutilización de código ya existentes. 
                Es interoperable en ambos sentidos.
                \item Permite escribir código más seguro, ya que en el diseño del lenguaje se solucionaron problemas
                crónicos de Java como las \textit{Null Pointer Exception}. Según datos internos de Google, las 
                aplicaciones escritas en Kotlin tienen un 20\% de probabilidades menos de fallar.
                \item Soporte nativo y estructurado para la programación concurrente y asíncrona mediante 
                construcciones como las corrutinas y los flujos.
                \item Desarrollo multiplataforma, no solo para Android: aplicaciones web, \textit{backend} e iOS.
                \item Permite desarrollos en varios paradigmas: orientada a objetos, funcional, imperativa...
            \end{itemize}
            
            Asimismo, al convertirse en la \textit{I/O} de 2019 en el lenguaje de referencia para el desarrollo 
            de Android \cite{braun_celebrating_2022}, los desarrollos de librerías y herramientas relacionadas con 
            Android están escritas en este lenguaje, aprovechando al máximo sus nuevas características. Por tanto, si
            bien es interoperable con Java, está recomendado que los nuevos desarrollos lo utilicen 
            \cite{lardinois_kotlin_2019}, como se ha realizado en este proyecto.
            

        \subsubsection{Jetpack Compose}

            Jetpack Compose es un conjunto de herramientas \textit{modernas} de Android para el desarrollo de 
            interfaces gráficas, lanzado en su primera versión estable el 28 de julio de 2021 por Google
            \cite{bellini_jetpack_2021}. Este kit de librerías permite desarrollar en el ecosistema Android de 
            forma nativa interfaces gráficas de manera declarativa como en los sistemas React, Flutter o SwiftUI;
            siguiendo las tendencias actuales de la industria en el desarrollo de aplicaciones móviles. \newline

            Este enfoque declarativo nos permite describir cómo queremos que sea nuestra interfaz gráfica. 
            Asimismo, las interfaces que construimos con este sistema pueden estar interconectadas a un estado 
            que definamos; describiendo cómo será nuestra interfaz para cada posible estado. Cuando ese estado cambie, 
            nuestra interfaz gráfica cambiará automáticamente para mostrar el nuevo estado, simplificando enormemente 
            el desarrollo y reduciendo el código necesario \cite{leiva_que_2021}. \newline

            Hasta la aparición de Jetpack Compose, el desarrollo interfaces gráficas nativas en Android se realizaba 
            con el enfoque conocido como programación imperativa. En este tipo de desarrollo es necesario especificar 
            paso por paso cómo se va a construir dicha interfaz gráfica exahustivamente. En dicho proceso (conocido 
            en Android como sistema de vistas) se codificaba un fichero XML, en el que se describían todos los 
            elementos gráficos (botones, textos...); para en el código Java/Kotlin de la aplicación se accediera 
            a dichos elementos y se le aplicaran manualmente modificaciones y transformaciones 
            \cite{noauthor_programacion_2021}. \newline

            Además, en Jetpack Compose, a diferencia del sistema anterior, los componentes gráficos están desacoplados 
            del sistema operativo; por lo que no dependemos de la versión del terminal para mostrar correctamente 
            nuestra interfaz gráfica. Eso ocurría anteriormente y como ya vimos, la fragmentación en Android es un 
            problema endémico, lo que complicaba bastante el desarrollo. Asimismo, es compatible con los componentes XML 
            del sistema anterior, lo que facilita la migración de los proyectos antiguos a este nuevo paradigma. \newline

            No obstante, como ya vimos en el apartado anterior, está diseñado para ser utilizado desde Kotlin, por lo 
            que en la práctica obliga a usar dicho lenguaje; lo que en algunos casos puede resultar en un pico de 
            dificultad hasta que se domina el lenguaje.
        

        \subsubsection{Material Design 3}
            Material Design 3 (también conocido como \textit{Material You}) es la tercera iteración del 
            conjunto de principios y directrices de diseño de Google, 
            como respuesta a la creciente ubiquidad de Android: móviles con pantallas
            plegables, \textit{smartwatch}, televisores \cite{ramirez_que_2022}... 
            Su primera implementación estable para Jetpack Compose fue lanzada el 
            24 de octubre de 2022 \cite{singh_material_2022}. \newline
            
            Al ser utilizado por Google para la creación de elementos 
            gráficos tanto en sus aplicaciones como en el sistema operativo, es la guía de diseño de facto dentro del
            ecosistema Android. \newline

            Sus principales características son las siguientes \cite{noauthor_material_nodate}:
            \begin{itemize}
                \item Centrado en la personalización de la interfaz gráfica. Los diseñadores definirán tres colores
                principales, los cuales serán utilizados para los elementos gráficos de forma totalmente transparente
                al programador. A partir de dichos colores, se elaborará mediante la herramienta 
                \textit{Material Design Builder} \cite{noauthor_material_nodate-1} una paleta de colores con 
                variantes de los mismos, 
                diseñada para cumplir estándares de accesibilidad\footnote{Describir dichos
                estándares está fuera del alcance del proyecto, pero es un proceso basado en proporciones de 
                luminitancia}, garantizando el nivel de contraste correcto.

                    \begin{figure}[h]
                        \centering
                        \includegraphics[width=0.75\textwidth]{figures/Material Design Builder example.png}
                        \caption[Ejemplo de uso de la herramienta \textit{Material Design Builder}]
                        {Ejemplo de uso de la herramienta \textit{Material Design Builder}. Imagen extraída de \cite{singh_material_2022}}
                        \label{figure:material_design_3:builder}
                    \end{figure}

                Además, si el dispositivo dispone de Android 12 o superior, pueden tomarse dichos
                colores desde el fondo de pantalla del usuario, incrementando exponencialmente la personalización;
                si bien se permite establecer colores \textit{fijos} para ciertos contenidos.
                \item Soporte nativo para categorizar el tamaño de la pantalla del dispositivo, tanto en altura como
                en anchura.
                    \begin{figure}[h]
                        \centering
                        \includegraphics[width=0.75\textwidth]{figures/Tamaños de ventana.png}
                        \caption[Categorías de pantalla según anchura]
                        {Categorías de pantalla según anchura. Imagen extraída de \cite{singh_material_2022}}
                        \label{figure:material_design_3:width_classes}
                    \end{figure}
                \item Sistema de fuentes basado en estilos principales para cada tipo de contenido: desde titulares 
                hasta etiquetas, pasando por títulos, cuerpos de texto...
                \item Soporte nativo para animaciones, las cuales ya son utilizadas en los componentes gráficos nativos,
                como los \textit{switch}.
                \item Evolución de muchos elementos gráficos, como las tarjetas, botones, selectores de fechas...
                    \begin{figure}[h]
                        \centering
                        \includegraphics[width=0.75\textwidth]{figures/Elementos gráficos material design 3.png}
                        \caption[Algunos elementos gráficos de Material Design 3]
                        {Algunos elementos gráficos de Material Design 3. Imagen extraída de \cite{cerda_material_2022}}
                        \label{figure:material_design_3:elementos_graficos}
                    \end{figure}
                \item Sistema de formas o \textit{redondeo} multinivel para modernizar nuestros elementos y hacerlos
                más distingubles entre sí.
        
                \item Mejora el concepto conocido como \textit{elevación}, basándose en colores y no en sombras. Este
                elemento permite superponer elementos y transmitir gráfica y visualmente la importancia de cada 
                uno de ellos.
                \item Soporte nativo para tema claro y oscuro, ya que en los últimos años los temas oscuros 
                se han hecho cada vez más populares en las aplicaciones y no estaba presente previamente de forma nativa.
            \end{itemize}

            En pocas palabras, en esta versión se han dedicado a modernizar el lenguaje de diseño, haciéndolo más 
            atractivo y completo; alejándose de lo puramente funcional, mejorando la accesibilidad y explorando 
            el mundo de la personalización para el usuario.

        \subsubsection{Salud Conectada}
            En la conferencia \textit{I/O} (sí, otra vez) de 2022 se anunció \textit{Health Connect} (o Salud Conectada
            en castellano), una aplicación creada por Google (junto con Samsung \cite{wilk_introducing_2022}) 
            que aglutinará todos los datos relacionados con salud dentro del ecosistema
            Android. La aplicación (desde el 11 de noviembre de 2022 en estado beta) es compatible con Android 9 o 
            superior, mientras que está previsto que para Android 14 venga
            incorporada aplicación de fábrica o preinstalada\footnote{Recordar aquí que esta aplicación será
            parte de los servicios de Google y no del sistema operativo propiamente dicho o AOSP.} 
            \cite{pandey_health_2023}. \newline

            \begin{figure}[h]
                \centering
                \includegraphics[width=0.25\textwidth]{figures/Health connect logo.png}
                \caption[Logo de Salud Conectada.]
                {Logo de Salud Conectada. Imagen extraída de \cite{noauthor_health_nodate}}
                \label{figure:health_connect:logo}
            \end{figure}

            Esta herramienta viene a solucionar una probelmática importante y es la ausencia del reaprovechamiento de 
            los datos, ya que hasta ese momento, la posición casi unánime dentro la industria se basaba en
            que los datos recoletados por sus dispositivos \textit{wearables} o aplicación se quedaran en su
            ecosistema \cite{ramirez_android_2022} \cite{rahman_android_2023}. \newline
            
            La única manera utilizar dichos datos 
            en otras aplicaciones estaba restringuida a sistemas propietarios como Google Fit, cuyos datos se 
            almacenaban enla nube con los problemas de privacidad asociados, o proyectos \textit{open source} como 
            Gadget Bridge \cite{freeyourgadget_gadgetbridge_nodate}; que accedían a datos mediante ingeniería inversa 
            de pulseras como las Xiaomi Mi Band. \newline

            A fecha de mayo de 2023, se estima que más de 100 aplicaciones han integrado Salud Conectada, incluyendo
            las aplicaciones de las pulseras Fitbit y Samsung, Peloton, Oura \cite{malik_googles_2023}... \newline

            Centrándonos en Salud Conectada, se trata tanto de una plataforma como una API para los desarrolladores.
            La plataforma puede registrar datos como la actividad física, el sueño, la nutrición o incluso el ciclo
            menstrual, siendo un intermediario entre las aplicaciones que generan o escriben dichos datos y las 
            que quieren acceder a esos datos. \newline

            \begin{figure}[h]
                \centering
                \includegraphics[width=0.33\textwidth]{figures/Arquitectura básica de Health Connect.png}
                \caption[Arquitectura básica de Salud Conectada.]
                {Arquitectura básica de  Salud Conectada. Imagen extraída de \cite{wilk_introducing_2022}}
                \label{figure:health_connect:logo}
            \end{figure}
            
            Asimismo, está diseñada con la privacidad en mente: los datos se almacenan localmente, mientras que el 
            acceso a los mismos esta fuertemente granularizado: en la que el usuario puede
            decidir qué aplicaciones tienen acceso (tanto lectura como escritura) a cada tipo de registro 
            \cite{saez_google_2022}.

            

        \subsubsection{Room}

    
    \subsection{Servidor}
        \subsubsection{Python}

        \subsubsection{Flask}

        \subsubsection{MongoDB}
