Capítulo 9 - Bibliografía
Accelerating AI with GPUs: A New Computing Model | NVIDIA Blog. (2016, enero 12). The Official NVIDIA Blog. https://blogs.nvidia.com/blog/2016/01/12/accelerating-ai-artificial-intelligence-gpus/
Aff-Wild2 database. (s. f.). Recuperado 20 de junio de 2021, de https://ibug.doc.ic.ac.uk/resources/aff-wild2/
Albert, P. R. (2015). Why is depression more prevalent in women? Journal of Psychiatry & Neuroscience: JPN, 40(4), 219-221. https://doi.org/10.1503/jpn.150205
Albrecht, A. T., & Herrick, C. (2007). 100 preguntas y respuestas sobre la depresión. EDAF.
Alfonso Ibañez. (2019, mayo 30). Semi-Supervised Learning… the great unknown. Think Big. https://business.blogthinkbig.com/semi-supervised-learning-the-great-unknown/
Alghowinem, S., Goecke, R., Wagner, M., Epps, J., Hyett, M., Parker, G., & Breakspear, M. (2016). Multimodal Depression Detection: Fusion Analysis of Paralinguistic, Head Pose and Eye Gaze Behaviors. IEEE Transactions on Affective Computing, PP, 1-1. https://doi.org/10.1109/TAFFC.2016.2634527
Alonso Romero, L. (s. f.). Redes Neuronales. Recuperado 24 de agosto de 2020, de http://avellano.fis.usal.es/~lalonso/RNA/index.htm
AlphaGo. (2019). En Wikipedia, la enciclopedia libre. https://es.wikipedia.org/w/index.php?title=AlphaGo&oldid=119976652
AlphaGo Zero: Starting from scratch. (2017, octubre 18). Deepmind. /blog/article/alphago-zero-starting-scratch
American Psychiatric Association. (2010a). Practice Guideline for the Treatment of Patients With Major Depressive Disorder (Tercera). American Journal of Psychiatry. https://www.apa.org/depression-guideline/guideline.pdf
American Psychiatric Association. (2010b). Diagnostic and Statistical Manual of Mental Disorders, Fourth Edition, Text Revision (DSM-IV-TR®). American Psychiatric Association Publishing.
American Psychiatric Association. (2013). Diagnostic and Statistical Manual of Mental Disorders (Fifth Edition). American Psychiatric Association. https://doi.org/10.1176/appi.books.9780890425596
Amidi, A., & Amidi, S. (2018, noviembre 26). CS 230—Recurrent Neural Networks Cheatsheet. https://stanford.edu/~shervine/teaching/cs-230/cheatsheet-recurrent-neural-networks
Amos, B., Ludwiczuk, B., & Satyanarayanan, M. (2016). OpenFace: A general-purpose face recognition library with mobile applications. CMU-CS-16-118, CMU School of Computer Science.
Anderson, R. J., Freedland, K. E., Clouse, R. E., & Lustman, P. J. (2001). The Prevalence of Comorbid Depression in Adults With Diabetes: A meta-analysis. Diabetes Care, 24(6), 1069-1078. https://doi.org/10.2337/diacare.24.6.1069
Apriori algorithm. (2020). En Wikipedia. https://en.wikipedia.org/w/index.php?title=Apriori_algorithm&oldid=961797201
Artificial Neural Network. (2018, abril 23). NVIDIA Developer. https://developer.nvidia.com/discover/artificial-neural-network
Asiri, S. (2018, junio 11). Machine Learning Classifiers. Medium. https://towardsdatascience.com/machine-learning-classifiers-a5cc4e1b0623
AudioVisualEmotionChallenge. (2021). AudioVisualEmotionChallenge/AVEC2019 [Python]. https://github.com/AudioVisualEmotionChallenge/AVEC2019 (Original work published 2019)
Bahdanau, D., Cho, K., & Bengio, Y. (2016). Neural Machine Translation by Jointly Learning to Align and Translate. arXiv:1409.0473 [cs, stat]. http://arxiv.org/abs/1409.0473
Bakator, M., & Radosav, D. (2018). Deep Learning and Medical Diagnosis: A Review of Literature. Multimodal Technologies and Interaction, 2(3), 47. https://doi.org/10.3390/mti2030047
Baltrusaitis, T. (2021). TadasBaltrusaitis/OpenFace [MATLAB]. https://github.com/TadasBaltrusaitis/OpenFace (Original work published 2016)
Barbosa, A. M. (2020, enero 17). Area under the precision-recall curve. R-Bloggers. https://www.r-bloggers.com/2020/01/area-under-the-precision-recall-curve/
Barchas, J. D., & Altemus, M. (1999). Monoamine Hypotheses of Mood Disorders. Basic Neurochemistry: Molecular, Cellular and Medical Aspects. 6th Edition. https://www.ncbi.nlm.nih.gov/books/NBK28257/
Basic Linear Algebra Subprograms. (2017, noviembre 14). http://www.netlib.org/blas/
Beck, A. T., & Alford, B. A. (2009). Depression: Causes and Treatment. University of Pennsylvania Press, Incorporated. https://books.google.es/books?id=Ntw8AwAAQBAJ
Beck Depression Inventory. (2020). En Wikipedia. https://en.wikipedia.org/w/index.php?title=Beck_Depression_Inventory&oldid=951359029
Belmaker, R. H., & Agam, G. (2008). Major Depressive Disorder. New England Journal of Medicine, 358(1), 55-68. https://doi.org/10.1056/NEJMra073096
Bengio, Y., Simard, P., & Frasconi, P. (1994). Learning long-term dependencies with gradient descent is difficult. IEEE Transactions on Neural Networks, 5(2), 157-166. https://doi.org/10.1109/72.279181
Bennabi, D., Charpeaud, T., Yrondi, A., Genty, J.-B., Destouches, S., Lancrenon, S., Alaïli, N., Bellivier, F., Bougerol, T., Camus, V., Dorey, J.-M., Doumy, O., Haesebaert, F., Holtzmann, J., Lançon, C., Lefebvre, M., Moliere, F., Nieto, I., Rabu, C., … Haffen, E. (2019). Clinical guidelines for the management of treatment-resistant depression: French recommendations from experts, the French Association for Biological Psychiatry and Neuropsychopharmacology and the fondation FondaMental. BMC Psychiatry, 19(1), 262. https://doi.org/10.1186/s12888-019-2237-x
BERT (sistema computacional de comprensión de lenguaje). (2020). En Wikipedia, la enciclopedia libre. https://es.wikipedia.org/w/index.php?title=BERT_(sistema_computacional_de_comprensi%C3%B3n_de_lenguaje)&oldid=129286911
Binary Classification—Amazon Machine Learning. (s. f.). Recuperado 8 de septiembre de 2020, de https://docs.aws.amazon.com/machine-learning/latest/dg/binary-classification.html
Black Dog Institute. (2020). En Wikipedia. https://en.wikipedia.org/w/index.php?title=Black_Dog_Institute&oldid=965744814
Blockeel, H. (2010). Hypothesis Space. En C. Sammut & G. I. Webb (Eds.), Encyclopedia of Machine Learning (pp. 511-513). Springer US. https://doi.org/10.1007/978-0-387-30164-8_373
Boucher, P. (2019). How artificial intelligence works. Panel for the Future of Science and Technology, 10.
Brownlee, J. (2014, noviembre 4). How To Get Baseline Results And Why They Matter. Machine Learning Mastery. https://machinelearningmastery.com/how-to-get-baseline-results-and-why-they-matter/
Brownlee, J. (2019, febrero 26). How to use Learning Curves to Diagnose Machine Learning Model Performance. Machine Learning Mastery. https://machinelearningmastery.com/learning-curves-for-diagnosing-machine-learning-model-performance/
Buch, A. M., & Liston, C. (2020). Dissecting diagnostic heterogeneity in depression by integrating neuroimaging and genetics. Neuropsychopharmacology, 1-22. https://doi.org/10.1038/s41386-020-00789-3
Budhiraja, A. (2018, marzo 6). Learning Less to Learn Better—Dropout in (Deep) Machine learning. Medium. https://medium.com/@amarbudhiraja/https-medium-com-amarbudhiraja-learning-less-to-learn-better-dropout-in-deep-machine-learning-74334da4bfc5
Burcusa, S. L., & Iacono, W. G. (2007). Risk for recurrence in depression. Clinical Psychology Review, 27(8), 959-985. https://doi.org/10.1016/j.cpr.2007.02.005
Chambon, S., Galtier, M. N., Arnal, P. J., Wainrib, G., & Gramfort, A. (2018). A Deep Learning Architecture for Temporal Sleep Stage Classification Using Multivariate and Multimodal Time Series. IEEE Transactions on Neural Systems and Rehabilitation Engineering, 26(4), 758-769. https://doi.org/10.1109/TNSRE.2018.2813138
Chapelle, O., Schölkopf, B., & Zien, A. (Eds.). (2006). Semi-supervised learning. MIT Press.
Chartrand, G., Cheng, P. M., Vorontsov, E., Drozdzal, M., Turcotte, S., Pal, C. J., Kadoury, S., & Tang, A. (2017). Deep Learning: A Primer for Radiologists. Radiographics: A Review Publication of the Radiological Society of North America, Inc, 37(7), 2113-2131. https://doi.org/10.1148/rg.2017170077
Chauhan, N. S. (2020, junio 17). Market Basket Analysis. Medium. https://towardsdatascience.com/market-basket-analysis-978ac064d8c6
Chawla, N. V., Bowyer, K. W., Hall, L. O., & Kegelmeyer, W. P. (2002). SMOTE: Synthetic Minority Over-sampling Technique. Journal of Artificial Intelligence Research, 16, 321-357. https://doi.org/10.1613/jair.953
Chollet, F. (2018). Deep learning with Python (Segunda). Manning Publications Co. https://www.manning.com/books/deep-learning-with-python-second-edition
Clark, K., Luong, M.-T., Le, Q. V., & Manning, C. D. (2019, septiembre 25). ELECTRA: Pre-training Text Encoders as Discriminators Rather Than Generators. International Conference on Learning Representations. https://openreview.net/forum?id=r1xMH1BtvB
Clarke, D. M., & Currie, K. C. (2009). Depression, anxiety and their relationship with chronic diseases: A review of the epidemiology, risk and treatment evidence. The Medical Journal of Australia, 190(S7), S54-60.
Clasificación: Exactitud | Curso intensivo de aprendizaje automático. (2020, febrero 10). Google Developers. https://developers.google.com/machine-learning/crash-course/classification/accuracy?hl=es-419
Clasificación: Precisión y exhaustividad. (2020, febrero 10). https://developers.google.com/machine-learning/crash-course/classification/precision-and-recall?hl=es-419
Clasificación: ROC y AUC | Curso intensivo de aprendizaje automático. (2020, febrero 10). Google Developers. https://developers.google.com/machine-learning/crash-course/classification/roc-and-auc?hl=es-419
Clasificación: Sesgo de predicción. (2020, febrero 2). Google Developers. https://developers.google.com/machine-learning/crash-course/classification/prediction-bias?hl=es-419
Clasificación: Umbral | Curso intensivo de aprendizaje automático. (2020, febrero 10). Google Developers. https://developers.google.com/machine-learning/crash-course/classification/thresholding?hl=es-419
Clasificación: Verdadero o falso y positivo o negativo. (2020, febrero 10). Google Developers. https://developers.google.com/machine-learning/crash-course/classification/true-false-positive-negative?hl=es-419
Clinical depression—Psychotic depression. (2017, octubre 20). Nhs.Uk. https://www.nhs.uk/conditions/clinical-depression/psychotic-depression/
Clipping (audio). (2021). En Wikipedia. https://en.wikipedia.org/w/index.php?title=Clipping_(audio)&oldid=1028204513
Clustering Algorithms | Clustering in Machine Learning. (2020, febrero 10). Google Developers. https://developers.google.com/machine-learning/clustering/clustering-algorithms?hl=es-419
Comparison of deep-learning software. (2020). En Wikipedia. https://en.wikipedia.org/w/index.php?title=Comparison_of_deep-learning_software&oldid=971790766
Confusion matrix. (2020). En Wikipedia. https://en.wikipedia.org/w/index.php?title=Confusion_matrix&oldid=976567428
Cuijpers, P., Berking, M., Andersson, G., Quigley, L., Kleiboer, A., & Dobson, K. S. (2013). A Meta-Analysis of Cognitive-Behavioural Therapy for Adult Depression, Alone and in Comparison With Other Treatments. Canadian Journal of Psychiatry, 58(7), 376-385. https://doi.org/10.1177/070674371305800702
Cuijpers, P., Donker, T., Weissman, M. M., Ravitz, P., & Cristea, I. A. (2016). Interpersonal Psychotherapy for Mental Health Problems: A Comprehensive Meta-Analysis. American Journal of Psychiatry, 173(7), 680-687. https://doi.org/10.1176/appi.ajp.2015.15091141
Cummins, N., Epps, J., Breakspear, M., & Goecke, R. (2011). An Investigation of Depressed Speech Detection: Features and Normalization. 4.
Curva ROC. (2020). En Wikipedia, la enciclopedia libre. https://es.wikipedia.org/w/index.php?title=Curva_ROC&oldid=128048916
Dawani, J. (2020). Hands-On Mathematics for Deep Learning: Build a Solid Mathematical Foundation for Training Efficient Deep Neural Networks. Packt Publishing. https://books.google.es/books?id=Ujl1zQEACAAJ
Deep Learning in Fashion. (2017, noviembre 1). https://experiencesutra.com/experiments/deep-learning-in-fashion/
Delgado, P. L. (2000). Depression: The case for a monoamine deficiency. The Journal of Clinical Psychiatry, 61 Suppl 6, 7-11.
Deng, J., Dong, W., Socher, R., Li, L., Kai Li, & Li Fei-Fei. (2009). ImageNet: A large-scale hierarchical image database. 2009 IEEE Conference on Computer Vision and Pattern Recognition, 248-255. https://doi.org/10.1109/CVPR.2009.5206848
Depresión y conducta suicida. (2017, febrero 22). Comunidad de Madrid. https://www.comunidad.madrid/servicios/salud/depresion-conducta-suicida
DeRubeis, R. J., Siegle, G. J., & Hollon, S. D. (2008). Cognitive therapy vs. medications for depression: Treatment outcomes and neural mechanisms. Nature reviews. Neuroscience, 9(10), 788-796. https://doi.org/10.1038/nrn2345
Despolarización. (2020). En Wikipedia, la enciclopedia libre. https://es.wikipedia.org/w/index.php?title=Despolarizaci%C3%B3n&oldid=127203991
DeVault, D., Artstein, R., Benn, G., Dey, T., Fast, E., Gainer, A., Georgila, K., Gratch, J., Hartholt, A., Lhommet, M., Lucas, G., Marsella, S., Morbini, F., Nazarian, A., Scherer, S., Stratou, G., Suri, A., Traum, D., Wood, R., … Morency, L.-P. (2014). SimSensei kiosk: A virtual human interviewer for healthcare decision support. Proceedings of the 2014 international conference on Autonomous agents and multi-agent systems, 1061-1068.
Devlin, J., & Chang, M.-W. (2018, noviembre 2). Open Sourcing BERT: State-of-the-Art Pre-training for Natural Language Processing. Google AI Blog. http://ai.googleblog.com/2018/11/open-sourcing-bert-state-of-art-pre.html
Devlin, J., Chang, M.-W., Lee, K., & Toutanova, K. (2019). BERT: Pre-training of Deep Bidirectional Transformers for Language Understanding. arXiv:1810.04805 [cs]. http://arxiv.org/abs/1810.04805
Di Trani, M., Di Roma, F., Elda, A., Daniela, L., Pasquale, P., Silvia, M., & Renato, D. (2014). Comorbid Depressive Disorders in ADHD: The Role of ADHD Severity, Subtypes and Familial Psychiatric Disorders. Psychiatry Investigation, 11(2), 137-142. https://doi.org/10.4306/pi.2014.11.2.137
Dinkel, H., Wu, M., & Yu, K. (2019). Text-based Depression Detection: What Triggers An Alert. https://www.arxiv-vanity.com/papers/1904.05154/
Discrete cosine transform. (2021). En Wikipedia. https://en.wikipedia.org/w/index.php?title=Discrete_cosine_transform&oldid=1029745869
Disease burden. (2020). En Wikipedia. https://en.wikipedia.org/w/index.php?title=Disease_burden&oldid=981871560
Dockrill, P. (2017, diciembre 7). In Just 4 Hours, Google’s AI Mastered All The Chess Knowledge in History. ScienceAlert. https://www.sciencealert.com/it-took-4-hours-google-s-ai-world-s-best-chess-player-deepmind-alphazero
Drevets, W. C. (2000). Neuroimaging studies of mood disorders. Biological Psychiatry, 48(8), 813-829. https://doi.org/10.1016/s0006-3223(00)01020-9
Driessen, E., & Hollon, S. D. (2010). Cognitive Behavioral Therapy for Mood Disorders: Efficacy, Moderators and Mediators. The Psychiatric clinics of North America, 33(3), 537-555. https://doi.org/10.1016/j.psc.2010.04.005
Ekman, P., Rosenberg, P. E. E. L., Ekman, R., Rosenberg, E. L., Rosenberg, L. D. P. E. L., Smith, M. B., & Professor of Psychology Paul Ekman, P. H. D. (1997). What the Face Reveals: Basic and Applied Studies of Spontaneous Expression Using the Facial Action Coding System (FACS). Oxford University Press. https://books.google.es/books?id=fFGYs079-7YC
Electrical synapse. (2020). En Wikipedia. https://en.wikipedia.org/w/index.php?title=Electrical_synapse&oldid=969873937
Ellgring, H., & Scherer, K. R. (1996). Vocal indicators of mood change in depression. Journal of Nonverbal Behavior, 20(2), 83-110. https://doi.org/10.1007/BF02253071
Elsevier. (2018, enero 23). Estructura y tipos de neuronas. Elsevier Connect. https://www.elsevier.com/es-es/connect/medicina/estructura-y-tipos-de-neuronas
Ensemble learning. (2020). En Wikipedia. https://en.wikipedia.org/w/index.php?title=Ensemble_learning&oldid=977960246
Evans, E. A., & Sullivan, M. A. (2014). Abuse and misuse of antidepressants. Substance Abuse and Rehabilitation, 5, 107-120. https://doi.org/10.2147/SAR.S37917
Eyben, F., Scherer, K. R., Schuller, B. W., Sundberg, J., André, E., Busso, C., Devillers, L. Y., Epps, J., Laukka, P., Narayanan, S. S., & Truong, K. P. (2016). The Geneva Minimalistic Acoustic Parameter Set (GeMAPS) for Voice Research and Affective Computing. IEEE Transactions on Affective Computing, 7(2), 190-202. https://doi.org/10.1109/TAFFC.2015.2457417
Eyben, F., & Schuller, B. (2015). openSMILE:): The Munich open-source large-scale multimedia feature extractor. ACM SIGMultimedia Records, 6(4), 4-13. https://doi.org/10.1145/2729095.2729097
F1 score. (2020). En Wikipedia. https://en.wikipedia.org/w/index.php?title=F1_score&oldid=976442604
Farrús, M., Hernando, J., & Ejarque, P. (s. f.). Jitter and Shimmer Measurements for Speaker Recognition. 4.
Fava, M., & Kendler, K. S. (2000). Major depressive disorder. Neuron, 28(2), 335-341. https://doi.org/10.1016/s0896-6273(00)00112-4
Feature scaling. (2020). En Wikipedia. https://en.wikipedia.org/w/index.php?title=Feature_scaling&oldid=973370127
First, M. B. (2013). DSM-5 Handbook of Differential Diagnosis. American Psychiatric Publishing, a division of American Psychiatric Association. https://dsm.psychiatryonline.org/doi/book/10.1176/appi.books.9781585629992
Francesco Mosconi. (2019). Zero to Deep Learning. https://book.zerotodeeplearning.com/
Freeman, A., Tyrovolas, S., Koyanagi, A., Chatterji, S., Leonardi, M., Ayuso-Mateos, J. L., Tobiasz-Adamczyk, B., Koskinen, S., Rummel-Kluge, C., & Haro, J. M. (2016). The role of socio-economic status in depression: Results from the COURAGE (aging survey in Europe). BMC Public Health, 16. https://doi.org/10.1186/s12889-016-3638-0
Furio, E. (2009, mayo 24). Comportamiento animal y tipos de aprendizaje. Kopher’s blog. https://kopher.wordpress.com/2009/05/24/comportamiento-animal-y-tipos-de-aprendizaje/
Gandhi, A. (2018, noviembre 15). Data Augmentation | How to use Deep Learning when you have Limited Data. AI & Machine Learning Blog. https://nanonets.com/blog/data-augmentation-how-to-use-deep-learning-when-you-have-limited-data-part-2/
Garver, D. L., & Davis, J. M. (1979). Biogenic amine hypothesis of affective disorders. Life Sciences, 24(5), 383-394. https://doi.org/10.1016/0024-3205(79)90208-X
Géron, A. (2019). Hands-On Machine Learning with Scikit-Learn, Keras, and TensorFlow: Concepts, Tools, and Techniques to Build Intelligent Systems. O’Reilly Media. https://learning.oreilly.com/library/view/hands-on-unsupervised-learning/9781492035633/
Ghio, L., Gotelli, S., Cervetti, A., Respino, M., Natta, W., Marcenaro, M., Serafini, G., Vaggi, M., Amore, M., & Belvederi Murri, M. (2015). Duration of untreated depression influences clinical outcomes and disability. Journal of Affective Disorders, 175, 224-228. https://doi.org/10.1016/j.jad.2015.01.014
Gilbody, S. M., House, A. O., & Sheldon, T. A. (2002). Psychiatrists in the UK do not use outcomes measures. National survey. The British Journal of Psychiatry: The Journal of Mental Science, 180, 101-103. https://doi.org/10.1192/bjp.180.2.101
Giordano, A. (2021). Multi-modal Analysis of Emotions in Multimedia [Master Dissertation]. University of Naples Federico II.
Glassman, A. H. (2007). Depression and cardiovascular comorbidity. Dialogues in Clinical Neuroscience, 9(1), 9-17.
Glorot, X., & Bengio, Y. (2010). Understanding the difficulty of training deep feedforward neural networks. 8.
Goodfellow, I., Bengio, Y., & Courville, A. (2016). Deep Learning. MIT Press.
Gratch, J., Artstein, R., Lucas, G., Stratou, G., Scherer, S., Nazarian, A., Wood, R., Boberg, J., DeVault, D., Marsella, S., Traum, D., Rizzo, S., & Morency, L.-P. (s. f.). The Distress Analysis Interview Corpus of human and computer interviews. 6.
Gray, V., Douglas, K. M., & Porter, R. J. (2021). Emotion processing in depression and anxiety disorders in older adults: Systematic review. BJPsych Open, 7(1), e7. https://doi.org/10.1192/bjo.2020.143
Gzip. (2020). En Wikipedia, la enciclopedia libre. https://es.wikipedia.org/w/index.php?title=Gzip&oldid=122683658
Hahn, T., Marquand, A. F., Ehlis, A.-C., Dresler, T., Kittel-Schneider, S., Jarczok, T. A., Lesch, K.-P., Jakob, P. M., Mourao-Miranda, J., Brammer, M. J., & Fallgatter, A. J. (2011). Integrating neurobiological markers of depression. Archives of General Psychiatry, 68(4), 361-368. https://doi.org/10.1001/archgenpsychiatry.2010.178
Halfin, A. (2007). Depression: The benefits of early and appropriate treatment. The American Journal of Managed Care, 13(4 Suppl), S92-97.
Hamilton Rating Scale for Depression. (2020). En Wikipedia. https://en.wikipedia.org/w/index.php?title=Hamilton_Rating_Scale_for_Depression&oldid=963928060
Hawton, K., Casañas i Comabella, C., Haw, C., & Saunders, K. (2013). Risk factors for suicide in individuals with depression: A systematic review. Journal of Affective Disorders, 147(1), 17-28. https://doi.org/10.1016/j.jad.2013.01.004
He, K., Zhang, X., Ren, S., & Sun, J. (2015a). Deep Residual Learning for Image Recognition. arXiv:1512.03385 [cs]. http://arxiv.org/abs/1512.03385
He, K., Zhang, X., Ren, S., & Sun, J. (2015b). Delving Deep into Rectifiers: Surpassing Human-Level Performance on ImageNet Classification. 2015 IEEE International Conference on Computer Vision (ICCV), 1026-1034. https://doi.org/10.1109/ICCV.2015.123
Hochreiter, S., & Schmidhuber, J. (1997). Long Short-Term Memory. Neural Computation, 9(8), 1735-1780. https://doi.org/10.1162/neco.1997.9.8.1735
Horwitz, R., Quatieri, T. F., Helfer, B. S., Yu, B., Williamson, J. R., & Mundt, J. (2013). On the relative importance of vocal source, system, and prosody in human depression. 2013 IEEE International Conference on Body Sensor Networks, 1-6. https://doi.org/10.1109/BSN.2013.6575522
Huang, C.-Q., Dong, B.-R., Lu, Z.-C., Yue, J.-R., & Liu, Q.-X. (2010). Chronic diseases and risk for depression in old age: A meta-analysis of published literature. Ageing Research Reviews, 9(2), 131-141. https://doi.org/10.1016/j.arr.2009.05.005
Huang, G., Liu, Z., van der Maaten, L., & Weinberger, K. Q. (2018). Densely Connected Convolutional Networks. arXiv:1608.06993 [cs]. http://arxiv.org/abs/1608.06993
Husseini Orabi, A., Buddhitha, P., Husseini Orabi, M., & Inkpen, D. (2018). Deep Learning for Depression Detection of Twitter Users. Proceedings of the Fifth Workshop on Computational Linguistics and Clinical Psychology: From Keyboard to Clinic, 88-97. https://doi.org/10.18653/v1/W18-0609
Hyperpolarization (biology). (2019). En Wikipedia. https://en.wikipedia.org/w/index.php?title=Hyperpolarization_(biology)&oldid=906223773
Iizuka, S., Simo-Serra, E., & Ishikawa, H. (2016). Let there be color! Joint end-to-end learning of global and local image priors for automatic image colorization with simultaneous classification. ACM Transactions on Graphics, 35(4), 110:1-110:11. https://doi.org/10.1145/2897824.2925974
Index of Effects, Generators, Analyzers and Tools—Audacity Manual. (2021, abril 13). https://manual.audacityteam.org/man/index_of_effects_generators_and_analyzers.html
Information, N. C. for B., Pike, U. S. N. L. of M. 8600 R., MD, B., & Usa, 20894. (2020). Depression: How effective are antidepressants? En InformedHealth.org [Internet]. Institute for Quality and Efficiency in Health Care (IQWiG). https://www.ncbi.nlm.nih.gov/books/NBK361016/
Inteligencia artificial. (2020). En Wikipedia, la enciclopedia libre. https://es.wikipedia.org/w/index.php?title=Inteligencia_artificial&oldid=128219790
Inteligencia artificial débil. (2020). En Wikipedia, la enciclopedia libre. https://es.wikipedia.org/w/index.php?title=Inteligencia_artificial_d%C3%A9bil&oldid=126205185
Inteligencia artificial fuerte. (2020). En Wikipedia, la enciclopedia libre. https://es.wikipedia.org/w/index.php?title=Inteligencia_artificial_fuerte&oldid=123209221
Inteligencia artificial simbólica. (2019). En Wikipedia, la enciclopedia libre. https://es.wikipedia.org/w/index.php?title=Inteligencia_artificial_simb%C3%B3lica&oldid=117282266
International Affective Picture System. (2020). En Wikipedia. https://en.wikipedia.org/w/index.php?title=International_Affective_Picture_System&oldid=971124703
Ioffe, S., & Szegedy, C. (2015). Batch Normalization: Accelerating Deep Network Training by Reducing Internal Covariate Shift. ArXiv:1502.03167 [Cs]. http://arxiv.org/abs/1502.03167
Ippolito, P. P. (2019, septiembre 26). Hyperparameters Optimization. Medium. https://towardsdatascience.com/hyperparameters-optimization-526348bb8e2d
Jianguo, Yu. (2019). Multimodal Information Fusion based on Deep Learning [Doctoral Dissertation, University of Aizu]. https://u-aizu.repo.nii.ac.jp/?action=repository_uri&item_id=172&file_id=20
Jing, X.-Y., Liu, Q., Wu, F., Xu, B., Zhu, Y., & Chen, S. (2015). Web Page Classification Based on Uncorrelated Semi-Supervised Intra-View and Inter-View Manifold Discriminant Feature Extraction. 7.
Joffres, M., Jaramillo, A., Dickinson, J., Lewin, G., Pottie, K., Shaw, E., Gorber, S. C., & Tonelli, M. (2013). Recommendations on screening for depression in adults. CMAJ : Canadian Medical Association Journal, 185(9), 775-782. https://doi.org/10.1503/cmaj.130403
Joshi, J., Goecke, R., Alghowinem, S., Dhall, A., Wagner, M., Epps, J., Parker, G., & Breakspear, M. (2013). Multimodal assistive technologies for depression diagnosis and monitoring. Journal on Multimodal User Interfaces, 7(3), 217-228. https://doi.org/10.1007/s12193-013-0123-2
K vecinos más próximos. (2020). En Wikipedia, la enciclopedia libre. https://es.wikipedia.org/w/index.php?title=K_vecinos_m%C3%A1s_pr%C3%B3ximos&oldid=129076386
Kahou, S. E., Bouthillier, X., Lamblin, P., Gulcehre, C., Michalski, V., Konda, K., Jean, S., Froumenty, P., Dauphin, Y., Boulanger-Lewandowski, N., Chandias Ferrari, R., Mirza, M., Warde-Farley, D., Courville, A., Vincent, P., Memisevic, R., Pal, C., & Bengio, Y. (2016). EmoNets: Multimodal deep learning approaches for emotion recognition in video. Journal on Multimodal User Interfaces, 10(2), 99-111. https://doi.org/10.1007/s12193-015-0195-2
Kandel, E., Schwartz, J., Jessell, T., Siegelbaum, S., & Hudspeth, A. J. (2000). Princípios de Neurociencia (Cuarta). McGraw-Hill Interamericana.
Kapfhammer, H.-P. (2006). Somatic symptoms in depression. Dialogues in Clinical Neuroscience, 8(2), 227-239.
Katanforoosh, K., & Kunin, D. (2019). Initializing neural networks. deeplearning.ai. https://www.deeplearning.ai/ai-notes/initialization/
Kessler, R. C., Foster, C. L., Saunders, W. B., & Stang, P. E. (1995). Social consequences of psychiatric disorders, I: Educational attainment. The American Journal of Psychiatry, 152(7), 1026-1032. https://doi.org/10.1176/ajp.152.7.1026
Kessler, R. C., McGonagle, K. A., Swartz, M., Blazer, D. G., & Nelson, C. B. (1993). Sex and depression in the National Comorbidity Survey. I: Lifetime prevalence, chronicity and recurrence. Journal of Affective Disorders, 29(2-3), 85-96. https://doi.org/10.1016/0165-0327(93)90026-g
Kollias, D., Tzirakis, P., Nicolaou, M. A., Papaioannou, A., Zhao, G., Schuller, B., Kotsia, I., & Zafeiriou, S. (2019). Deep Affect Prediction in-the-wild: Aff-Wild Database and Challenge, Deep Architectures, and Beyond. International Journal of Computer Vision, 127(6-7), 907-929. https://doi.org/10.1007/s11263-019-01158-4
Kononenko, I. (2001). Machine learning for medical diagnosis: History, state of the art and perspective. Artificial Intelligence in Medicine, 23(1), 89-109. https://doi.org/10.1016/S0933-3657(01)00077-X
Kopczyk, D. (2018, marzo 23). Hyperparameter optimization: Explanation of automatized algorithms. Dawid Kopczyk. https://dkopczyk.quantee.co.uk/hyperparameter-optimization/
Krizhevsky, A., Sutskever, I., & Hinton, G. E. (2012). ImageNet Classification with Deep Convolutional Neural Networks. En F. Pereira, C. J. C. Burges, L. Bottou, & K. Q. Weinberger (Eds.), Advances in Neural Information Processing Systems 25 (pp. 1097-1105). Curran Associates, Inc. http://papers.nips.cc/paper/4824-imagenet-classification-with-deep-convolutional-neural-networks.pdf
Kroenke, K., Spitzer, R. L., & Williams, J. B. W. (2001). The PHQ-9. Journal of General Internal Medicine, 16(9), 606-613. https://doi.org/10.1046/j.1525-1497.2001.016009606.x
Kroenke, K., Spitzer, R. L., Williams, J. B. W., & Löwe, B. (2010). The Patient Health Questionnaire Somatic, Anxiety, and Depressive Symptom Scales: A systematic review. General Hospital Psychiatry, 32(4), 345-359. https://doi.org/10.1016/j.genhosppsych.2010.03.006
Kroenke, K., Strine, T. W., Spitzer, R. L., Williams, J. B. W., Berry, J. T., & Mokdad, A. H. (2009). The PHQ-8 as a measure of current depression in the general population. Journal of Affective Disorders, 114(1-3), 163-173. https://doi.org/10.1016/j.jad.2008.06.026
Kunze, J., Kirsch, L., Kurenkov, I., Krug, A., Johannsmeier, J., & Stober, S. (2017). Transfer Learning for Speech Recognition on a Budget. Proceedings of the 2nd Workshop on Representation Learning for NLP, 168-177. https://doi.org/10.18653/v1/W17-2620
Lépine, J.-P., & Briley, M. (2011). The increasing burden of depression. Neuropsychiatric Disease and Treatment, 7(Suppl 1), 3-7. https://doi.org/10.2147/NDT.S19617
Levinson, D. F. (2006). The Genetics of Depression: A Review. Biological Psychiatry, 60(2), 84-92. https://doi.org/10.1016/j.biopsych.2005.08.024
Levkovitz, Y., Tedeschini, E., & Papakostas, G. I. (2011). Efficacy of antidepressants for dysthymia: A meta-analysis of placebo-controlled randomized trials. The Journal of Clinical Psychiatry, 72(4), 509-514. https://doi.org/10.4088/JCP.09m05949blu
Liang, F. (2020, agosto 25). Optimization Techniques: Genetic Algorithm. Medium. https://towardsdatascience.com/optimizing-machine-learning-models-with-genetic-algorithms-2a38682a0610
Lin, M., Chen, Q., & Yan, S. (2014). Network In Network. arXiv:1312.4400 [cs]. http://arxiv.org/abs/1312.4400
Lisanby, S. H. (2007). Electroconvulsive Therapy for Depression. The New England Journal of Medicine, 7.
Liu, K., Li, Y., Xu, N., & Natarajan, P. (2018). Learn to Combine Modalities in Multimodal Deep Learning. https://arxiv.org/abs/1805.11730v1
Logan, B. (2000). Mel Frequency Cepstral Coefficients for Music Modeling. In International Symposium on Music Information Retrieval.
Lopez, A. D., Mathers, C. D., Ezzati, M., Jamison, D. T., & Murray, C. J. (2006). Global and regional burden of disease and risk factors, 2001: Systematic analysis of population health data. The lancet, 367(9524), 1747-1757.
Lund, J., & Ng, Y.-K. (2018). Movie Recommendations Using the Deep Learning Approach. 2018 IEEE International Conference on Information Reuse and Integration (IRI), 47-54. https://doi.org/10.1109/IRI.2018.00015
Macros—Audacity Manual. (2021, abril 13). https://manual.audacityteam.org/man/macros.html
Malingering. (2020). En Wikipedia. https://en.wikipedia.org/w/index.php?title=Malingering&oldid=952656725
Mandrekar, J. N. (2010). Receiver Operating Characteristic Curve in Diagnostic Test Assessment. Journal of Thoracic Oncology, 5(9), 1315-1316. https://doi.org/10.1097/JTO.0b013e3181ec173d
Maxim, L. D., Niebo, R., & Utell, M. J. (2014). Screening tests: A review with examples. Inhalation Toxicology, 26(13), 811-828. https://doi.org/10.3109/08958378.2014.955932
Mel-frequency cepstrum. (2020). En Wikipedia. https://en.wikipedia.org/w/index.php?title=Mel-frequency_cepstrum&oldid=988437240
Membrane potential. (s. f.). Khan Academy. Recuperado 24 de agosto de 2020, de https://www.khanacademy.org/science/biology/human-biology/neuron-nervous-system/a/the-membrane-potential
Merino, M. (2019, enero 27). Conceptos de inteligencia artificial: Qué es el aprendizaje por refuerzo. Xataka. https://www.xataka.com/inteligencia-artificial/conceptos-inteligencia-artificial-que-aprendizaje-refuerzo
MFCC. (2020). En Wikipedia, la enciclopedia libre. https://es.wikipedia.org/w/index.php?title=MFCC&oldid=128384562
Microsoft researchers achieve speech recognition milestone. (2016, septiembre 13). The AI Blog. https://blogs.microsoft.com/ai/microsoft-researchers-achieve-speech-recognition-milestone/
Mikolov, T., Sutskever, I., Chen, K., Corrado, G. S., & Dean, J. (2013). Distributed Representations of Words and Phrases and their Compositionality. En C. J. C. Burges, L. Bottou, M. Welling, Z. Ghahramani, & K. Q. Weinberger (Eds.), Advances in Neural Information Processing Systems 26 (pp. 3111-3119). Curran Associates, Inc. http://papers.nips.cc/paper/5021-distributed-representations-of-words-and-phrases-and-their-compositionality.pdf
Montgomery–Åsberg Depression Rating Scale. (2020). En Wikipedia. https://en.wikipedia.org/w/index.php?title=Montgomery%E2%80%93%C3%85sberg_Depression_Rating_Scale&oldid=950308817
Moolayil, J. J. (2020, mayo 30). A Layman’s Guide to Deep Neural Networks. Medium. https://towardsdatascience.com/a-laymans-guide-to-deep-neural-networks-ddcea24847fb
Moore, E., Clements, M., Peifer, J., & Weisser, L. (2004). Comparing objective feature statistics of speech for classifying clinical depression. Conference Proceedings: ... Annual International Conference of the IEEE Engineering in Medicine and Biology Society. IEEE Engineering in Medicine and Biology Society. Annual Conference, 2006, 17-20. https://doi.org/10.1109/IEMBS.2004.1403079
Müller, A. C., & Guido, S. (2016). Introduction to Machine Learning with Python: A Guide for Data Scientists. O’Reilly Media. https://books.google.es/books?id=vbQlDQAAQBAJ
Murphy, G. E., Simons, A. D., Wetzel, R. D., & Lustman, P. J. (1984). Cognitive therapy and pharmacotherapy. Singly and together in the treatment of depression. Archives of General Psychiatry, 41(1), 33-41. https://doi.org/10.1001/archpsyc.1984.01790120037006
Naderi, H., Soleimani, B. H., & Matwin, S. (2020). Multimodal Deep Learning for Mental Disorders Prediction from Audio Speech Samples. arXiv:1909.01067 [cs, eess, stat]. http://arxiv.org/abs/1909.01067
Nagyfi, R. (2018, septiembre 4). The differences between Artificial and Biological Neural Networks. Medium. https://towardsdatascience.com/the-differences-between-artificial-and-biological-neural-networks-a8b46db828b7
Narumoto, M. (2017, junio 23). Patrón Pipes and Filters—Cloud Design Patterns. https://docs.microsoft.com/es-es/azure/architecture/patterns/pipes-and-filters
National Collaborating Centre for Mental Health. (2010). Depression: The Treatment and Management of Depression in Adults (Updated Edition). British Psychological Society. http://www.ncbi.nlm.nih.gov/books/NBK63748/
National Health Service. (2017, octubre 24). Antidepressants. Nhs.Uk. https://www.nhs.uk/conditions/antidepressants/
National Health Service. (2018, agosto 31). Antidepressants—Side effects. Nhs.Uk. https://www.nhs.uk/conditions/antidepressants/side-effects/
National Institute of Mental Health. (2016). Depression Basics. 6.
Neuron doctrine. (2020). En Wikipedia. https://en.wikipedia.org/w/index.php?title=Neuron_doctrine&oldid=958614915
Ng, A. (2018). Machine Learning Yearning.
Ng, C. W. M., How, C. H., & Ng, Y. P. (2016). Major depression in primary care: Making the diagnosis. Singapore Medical Journal, 57(11), 591-597. https://doi.org/10.11622/smedj.2016174
Nischal, A., Tripathi, A., Nischal, A., & Trivedi, J. K. (2012). Suicide and Antidepressants: What Current Evidence Indicates. Mens Sana Monographs, 10(1), 33-44. https://doi.org/10.4103/0973-1229.87287
NVIDIA. (2016, julio 29). The Difference Between AI, Machine Learning, and Deep Learning? | NVIDIA Blog. The Official NVIDIA Blog. https://blogs.nvidia.com/blog/2016/07/29/whats-difference-artificial-intelligence-machine-learning-deep-learning-ai/
Organización Mundial de la Salud. (2011). Carga mundial de trastornos mentales y necesidad de que el sector de la salud y el sector social respondan de modo integral y coordinado a escala de país. Consejo Ejecutivo. https://apps.who.int/gb/ebwha/pdf_files/EB130/B130_9-sp.pdf
Organización Mundial de la Salud. (2017). Depression and Other Common Mental Disorders Global Health Estimates. https://apps.who.int/iris/bitstream/handle/10665/254610/WHO-MSD-MER-2017.2-eng.pdf?sequence=1
Organización Mundial de la Salud. (30 de enero). Depresión. https://www.who.int/es/news-room/fact-sheets/detail/depression
Osgood, B. (2019). Lectures on the Fourier transform and its applications.
O’Shaughnessy, D. (1987). Speech Communication: Human and Machine. Addison-Wesley Publishing Company.
Ozdas, A., Shiavi, R. G., Silverman, S. E., Silverman, M. K., & Wilkes, D. M. (2000). Analysis of fundamental frequency for near term suicidal risk assessment. Smc 2000 conference proceedings. 2000 ieee international conference on systems, man and cybernetics. «cybernetics evolving to systems, humans, organizations, and their complex interactions» (cat. no.0, 3, 1853-1858 vol.3. https://doi.org/10.1109/ICSMC.2000.886379
Pan, S. J., & Yang, Q. (2010). A Survey on Transfer Learning. IEEE Transactions on Knowledge and Data Engineering, 22(10), 1345-1359. https://doi.org/10.1109/TKDE.2009.191
Papers with Code—ImageNet Benchmark (Image Classification). (s. f.). Recuperado 26 de agosto de 2020, de https://paperswithcode.com/sota/image-classification-on-imagenet
Paris, J. (2014). The Mistreatment of Major Depressive Disorder. Canadian Journal of Psychiatry. Revue Canadienne de Psychiatrie, 59(3), 148-151.
Parker, G., & Brotchie, H. (2010). Gender differences in depression. International Review of Psychiatry (Abingdon, England), 22(5), 429-436. https://doi.org/10.3109/09540261.2010.492391
Parker, Robert, Graff, David, Kong, Junbo, Chen, Ke, & Maeda, Kazuaki. (2011). English Gigaword Fifth Edition [Data set]. Linguistic Data Consortium. https://doi.org/10.35111/WK4F-QT80
Patel, A. A. (2019). Hands-On Unsupervised Learning Using Python: How to Build Applied Machine Learning Solutions from Unlabeled Data. O’Reilly Media, Inc.
Patient Health Questionnaire. (2020). En Wikipedia. https://en.wikipedia.org/w/index.php?title=Patient_Health_Questionnaire&oldid=984404806
Patwari, R. (2013, junio 19). The tradeoff between sensitivity and specificity. https://www.youtube.com/watch?v=vtYDyGGeQyo
Pennington, J. (2014, agosto). GloVe: Global Vectors for Word Representation. https://nlp.stanford.edu/projects/glove/
Pennington, J., Socher, R., & Manning, C. (2014). Glove: Global Vectors for Word Representation. Proceedings of the 2014 Conference on Empirical Methods in Natural Language Processing (EMNLP), 1532-1543. https://doi.org/10.3115/v1/D14-1162
PHQ-9. (2020). En Wikipedia. https://en.wikipedia.org/w/index.php?title=PHQ-9&oldid=980815980
Physical symbol system. (2020). En Wikipedia. https://en.wikipedia.org/w/index.php?title=Physical_symbol_system&oldid=967737824
Pita Fernández, S., P. D., S. (2010, diciembre 7). Pruebas diagnósticas: Sensibilidad y especificidad. Pruebas diagnósticas: Sensibilidad y especificidad. http://www.fisterra.com/mbe/investiga/pruebas_diagnosticas/pruebas_diagnosticas.asp
Prabhu. (2019, noviembre 21). Understanding of Convolutional Neural Network (CNN)—Deep Learning. Medium. https://medium.com/@RaghavPrabhu/understanding-of-convolutional-neural-network-cnn-deep-learning-99760835f148
Prabhu, K. M. M. (2013). Window functions and their applications in signal processing. https://doi.org/10.1201/b15570
Pujol, P., Macho, D., & Nadeu, C. (2006). On Real-Time Mean-and-Variance Normalization of Speech Recognition Features. 2006 IEEE International Conference on Acoustics Speed and Signal Processing Proceedings, 1, I-773-I-776. https://doi.org/10.1109/ICASSP.2006.1660135
Qato, D. M., Ozenberger, K., & Olfson, M. (2018). Prevalence of Prescription Medications With Depression as a Potential Adverse Effect Among Adults in the United States. JAMA, 319(22), 2289-2298. https://doi.org/10.1001/jama.2018.6741
Ramírez, J. (2018, julio 19). Curvas PR y ROC. Medium. https://medium.com/bluekiri/curvas-pr-y-roc-1489fbd9a527
Raschka, S., & Mirjalili, V. (2017). Python Machine Learning. Packt Publishing. https://books.google.es/books?id=_plGDwAAQBAJ
Rating scales for depression. (2020). En Wikipedia. https://en.wikipedia.org/w/index.php?title=Rating_scales_for_depression&oldid=964770176
Ravishankar, H., Sudhakar, P., Venkataramani, R., Thiruvenkadam, S., Annangi, P., Babu, N., & Vaidya, V. (2017). Understanding the Mechanisms of Deep Transfer Learning for Medical Images. arXiv:1704.06040 [cs]. http://arxiv.org/abs/1704.06040
Ray, A., Kumar, S., Reddy, R., Mukherjee, P., & Garg, R. (2019). Multi-level Attention network using text, audio and video for Depression Prediction. arXiv:1909.01417 [cs, eess]. http://arxiv.org/abs/1909.01417
Razykov, I., Ziegelstein, R. C., Whooley, M. A., & Thombs, B. D. (2012). The PHQ-9 versus the PHQ-8 — Is item 9 useful for assessing suicide risk in coronary artery disease patients? Data from the Heart and Soul Study. Journal of Psychosomatic Research, 73(3), 163-168. https://doi.org/10.1016/j.jpsychores.2012.06.001
Reddy, M. S. (2010). Depression: The Disorder and the Burden. Indian Journal of Psychological Medicine, 32(1), 1-2. https://doi.org/10.4103/0253-7176.70510
Reimers, N., & Gurevych, I. (2019). Sentence-BERT: Sentence Embeddings using Siamese BERT-Networks. arXiv:1908.10084 [cs]. http://arxiv.org/abs/1908.10084
Reinforcement Learning: Applications in Finance. (2019, noviembre 30). Finance Talks. https://financetalks.ie.edu/reinforcement-learning-applications-in-finance/
Rejaibi, E., Komaty, A., Meriaudeau, F., Agrebi, S., & Othmani, A. (2020). MFCC-based Recurrent Neural Network for Automatic Clinical Depression Recognition and Assessment from Speech. arXiv:1909.07208 [cs, eess]. http://arxiv.org/abs/1909.07208
Ringeval, F., Schuller, B., Valstar, M., Cummins, Ni., Cowie, R., Tavabi, L., Schmitt, M., Alisamir, S., Amiriparian, S., Messner, E.-M., Song, S., Liu, S., Zhao, Z., Mallol-Ragolta, A., Ren, Z., Soleymani, M., & Pantic, M. (2019). AVEC 2019 Workshop and Challenge: State-of-Mind, Detecting Depression with AI, and Cross-Cultural Affect Recognition. arXiv:1907.11510 [cs, stat]. http://arxiv.org/abs/1907.11510
Rm, H. (1999). Efficacy of SSRIs and newer antidepressants in severe depression: Comparison with TCAs. The Journal of Clinical Psychiatry, 60(5), 326-335. https://doi.org/10.4088/jcp.v60n0511
Roger Jang, J.-S. (2011). MFCC - Audio Signal Processing and Recognition. http://mirlab.org/jang/books/audiosignalprocessing/speechFeatureMfcc.asp?title=12-2%20MFCC
Rogers, D., & Pies, R. (2008). General Medical Drugs Associated with Depression. Psychiatry (Edgmont), 5(12), 28-41.
Rohanian, M., Hough, J., & Purver, M. (2019). Detecting Depression with Word-Level Multimodal Fusion. Interspeech 2019, 1443-1447. https://doi.org/10.21437/Interspeech.2019-2283
Romera, I., Montejo, Á. L., Aragonés, E., Arbesú, J. Á., Iglesias-García, C., López, S., Lozano, J. A., Pamulapati, S., Yruretagoyena, B., & Gilaberte, I. (2013). Systematic depression screening in high-risk patients attending primary care: A pragmatic cluster-randomized trial. BMC Psychiatry, 13(1), 83. https://doi.org/10.1186/1471-244X-13-83
Ross, A. (2009). Fusion, Feature-Level. En S. Z. Li & A. Jain (Eds.), Encyclopedia of Biometrics (pp. 597-602). Springer US. https://doi.org/10.1007/978-0-387-73003-5_157
Santana Vega, C. (2017, noviembre 1). ¿Qué es el Machine Learning? ¿Y Deep Learning? Un mapa conceptual | DotCSV. https://www.youtube.com/watch?v=KytW151dpqU
Santana Vega, C. (2019, septiembre 19). ¿Por qué las GPUs son buenas para la IA? | Data Coffee #12. https://www.youtube.com/watch?v=C_wSHKG8_fg
Sarkar, D. (DJ). (2018, noviembre 17). A Comprehensive Hands-on Guide to Transfer Learning with Real-World Applications in Deep Learning. Medium. https://towardsdatascience.com/a-comprehensive-hands-on-guide-to-transfer-learning-with-real-world-applications-in-deep-learning-212bf3b2f27a
Schartz Rehan. (2019, junio 7). The Shape of Tensor. Mc.Ai. https://mc.ai/the-shape-of-tensor/
Schildkraut, J. J. (1965). The catecholamine hypothesis of affective disorders: A review of supporting evidence. The American Journal of Psychiatry, 122(5), 509-522. https://doi.org/10.1176/ajp.122.5.509
Schmidt, H. D., Shelton, R. C., & Duman, R. S. (2011). Functional biomarkers of depression: Diagnosis, treatment, and pathophysiology. Neuropsychopharmacology: Official Publication of the American College of Neuropsychopharmacology, 36(12), 2375-2394. https://doi.org/10.1038/npp.2011.151
Semmlow, J. (2005). Circuits, Signals, and Systems for Bioengineers: A MATLAB-Based Introduction. Elsevier Science. https://books.google.es/books?id=bvtB7tfsyFMC
Sensibilidad y especificidad. (2021). En Wikipedia, la enciclopedia libre. https://es.wikipedia.org/w/index.php?title=Sensibilidad_y_especificidad&oldid=133299954
Sentís, J. I. L. (2019). Ética para máquinas. Grupo Planeta.
Sheehan, D. V., Lecrubier, Y., Sheehan, K. H., Amorim, P., Janavs, J., Weiller, E., Hergueta, T., Baker, R., & Dunbar, G. C. (1998). The Mini-International Neuropsychiatric Interview (M.I.N.I.): The development and validation of a structured diagnostic psychiatric interview for DSM-IV and ICD-10. The Journal of Clinical Psychiatry, 59 Suppl 20, 22-33;quiz 34-57.
Short-time Fourier transform. (2020). En Wikipedia. https://en.wikipedia.org/w/index.php?title=Short-time_Fourier_transform&oldid=996725594
Simonyan, K., & Zisserman, A. (2015). Very Deep Convolutional Networks for Large-Scale Image Recognition. arXiv:1409.1556 [cs]. http://arxiv.org/abs/1409.1556
Singla, K., Bose, J., & Naik, C. (2018). Analysis of Software Engineering for Agile Machine Learning Projects. 2018 15th IEEE India Council International Conference (INDICON), 1-5. https://doi.org/10.1109/INDICON45594.2018.8987154
Siu, A. L., US Preventive Services Task Force (USPSTF), Bibbins-Domingo, K., Grossman, D. C., Baumann, L. C., Davidson, K. W., Ebell, M., García, F. A. R., Gillman, M., Herzstein, J., Kemper, A. R., Krist, A. H., Kurth, A. E., Owens, D. K., Phillips, W. R., Phipps, M. G., & Pignone, M. P. (2016). Screening for Depression in Adults: US Preventive Services Task Force Recommendation Statement. JAMA, 315(4), 380-387. https://doi.org/10.1001/jama.2015.18392
Sloan, D. M. E., & Kornstein, S. G. (2003). Gender differences in depression and response to antidepressant treatment. The Psychiatric Clinics of North America, 26(3), 581-594. https://doi.org/10.1016/s0193-953x(03)00044-3
Smith, L. N. (2017). Cyclical Learning Rates for Training Neural Networks. arXiv:1506.01186 [cs]. http://arxiv.org/abs/1506.01186
Solon, O. (2017, enero 30). Oh the humanity! Poker computer trounces humans in big step for AI. The Guardian. http://www.theguardian.com/technology/2017/jan/30/libratus-poker-artificial-intelligence-professional-human-players-competition
Sommerville, I. (2011). Software Engineering. Pearson.
Srivastava, N., Hinton, G., Krizhevsky, A., Sutskever, I., & Salakhutdinov, R. (2014). Dropout: A Simple Way to Prevent Neural Networks from Overfitting. 30.
Stevens, L., & Rodin, I. (2011). Management plan and formulation. En L. Stevens & I. Rodin (Eds.), Psychiatry (Second Edition) (Second Edition, pp. 14-15). Churchill Livingstone. https://doi.org/10.1016/B978-0-7020-3396-4.00011-1
Stevens, S. S., Volkmann, J., & Newman, E. B. (1937). A scale for the measurement of the psychological magnitude pitch. Journal of the Acoustical Society of America, 8, 185-190. https://doi.org/10.1121/1.1915893
Strawbridge, R., Young, A. H., & Cleare, A. J. (2017). Biomarkers for depression: Recent insights, current challenges and future prospects. Neuropsychiatric Disease and Treatment, 13, 1245-1262. https://doi.org/10.2147/NDT.S114542
Strimbu, K., & Tavel, J. A. (2010). What are Biomarkers? Current opinion in HIV and AIDS, 5(6), 463-466. https://doi.org/10.1097/COH.0b013e32833ed177
Superinteligencia. (2020). En Wikipedia, la enciclopedia libre. https://es.wikipedia.org/w/index.php?title=Superinteligencia&oldid=126782951
Taddeo, S. (2014, junio 5). Santiago Felipe Ramon y Cajal (1852-1934) | The Embryo Project Encyclopedia. https://embryo.asu.edu/pages/santiago-felipe-ramon-y-cajal-1852-1934
Tar. (2019). En Wikipedia, la enciclopedia libre. https://es.wikipedia.org/w/index.php?title=Tar&oldid=121690445
Tennant, C. (2016). Life Events, Stress and Depression: A Review of Recent Findings: Australian & New Zealand Journal of Psychiatry. https://journals.sagepub.com/doi/10.1046/j.1440-1614.2002.01007.x
Thanks · Issue #3646 · microsoft/CNTK. (2019, abril 25). GitHub. https://github.com/microsoft/CNTK/issues/3646
The Cost of Hard Drives Over Time. (2017, julio 11). Backblaze Blog | Cloud Storage & Cloud Backup. https://www.backblaze.com/blog/hard-drive-cost-per-gigabyte/
The Synapse. (s. f.). Khan Academy. Recuperado 24 de agosto de 2020, de https://www.khanacademy.org/science/biology/human-biology/neuron-nervous-system/a/the-synapse
Torfi, A. (2018). SpeechPy-A Library for Speech Processing and Recognition. arXiv preprint arXiv:1803.01094.
Touvron, H., Vedaldi, A., Douze, M., & Jégou, H. (2020). Fixing the train-test resolution discrepancy: FixEfficientNet. arXiv:2003.08237 [cs]. http://arxiv.org/abs/2003.08237
Transforming Categorical Data. (2019, septiembre 7). Google Developers. https://developers.google.com/machine-learning/data-prep/transform/transform-categorical?hl=es-419
Trastorno facticio. (2020). En Wikipedia, la enciclopedia libre. https://es.wikipedia.org/w/index.php?title=Trastorno_facticio&oldid=122892207
Trastorno facticio impuesto a otro—Trastornos de la salud mental. (2019, septiembre). Manual MSD versión para público general. https://www.msdmanuals.com/es/hogar/trastornos-de-la-salud-mental/trastornos-som%C3%A1ticos-y-trastornos-relacionados/trastorno-facticio-impuesto-a-otro
Trastorno facticio impuesto a uno mismo—Trastornos de la salud mental. (2019, septiembre). Manual MSD versión para público general. https://www.msdmanuals.com/es/hogar/trastornos-de-la-salud-mental/trastornos-som%C3%A1ticos-y-trastornos-relacionados/trastorno-facticio-impuesto-a-uno-mismo
Turc, I., Chang, M.-W., Lee, K., & Toutanova, K. (2019). Well-Read Students Learn Better: On the Importance of Pre-training Compact Models. arXiv preprint arXiv:1908.08962v2.
Types of talking therapies. (2019, noviembre 11). Nhs.Uk. https://www.nhs.uk/conditions/stress-anxiety-depression/types-of-therapy/
Vaswani, A., Shazeer, N., Parmar, N., Uszkoreit, J., Jones, L., Gomez, A. N., Kaiser, L., & Polosukhin, I. (2017). Attention Is All You Need. arXiv:1706.03762 [cs]. http://arxiv.org/abs/1706.03762
Vedia Domingo, V. (2016). Duelo Patológico: Factores de riesgo y protección. Revista Digital de Medicina Psicosomática y Psicoterapia, 6(2), 23.
Venkataramanan, K., & Rajamohan, H. R. (2019). Emotion Recognition from Speech. arXiv:1912.10458 [cs, eess]. http://arxiv.org/abs/1912.10458
Wagle, M. (2020, marzo 25). Association Rules: Unsupervised Learning in Retail. Medium. https://medium.com/@manilwagle/association-rules-unsupervised-learning-in-retail-69791aef99a
Wikimedia Downloads. (s. f.). Recuperado 16 de junio de 2021, de https://dumps.wikimedia.org/backup-index.html
Witten, I. H., Frank, E., & Hall, M. A. (2011). Data mining: Practical machine learning tools and techniques (3rd ed). Morgan Kaufmann.
Wolpert, L. (2001). Stigma of depression – a personal view. British Medical Bulletin, 57(1), 221-224. https://doi.org/10.1093/bmb/57.1.221
Xia, W., Shen, L., & Zhang, J. (2015). Comorbid anxiety and depression in school-aged children with attention deficit hyperactivity disorder (ADHD) and selfreported symptoms of ADHD, anxiety, and depression among parents of school-aged children with and without ADHD. Shanghai Archives of Psychiatry, 27(6), 356-367. https://doi.org/10.11919/j.issn.1002-0829.215115
Yang, L., Jiang, D., Xia, X., Pei, E., Oveneke, M. C., & Sahli, H. (2017). Multimodal Measurement of Depression Using Deep Learning Models. Proceedings of the 7th Annual Workshop on Audio/Visual Emotion Challenge, 53-59. https://doi.org/10.1145/3133944.3133948
Yeung, A. S., Jing, Y., Brenneman, S. K., Chang, T. E., Baer, L., Hebden, T., Kalsekar, I., McQuade, R. D., Kurlander, J., Siebenaler, J., & Fava, M. (2012). Clinical Outcomes in Measurement-Based Treatment (comet): A Trial of Depression Monitoring and Feedback to Primary Care Physicians. Depression and Anxiety, 29(10), 865-873. https://doi.org/10.1002/da.21983
Yin, S., Liang, C., Ding, H., & Wang, S. (2019). A Multi-Modal Hierarchical Recurrent Neural Network for Depression Detection. Proceedings of the 9th International on Audio/Visual Emotion Challenge and Workshop, 65-71. https://doi.org/10.1145/3347320.3357696
Zhang, A., Lipton, Z. C., Li, M., & Smola, A. J. (2020). Dive into Deep Learning.
Zhang, Y. (2010). New Advances in Machine Learning. IntechOpen. https://books.google.es/books?id=XAqhDwAAQBAJ
Zhang, Z., Lin, W., Liu, M., & Mahmoud, M. (s. f.). Multimodal Deep Learning Framework for Mental Disorder Recognition. 7.
Zheng, A., & Casari, A. (2018). Feature Engineering for Machine Learning: Principles and Techniques for Data Scientists. O’Reilly Media. https://books.google.es/books?id=sthSDwAAQBAJ
Zhu, X., & Goldberg, A. B. (2009). Introduction to Semi-Supervised Learning. Synthesis Lectures on Artificial Intelligence and Machine Learning, 3(1), 1-130. https://doi.org/10.2200/S00196ED1V01Y200906AIM006

