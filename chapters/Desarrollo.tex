\chapter{[En curso] Desarrollo del sistema}
    \label{chapter:desarrollo}

    \chapquote{¿Quién dijo miedo habiendo hospitales?}{Sabiduría popular de la ETSISI}
    
    \section{Setup del proyecto}

    \section{Implementación de la aplicación móvil}

    \section{Implementación de la API del servidor}

        \subsection{Subida de datos de usuario}

        \subsection{Subida de los datos de los cuestionarios}

        \subsection{Datos de la comunidad}

    \subsection{Grafos de navegación}
    \begin{figure}[h]
        \centering
        \begin{tikzpicture}[
                > = stealth, % estilo de la cabeza de la flecha
                shorten > = 1pt, % para que la flecha no toque el nodo
                node distance = 3cm, % distancia entre nodos
            ]
    
            \tikzstyle{every state}=[
                draw = black,
                thick,
                fill = white,
            ]

            \node[rectangle,draw] (click) {\textit{El usuario entra en la app}};
            \node[state] [below of=click] (splash) {Splash};
            \node[state] [below of=splash] (bienvenida) {Bienvenida};
            \node[state] [below of=bienvenida] (permisos) {Permisos};
            \node[state] [below of=permisos] (inicio) {\textit{Inicio}};
            
            \path[->] (click) edge node {} (splash);
            \path[->] (splash) edge node {} (bienvenida);
            \path[->] (bienvenida) edge node {} (permisos);
            \path[->] (permisos) edge node {} (inicio);
    
        \end{tikzpicture}
        \caption{Grafo de navegación en el primer uso de la app}
        \label{figure:disenio:grafo_primer_uso}
    \end{figure}
    
    \begin{figure}[h]
        \centering
        \begin{tikzpicture}[
                font=\small, % fuente de letra pequeña
                > = stealth, % estilo de la cabeza de la flecha
                shorten > = 1pt, % para que la flecha no toque el nodo
                node distance = 3.3cm, % distancia entre nodos
            ]
    
            \tikzstyle{every state}=[
                draw = black,
                thick,
                fill = white,
            ]

            \node[rectangle,draw] (click) {E\textit{l usuario entra en la app}};
            \node[state] [below of=click] (splash) {Splash};
            \node[state] [below of=splash] (inicio) {Inicio};
    
            \node[state] [below left of=inicio] (historial) {Historial};
            \node[state] [below right of=inicio] (comunidad) {Comunidad};
            \node[state, align=center] [right of=inicio] (incompletos) {Cuestionarios\\incompletos};
            \node[state] [below right of=historial] (ajustes) {Ajustes};
            \node[state] [left of=inicio] (consejo) {Consejo};
            \node[state] [left of=historial] (medida) {Medida};
        
            \node[state, align=center] [above right of=incompletos] (diaria) {\textit{Ronda}\\\textit{diaria}};
            \node[state, align=center] [below right of=incompletos] (puntual) {\textit{Ronda}\\\textit{puntual}};
            
            \node[state] [below of=ajustes] (bienvenida) {Bienvenida};
            \node[state] [left of=bienvenida] (mis_datos) {Mis datos};
            \node[state] [left of=mis_datos] (privacidad) {Privacidad};
            \node[state] [right of=bienvenida] (acerca) {Acerca de};
            \node[state] [right of=acerca] (créditos) {Créditos};

            \path[->] (click) edge node {} (splash);
            \path[->] (splash) edge node {} (inicio);
            
            \path[<->] (inicio) edge node {} (historial);
            \path[<->] (inicio) edge node {} (comunidad);
            \path[<->] (inicio) edge node {} (ajustes);
            \path[<->] (inicio) edge node {} (medida);
            \path[<->] (inicio) edge node {} (consejo);
            \path[<->, dashed] (inicio) edge node {} (incompletos);
    
            \path[<->] (historial) edge node {} (medida);
            \path[<->] (historial) edge node {} (comunidad);
            \path[<->] (historial) edge node {} (ajustes);
            
            \path[<->] (ajustes) edge node {} (comunidad);
            \path[<->] (ajustes) edge node {} (bienvenida);
            \path[<->] (ajustes) edge node {} (mis_datos);
            \path[<->] (ajustes) edge node {} (privacidad);
            \path[<->] (ajustes) edge node {} (acerca);
            \path[<->] (ajustes) edge node {} (créditos);
            
            \path[->] (incompletos) edge node {} (diaria);
            \path[->] (incompletos) edge node {} (puntual);
    
        \end{tikzpicture}
        \caption{Grafo de navegación principal de la app}
        \label{figure:disenio:grafo_principal}
    \end{figure}

    \begin{figure}[h]
        \centering
        \begin{tikzpicture}[
                font=\small, % fuente de letra pequeña
                > = stealth, % estilo de la cabeza de la flecha
                shorten > = 1pt, % para que la flecha no toque el nodo
                node distance = 3cm, % distancia entre nodos
            ]

            \tikzstyle{every state}=[
                draw = black,
                thick,
                fill = white,
            ]

            \node[state, align=center] (diaria) {Ronda\\diaria}; 

            \node[rectangle,draw, align=center] [above left of=diaria] (notificacion) {\textit{El usuario pulsa}\\\textit{en la notificación}};
            \node[rectangle,draw, align=center] [above right of=diaria] (incompletos) {\textit{El usuario pulsa en un} \\ \textit{cuestionario incompleto}};
            
            \node[state] [right of=diaria] (inicio) {\textit{Inicio}}; 
            
            \node[state, align=center] [below of=diaria] (suicidio_diario) {Suicidio\\diario};
            \node[state, align=center] [left of=suicidio_diario] (depresion_diario) {Depresión\\diario};
            \node[state, align=center] [left of=depresion_diario] (estres_diario) {Estrés\\diario};
            \node[state, align=center] [right of=suicidio_diario] (soledad_diario) {Soledad\\diario};
            \node[state, align=center] [right of=soledad_diario] (contraste_diario) {Contraste\\ diario};
    
            \node[state] [below of=suicidio_diario] (consejo) {Consejo};

            \path[->] (notificacion) edge node {} (diaria);
            \path[->] (incompletos) edge node {} (diaria);
            \path[->] (diaria) edge node {} (inicio);
            \path[<->,dashed] (diaria) edge node {} (estres_diario);
            \path[<->,dashed] (diaria) edge node {} (depresion_diario);
            \path[<->,dashed] (diaria) edge node {} (soledad_diario);
            \path[<->,dashed] (diaria) edge node {} (suicidio_diario);
            \path[<->,dashed] (diaria) edge node {} (contraste_diario);
    
            
            \path[<->] (consejo) edge node {} (estres_diario);
            \path[<->] (consejo) edge node {} (depresion_diario);
            \path[<->] (consejo) edge node {} (soledad_diario);
            \path[<->] (consejo) edge node {} (suicidio_diario);

        \end{tikzpicture}
        \caption{Grafo de navegación en los cuestionarios diarios de la app}
        \label{figure:disenio:grafo_diario}
    \end{figure}

    \begin{figure}[h]
        \centering
        \begin{tikzpicture}[
                font=\small, % fuente de letra pequeña
                > = stealth, % estilo de la cabeza de la flecha
                shorten > = 1pt, % para que la flecha no toque el nodo
                node distance = 3.5cm, % distancia entre nodos
            ]

            \tikzstyle{every state}=[
                draw = black,
                thick,
                fill = white,
            ]
            
            \node[state, align=center] (puntual) {Ronda\\puntual}; 
            
            \node[rectangle,draw, align=center] [above left of=diaria] (notificacion) {\textit{El usuario pulsa}\\\textit{en la notificación}};
            \node[rectangle,draw, align=center] [above right of=diaria] (incompletos) {\textit{El usuario pulsa en un} \\ \textit{cuestionario incompleto}};
            
            \node[state] [right of=diaria] (inicio) {\textit{Inicio}}; 
            
            \node[state, align=center] [below of=puntual] (depresion_puntual) {Depresión\\puntual};
            \node[state, align=center] [left of=depresion_puntual] (estres_puntual) {Estrés\\puntual};
            \node[state, align=center] [right of=depresion_puntual] (soledad_puntual) {Soledad\\puntual};
    
            \node[state] [below of=depresion_puntual] (consejo) {Consejo};
    
            \path[->] (puntual) edge node {} (inicio);
            \path[->] (notificacion) edge node {} (puntual);
            \path[->] (incompletos) edge node {} (puntual);
            
            \path[<->,dashed] (puntual) edge node {} (estres_puntual);
            \path[<->,dashed] (puntual) edge node {} (depresion_puntual);
            \path[<->,dashed] (puntual) edge node {} (soledad_puntual);
    
            \path[<->] (consejo) edge node {} (estres_puntual);
            \path[<->] (consejo) edge node {} (depresion_puntual);
            \path[<->] (consejo) edge node {} (soledad_puntual);

        \end{tikzpicture}
        \caption{Grafo de navegación en los cuestionarios puntuales de la app}
        \label{figure:disenio:grafo_puntuales}
    \end{figure}