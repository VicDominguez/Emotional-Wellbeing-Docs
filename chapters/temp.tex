\chapter{Temporal}
    \section{Setup del proyecto}
    
    \section{Sprint \#0}
        Este sprint se desarrolló entre el 5 y el 21 de septiembre, y consistió en la selección y puesta en marcha de las herramientas que se utilizarían para el desarrollo de este proyecto.  
        
        \subsection{Git y Github}
        
            Git es sistema de control de versiones open-source (bajo la licencia GNU GPL v2) diseñado por Linus Torvalds, siendo muy popular entre los equipos de desarrollo software. Si bien Git almacena y gestiona cambios en archivos de cualquier tipo, está pensado para ser más eficiente en los archivos de código fuente.
            
            Por otra parte, Github es una plataforma en línea que ofrece alojamiento y gestión de proyectos que utilicen Git. Para este proyecto, además del alojamiento, utilizamos sus herramientas de CI/CD (GitHub Actions), por la que a través de un fichero .yml podemos configurar un pipeline de acciones con las que realizar los procesos que queramos automáticamente. Además, pueden establecerse flujos de trabajos por ramas de desarrollo u otras condiciones, tales como formatos de tags, tipos de eventos...
            
        
        \subsection{\LaTeX y Overleaf}
        \LaTeX como medio de creación de la presente memoria y Overleaf como editor de dicho lenguaje.

    \section{Desarrollo}
        \subsection{Inyección de dependencias (Dagger Hilt)}
        \subsection{Cifrado de la base de datos (SQLCipher)}
        \subsection{Navegación entre ventanas (Compose Destination)}
        \subsection{Planificación de tareas (Work Manager)}
        \subsection{Comunicación con el servidor (Retrofit)}
        \subsection{Ajustes del usuario (Datastore)}
        \subsection{Localización de la app (Strings.xml y Lingliver)}
        \subsection{Onboarding (Lottie)}
        \subsection{Gráficas (Vico)}
        \subsection{Flujos de integración continua (GitHub Actions)}
            
            
            