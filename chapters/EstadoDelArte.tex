\chapter{[En curso] Estado del arte}
\label{chapter:estado_arte}

\chapquote{Solo sé que no se nada.}{Sócrates}


En este capítulo se introduce una revisión del estado del arte y del estado de la cuestión en lo que respecta al marco teórico del proyecto que se propone en este \gls{tfm}, así como una descripción sobre los sistemas que existen en el mercado o que han sido reportados en la literatura, los cuales presentan aspectos comunes con la solución propuesta.
%\section{Energías renovables: ¿Qué son?}


\section{Análisis de la situación actual}

En esta sección, se pone en valor y/o contraste la información introducida en las secciones anteriores, de tal forma que pueda realizarse un análisis del entorno en el que se desarrolla este proyecto, así como también se pueda poner de manifiesto la relevancia y la idoneidad del desarrollo propuesto. 


\todo[inline]{Aqui si hay que introducir los wearbles}

\subsection{Proyectos relacionados}

Tenemos aqui el proyecto DemonicSalmon de 2018 de 72 universitarios entre 18 y 23 años. Datos no wearables: movilidad (que mejor la de la pulsera), actividad (aqui igual), patrones de comunicación.

Student life es el mitico de 2014, el de las 10 semanas con cuestionarios. Aqui usaban datos como micro, sensor de luz, gps, bluetooth y acelerometro para determinar actividad, conversacion,sueño y localizacion.

Sus cuestionarios de contraste con PSS, PHQ9 y UCLA pero solo ANTES y DESPUES de acabar el estudio, no diarios. Tirar con las correlaciones desde aqui.

El paper de Smart Devices andWearable Technologies to Detect and
Monitor Mental Health Conditions and Stress:
A Systematic Review sirve para ver que datos valen para algo: variabilidad del ritmo cardiaco,  electroencefalograma (nope). Mirar los resultados para eso. Es de 2021.

Dataset wesad, 2018 alemania, pero si son invasivos. Ahi tiramos la trama para desmarcarnos de eso. Lo mismo con PASS, diciembre 2020 pero también pilla ritmo caridaco.

\subsection{Contribución de la solución propuesta}

Usamos datos de wearables compatibles con varios fabricantes.
Cuestionarios diarios además de los de 12 semanas, tanto mañana y noche.