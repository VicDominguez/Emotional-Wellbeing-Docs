\chapter{Presupuesto}
\label{chapter:presupuesto}

\chapquote{Lo perfecto es enemigo de lo bueno.}{Cita atribuida a Voltaire}

El presupuesto de este proyecto se puede desglosar en dos partes: costes relativos al material utilizado y sueldos propuestos.


\section{Coste de los materiales}

Para la estimación de los materiales, en aras de la simplicidad, no serán tenidos en cuenta gastos tales como el consumo energético de los componentes hardware o el coste del ordenador de desarrollo software.  Asimismo, en esta categoría incluimos los materiales tanto tangibles, como el hardware, como intangibles, como las licencias de desarrollo.

La siguiente lista contiene los materiales obtenidos a coste cero, ya sea por su propia naturaleza o por acuerdos que dispone la universidad:
\begin{itemize}
    \item Licencia de los entornos de desarrollo Android Studio y Pycharm.
    \item Módulos software utilizados, tales como MongoDB, las API de Android, librerías externas como Vico..
    \item Licencia de Github, para el alojamiento del código y la ejecución del \textit{pipeline} de CI/CD.
\end{itemize}

El resto de los costes materiales se muestra en la tabla \ref{tab:costes_materiales}

\todo{¿Info de las pulseras?}
\begin{table}
    \centering
    \begin{tabular}{|c|c|c|c|l|} \hline 
         Elemento&  Descripción&  Cantidad& Precio unitario&Precio total\\ \hline 
         Pulsera Fitbit X&  Pulsera para la obtención de datos&  X&  X&X\\ \hline 
    \end{tabular}
    \caption{Tabla con los costes de los materiales}
    \label{tab:costes_materiales}
\end{table}

\section{Sueldos propuestos}

Como en todo proyecto, una parte imprescindible para el desarrollo es el capital humano. Para la realización de este proyecto, se buscaría un perfil con la siguiente lista de atribuciones:

\begin{itemize}
    \item Capacidad para diseñar una solución que combine hardware y software. Construcción de dichos diagramas mediante estándares como SysML.
    \item Conocimientos en Ingeniería del Software: arquitecturas, patrones y metodologías software.
    \item Conocimientos de programación, tanto en Android para el desarrollo de aplicaciones nativas, como en Python para del desarrollo de protopiado rápido de aplicaciones del lado del servidor.
    \item Capacidad para el diseño e implementación de base de datos, tanto relacionales como no relacionales.
    \item Nociones tanto de diseño de interfaces de usuario como de la propia experiencia de usuario.
    \item Conocimiento teórico redes de computadores inalámbricas, tanto aquellas de área personal como de área local. 
    \item Nociones de programación en red y de construcción y uso de API REST.
    \item Conocimiento de las diferentes técnicas de cifrado, tanto para el propio cifrado de datos como para el uso de comunicaciones seguras.
    \item Creación de documentación, tanto interna al proyecto para otros desarrolladores como dirigida al público general.
\end{itemize}

Para estimar el salario de ese perfil, podemos establecer un paralelismo con el rol de Ingeniero de Software de Sistemas, el cual debería complementarse con una formación adicional en el desarrollado Android. Según el reputado portal de empleo Glassdoor, el sueldo medio para el puesto de Ingeniero de Software en España (redireccionado desde Ingeniero de Software de Sistemas) es de 39.000 euros al año \cite{noauthor_sueldo_nodate}.

Por otra parte, según el convenio "XX Convenio colectivo nacional de empresas de ingeniería; oficinas de estudios técnicos; inspección, supervisión y control técnico y de calidad" \cite{ministerio_de_trabajo_y_economia_social_resolucion_2023} la jornada máxima anual de un Ingeniero de Software constaría de 1.792 horas anuales, lo que arroja un salario de 21,76 euros/hora.

Por último, teniendo en cuenta que el presente TFM consta de 30 créditos ECTS, y suponiendo una equivalencia de 30 horas por cada crédito, serían necesarias 900 horas para la realización del proyecto, suponiendo un total de 19.584 euros.
