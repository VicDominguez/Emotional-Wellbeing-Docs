\chapter{Figuras, tablas, ecuaciones y fuentes}

En este capítulo vamos a hablar de los elementos denominados ``flotantes'' o \textit{floats}. Estos elementos son bloques de contenido que ``flotan'' por la página hasta que \LaTeX\space los coloca donde considera a través de ciertos algoritmos.

Lo general es declarar el elemento flotante inmediatamente \textbf{después} del párrafo donde se ha referenciado, y después dejar que \LaTeX\space elija el mejor sitio. Y si después de haber escrito todo el documento hay algo que no cuadre, pues ahí modificarlo.

Una pregunta típica es la siguiente: ``¿oye, y si no hago referencia a un elemento flotante, dónde lo pongo?'' La respuesta es sencilla: no se pone. Generalmente, si hay algo que no merece ser nombrado, tampoco tiene mucho sentido que aparezca. Remarco eso de ``generalmente''; lo mismo existe algún caso donde sí tiene sentido, pero así, para empezar, es muy raro.

Aunque más adelante veremos los diferentes tipos de \textit{floats}, en caso de que queramos modificar su comportamiento todos tienen los especificadores de posición indicados en el cuadro~\ref{tab:floats-options}:

\begin{table}
    \caption{\label{tab:floats-options}Opciones para los elementos flotantes de \LaTeX}
    \centering
    \begin{tabular}{@{}ll@{}}
        \toprule
        \textbf{Especificador} & \textbf{Acción} \\
        \midrule
        H & Colocar exactamente en el sitio indicado                           \\
        h & Colocar aproximadamente en el sitio indicado                       \\
        t & Colocar al comienzo de la página                                   \\
        b & Colocar al final de la página                                      \\
        p & Colocar en una página exclusiva para elementos ``flotantes''       \\
        ! & Forzar las indicaciones y obviar los mecanismos internos de \LaTeX \\
    \bottomrule
    \end{tabular}
\end{table}

\section{Figuras}

El código siguiente (listado~\ref{lst:figure-basic-insert}) renderizará una imagen.

\begin{lstlisting}[language=python, caption=Inserción de una figura,label={lst:figure-basic-insert}]
\begin{figure}[H]
    \centering
    \caption{\label{fig:img-vault-boy} Vault Boy approves that}
    \includegraphics[width=0.25\textwidth]{figures/vault-boy.png}
\end{figure}
\end{lstlisting}

La sintaxis es bastante autoexplicativa. El entorno \texttt{figure} es el que delimita el contenido de la figura. El comando \texttt{centering} determina que se tiene que centrar y, aunque hay varias formas de hacerlo, creemos que esta es la más sencilla. Tras éstos, el comando \texttt{caption} determina el pie de imagen, el cual además incluye una etiqueta (comando \texttt{label}) que sirve para referenciar. Por último, se incluye la imagen con el comando \texttt{includegraphics}.

En definitiva, la imagen (figura~\ref{fig:img-vault-boy}) se mostrará con un ancho igual a la mitad del ancho que ocupa el texto, centrada, con un pie de foto y una etiqueta para referenciar.

\begin{figure}[H]
    \centering
    \caption{\label{fig:img-vault-boy} Vault Boy approves that}
    \includegraphics[width=0.25\textwidth]{figures/vault-boy.png}
\end{figure}

El comando \texttt{includegraphics} puede importar los formatos típicos de imagen, como jpeg, png o pdf. También admite una serie de opciones como rotación, alto, ancho (éste le hemos especificado con \texttt{width}), etcétera. ¡Ojo! Siempre que se pueda, hay que intentar insertar imágenes vectoriales. De esta manera, se mantiene la calidad de la imagen. Si no, puede ocurrir que se pixele y no quede nada bien.

\subsection{Subfiguras}

¿Y qué pasa cuando queremos incluir múltiples imágenes dentro de una figura? Bueno, pues aquí hay que usar el entorno \texttt{subfigure}. En el listado~\ref{lst:subfigure-basic-insert} vemos un ejemplo de cómo se manejan.

\begin{lstlisting}[language=python, caption=Inserción de varias subfiguras,label={lst:subfigure-basic-insert}]
\begin{figure}[H]
	\centering
	\begin{subfigure}{.3\textwidth}
		\includegraphics[width=\linewidth]{figures/vault-boy.png}
		\caption{\label{fig:subfigure-1}Vault Boy 1}
	\end{subfigure}%
	\begin{subfigure}{.3\textwidth}
		\includegraphics[width=\textwidth]{figures/vault-boy.png}
		\caption{Vault Boy 2}
	\end{subfigure}%
	\begin{subfigure}{.3\textwidth}
		\includegraphics[width=\textwidth]{figures/vault-boy.png}
		\caption{Vault Boy 3}
	\end{subfigure}
	\caption{\label{fig:subfigures}Todos los Vault Boy}
\end{figure}
\end{lstlisting}

En realidad cada subfigura se trata como una figura normal, pero en relación con el \textit{float} contenedor. Cuando a una subfigura se le especifica un ancho, se le está diciendo al compilador de qué ancho es esa subfigura en concreto (en nuestro caso 0.3 veces el ancho de la línea). Sin embargo, a la imagen se le da un ancho total de \texttt{linewidth}, porque al estar dentro de su espacio de subfigura, el ancho ha cambiado. El resultado es el que se observa en la figura~\ref{fig:subfigures}. Por cierto, también podemos referenciar a los pies de las subfiguras (e.g. Así:~\ref{fig:subfigure-1}).

\begin{figure}[H]
	\centering
	\begin{subfigure}{.3\textwidth}
		\includegraphics[width=\linewidth]{figures/vault-boy.png}
		\caption{\label{fig:subfigure-1}Vault Boy 1}
	\end{subfigure}%
	\begin{subfigure}{.3\textwidth}
		\includegraphics[width=\textwidth]{figures/vault-boy.png}
		\caption{Vault Boy 2}
	\end{subfigure}%
	\begin{subfigure}{.3\textwidth}
		\includegraphics[width=\textwidth]{figures/vault-boy.png}
		\caption{Vault Boy 3}
	\end{subfigure}
	\caption{\label{fig:subfigures}Todos los Vault Boy}
\end{figure}

\section{Ecuaciones}

La facilidad de composición de ecuaciones es una de las cosas que más atrae de \LaTeX\space a muchos autores. \LaTeX mantiene dos renderizadores diferentes, uno para el texto y otro para las ecuaciones, denominados modo párrafo y modo matemático\footnote{Existe un tercer modo, denominado \textit{LR mode} o \textit{left-to-right mode}, raramente utilizado y que no trataremos aquí}. El modo párrafo es el modo por defecto y no se le llama explícitamente. Al modo matemático, sin embargo, se le invoca de varias maneras diferentes.

\subsection{Modo en párrafo}

La forma más común es la forma ``en línea'', donde el texto para el modo matemático se encierra entre dos signos \$. Por ejemplo, veamos la frase del listado~\ref{lst:in-line-equation}.

\begin{lstlisting}[language=tex,caption=Ejemplo de inserción de fórmulas en linea,label=lst:in-line-equation]
El pequeño teorema de Fermat dice que si $p$ es un número primo, entonces, para cada número natural $a$, con $a>0$, $a^p \equiv a (\mod p)$
\end{lstlisting}

La frase quedaría como sigue:

El pequeño teorema de Fermat dice que si $p$ es un número primo, entonces, para cada número natural $a$, con $a>0$, $a^p \equiv a (mod p)$

\subsection{Ecuaciones en bloque}

Cuando en lugar de poner una ecuación dentro de un párrafo existente la queremos insertar en su propio espacio independiente hacemos uso de los entorno \texttt{equation} o \texttt{align}, dependiendo de si queremos una o más ecuaciones en el bloque, respectivamente. Por ejemplo, en el caso de una única ecuación, sería similar al ejemplo siguiente:

\begin{minipage}[c]{.5\textwidth}
\begin{lstlisting}[language=tex]
\begin{equation}
	\int_a^b f'(x) \, dx=f(b)-f(a)
\end{equation}
\end{lstlisting}
\end{minipage}%
\begin{minipage}[c]{.5\textwidth}
\begin{equation}
    \int_a^b f'(x) \, dx=f(b)-f(a)
\end{equation}
\end{minipage}

Sin embargo, en el caso de que quisiésemos más de una ecuación en el mismo bloque haríamos uso del carácter \& para indicar en qué punto se alinean las ecuaciones; por ejemplo:

\begin{minipage}[h]{.5\textwidth}
\begin{lstlisting}[language=tex]
\begin{align}
    u  &= \arctan x \\ 
    du &= \frac{1}{1 + x^2}dx
\end{align}
\end{lstlisting}
\end{minipage}%
\begin{minipage}[h]{.5\textwidth}
\begin{align}
    u  &= \arctan x \\ 
    du &= \frac{1}{1 + x^2}dx
\end{align}
\end{minipage}

Por último, si no tenemos por qué referenciarlas en el texto, podemos hacer uso de los entornos \texttt{equation*} y \texttt{align*}

\begin{minipage}[c]{.5\textwidth}
\begin{lstlisting}[language=tex]
\begin{equation*}
	\int_a^b f'(x) \, dx=f(b)-f(a)
\end{equation*}
\end{lstlisting}
\end{minipage}%
\begin{minipage}[c]{.5\textwidth}
\begin{equation*}
	\int_a^b f'(x) \, dx=f(b)-f(a)
\end{equation*}
\end{minipage}

\section{Cuadros y tablas}

Los cuadros son una forma muy eficaz de presentar información. En los resultados de casi cualquier trabajo existen cuadros de algún tipo para que los datos se comprendan de un único vistazo (o para que al menos sea más fácil identificarlos.

\notebox{
    \textbf{¿Por qué ``cuadro'' en lugar de ``tabla''?}
    
    Tal y como se indica en el \href{http://www.aq.upm.es/Departamentos/Fisica/agmartin/webpublico/latex/FAQ-CervanTeX/FAQ-CervanTeX-6.html}{FAQ de CervanTex}, \textit{table} (inglés) y \textit{tabla} (español) son falsos amigos; el inglés \textit{table} tiene un sentido más general que el español \textit{tabla}, cuyo uso es únicamente para aquellos cuadros dedicados a la disposición de números (e.g. tabla de multiplicar o tabla de logaritmos).
}

Sin embargo, las tablas suelen ser bastante complicadas en \LaTeX. Para no escribir demasiado, la respuesta para casi toda maquetación de tabla está en \href{https://www.tablesgenerator.com/}{https://www.tablesgenerator.com/}. En serio, no perdáis el tiempo si no es estrictamente necesario. La maquetáis visualmente, la generáis (con estilo \textit{booktabs}) y a correr.

\section{Código fuente}

Para la gestión de los listados de código fuente se utiliza el paquete \texttt{listings}. El estilo usado es una modificación\footnote{En realidad la modificación es cambiar el fondo de crema a blanco} de \href{https://ethanschoonover.com/solarized/}{Solarized} desarrollado por Ethan Schoonover.

Existen muchas formas diferentes de incluir listados de código en una memoria. Aquí introducimos los más comunes.

\subsection{Código en párrafo}

\subsection{Bloques de código}

Insertar código en párrafos no es tan común como insertar bloques enteros. Para ello haremos uso del entorno \texttt{lstlisting}. Por ejemplo:

\lstinputlisting[language=tex, firstline=1, lastline=6]{sources/adding-blocks.tex}

Nos daría el siguiente resultado:

\lstinputlisting[language=python, firstline=2, lastline=5]{sources/adding-blocks.tex}

Una cosa que hay que tener en cuenta es que dentro de un entorno \texttt{lstlisting} se ignoran todos los comandos de \LaTeX\space\footnote{En realidad no todos, y si no mira en estos fuentes cómo hemos metido el fin de bloque \texttt{lstlisting} dentro del propio bloque.} y el texto se imprime tal y como se ha introducido. Esto incluye los tabuladores y espacios de principio de línea.

Al igual que en el código de párrafo, también podemos especificar en qué lenguaje está escrito el código para que se resalten en éste las palabras reservadas. Por ejemplo:

\lstinputlisting[language=tex, firstline=8, lastline=14]{sources/adding-blocks.tex}

Nos daría como resultado el siguiente bloque de código:
 
\lstinputlisting[language=python, firstline=9, lastline=12]{sources/adding-blocks.tex}

\subsection{Código directamente desde fichero}

Esta forma es muy común, ya que se usa tanto para hacer referencia al código fuente de la aplicación directamente, como a código separado en ficheros para mantener el tamaño de la memoria manejable.

Suponiendo que tenemos el fichero \texttt{sources/snippets.py}, para incluirlo entero basta con usar el comando \texttt{lstinputlisting}:

\begin{lstlisting}[language=TeX]
\lstinputlisting[language=python]{sources/snippets.py}
\end{lstlisting}

Con este comando conseguiríamos el siguiente resultado:

\lstinputlisting[language=python]{sources/snippets.py}

En caso de que se desease importar sólo parte del fichero, se pueden indicar las filas que delimitan el trozo de código. Por ejemplo:

\begin{lstlisting}[language=TeX]
\lstinputlisting[
    language=python,
    firstline=4,
    lastline=5
]{sources/snippets.py}
\end{lstlisting}

Esto haría que sólo se imprimiesen las filas 4 y 5, correspondientes a la segunda función:

\lstinputlisting[language=python, firstline=4, lastline=5]{sources/snippets.py}

Ambas opciones son opcionales, y los valores por defecto de \texttt{firstline} y \texttt{lastline} serán el principio y el final del fichero respctivamente.

\subsection{Etiquetando bloques de código}

Al igual que con el resto de bloques \textit{float} (e.g. figuras o tablas), se pueden\footnote{Pongo \textit{pueden} porque es opcional, pero en realidad se \textbf{deben} poner, porque si no los listados de fuentes quedan horribles, como el de esta plantilla por ejemplo (échale un vistazo si no lo has hecho antes).} (\textbf{deben}) añadir pies de texto a los bloques de código, lo cual hace más legible su cometido.

Para ello basta con añadir el argumento \texttt{caption} a las opciones del bloque. Por ejemplo:

\lstinputlisting[language=tex, firstline=15, lastline=20]{sources/adding-blocks.tex}

Nos daría como resultado el siguiente bloque de código:
 
\lstinputlisting[language=python, firstline=16, lastline=19, caption=Función para determinar cuando una palabra \texttt{w1} es anagrama de otra palabra \texttt{w2}]{sources/adding-blocks.tex}

\section{Referencias cruzadas: etiquetas (\textit{labels})}

Las etiquetas (\textit{label}) son una herramienta muy útil en el proceso de composición tipográfica. Se puede pensar en ellas como punteros a zonas de interés del documento, de tal manera que se les pueda referenciar sin necesidad de conocer su posición final en la composición.

Por ejemplo, lo normal es que en un libro, a la hora de referenciar una figura, aparezca una frase del estilo ``[\ldots] como muestra la Figura 3 [\ldots]''. Lo que es bastante raro son las frases del estilo ``[\ldots] como muestra la Figura de los moñecos amarillos [\ldots]'' o ``[\ldots] como muestra la siguiente Figura [\ldots]''\footnote{Sí, bueno, quizá la segunda frase no es tan rara, pero siempre es preferible referenciar directamente a dar posiciones relativas.}.

Una de las propiedades más útiles y, en ocasiones, infravaloradas de \LaTeX es la facilidad y potencia de su sistema de etiquetado. Este sistema permite referenciar tablas, listados de código fuente, ecuaciones, capítulos, secciones, etc., con facilidad y flexibilidad. Además, \LaTeX las numera y referencia automáticamente, cambiando la numeración en función de las adiciones y supresiones sin que el autor tenga que hacer nada.

Para referenciar un elemento, lo primero que hay que crear es una etiqueta \textbf{después} del elemento a referenciar. Esto ya lo hemos visto anteriormente, por ejemplo en el listado~\ref{lst:figure-basic-insert}. Si nos fijamos, se declara una etiqueta justo después de la etiqueta \texttt{caption} con el nombre \texttt{fig:img-vault-boy}. De esta manera, podemos referenciar varios indicadores de la figura, como se muestra en el listado~\ref{lst:basic-references}

\begin{lstlisting}[language=tex, caption=Referenciando una figura y su página,label={lst:basic-references},]
Mira la Figura~\ref{fig:img-vault-boy} en la página~\nameref{fig:img-vault-boy}.
\end{lstlisting}

Dicho listado daría el siguiente resultado:

\blockquote{Mira la Figura~\ref{fig:img-vault-boy} titulada \nameref{fig:img-vault-boy} en la página~\pageref{fig:img-vault-boy}.}

\notebox{
    \textbf{¿Por qué a mí me aparece el símbolo \texttt{??} en lugar de una referencia?} Pues lo más seguro es que sea un error a la hora de escribir la etiqueta, un type. Menos común, pero también puede pasar, es que el documento no se haya compilado bien. Hay que tener en cuenta que \LaTeX\space fue creado en una época donde las máquinas tenían poca (¡poquísima!) RAM, y para funcionar lo que se hacían eran varias compilaciones sobre el documento, almacenando los valores temporales en ficheros. Y como nadie quiere perder tiempo en cambiar y \textit{debuggear} algo que funciona estupendamente bien, no se reimplementa. De todas formas, si te animas, ahí tienes un buen proyecto que si lo sacas adelante te va a hacer muy famoso.
}
