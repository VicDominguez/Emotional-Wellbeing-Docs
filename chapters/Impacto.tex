\chapter{[En curso] Impacto social y medioambiental}
\label{chapter:aspectos}

\chapquote{Solo se vive una sola vez, pero si hiciste las cosas bien una vez es suficiente.}{Cita atribuida a Mae West}


\todo[inline]{Hablar aqui de la privacidad de los datos antes de health connect (usa, gpd)}
\todo[inline]{Estamos en la cresta de la ola al venir todo para Android 14}

\todo[inline]{Hablar de la protección de datos tan sensibles}
\todo[inline]{Hablar de las TIC para la parte psicológica BIEN, como decía Sandra, etico vamos}

En este capítulo se recogen los beneficios que la implantación del proyecto desarrollado podría generar tanto a nivel medioambiental como su impacto social.




\section{Aspectos éticos, sociales y económicos}

\subsection{Aspectos éticos}

\subsection{Aspectos sociales}

\subsection{Aspectos económicos}

%En primer lugar, mi motivación principal es el deseo de querer hacer algo para ayudar, aunque sea modestamente, a la mejora de la salud mental dentro de nuestra sociedad. Para bien o para mal, en mi vida he estado expuesto a la importancia de la salud mental. En ocasiones he podido ver a personas que se sienten realizadas y felices con situaciones que, a los ojos de terceras personas, quizás no sean precisamente ideales; pero por otra parte he sentido impotencia y dolor al ver cómo la depresión u otras enfermedades destruyen la vida de un ser querido. También he vivido incertidumbre y pánico cuando ves que hay algo dentro de ti que no está bien y que no sabes ni por qué, ni cómo resolver. Son situaciones cuanto menos, impactantes.
    
%Esa impotencia que sientes cuando ves a alguien cercano sufrir y no tener idea de cómo ayudar. También esa  Esas fuerzas son suficientes para motivarte de, cuando sea posible, intentar tomar cartas en este asunto. Quizás no ya para ayudar a alguien próximo, sino para intentar reducir el número de personas que puedan sentirlo alguna vez.    
    
%sino de un conjunto de vivencias personalmente. La Salud Mental es algo que, a pesar de que gran parte de la sociedad no quiera ver, está ahí, tanto para lo bueno o para lo malo. Ver de primera mano cómo una depresión puede condicionar la vida de un ser querido y sentir impotencia por no poder ayudar, sentir en tus propias carnes como a veces no te encuentras bien y no sabes por qué... es motivación suficiente como para interesarte por este tema.

\section{Contexto medioambiental}

Los Objetivos de Desarrollo Sostenible \textit{(ODS)}....