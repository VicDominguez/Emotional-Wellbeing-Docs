\chapter{Conclusiones y líneas futuras}
\label{chapter:conclusiones}

\chapquote{Realiza cada una de tus acciones como si fuera la última de tu vida.}{Marco Aurelio}

En este capítulo se reúnen las conclusiones obtenidas durante el desarrollo del proyecto y se proponen una serie de líneas futuras para continuarlo.

\section{Conclusiones}

    En primer lugar, se ha podido constatar en la realización de este \gls{tfm} la existencia de un marco téorico-práctico sólido para contribuir a la mejora de la salud mental de nuestra sociedad mediante soluciones informáticas. Si bien la labor de los profesionales de la Psicología es insustituible en el tratamiento de enfermedades como la depresión y se necesita a su vez una solución integral para resolver cuestiones como las listas de espera, se ha evidenciado que la Informática puede proveer soluciones ,que permiten apoyar y ahondar en la detección precoz de estos trastornos.

    Aunque durante la realización de este \gls{tfm} no se han producido avances teóricos significativos, el marco práctico si los ha experimentado. Actualmente, los \glspl{wearable} son un elemento cada vez más presente en nuestras vidas, democratizando el acceso a datos como la \gls{vfc}, o la realización de pruebas cada vez más complejas, como los \glspl{ecg}. 
    
    Estos dispositivos permitirían resolver el gran escollo mostrado por los trabajos previos: la recolección sistemática de datos mediante \glspl{no-invasiva}. Iniciativas como \textit{Salud Conectada}, a pesar de sus limitaciones; reducen sensiblemente la barrera técnica para desarrollar nuevos proyectos y amplían la potencial base de usuarios; mientras se garantiza la privacidad de los mismos.

    Por otra parte, este proyecto ha supuesto una profunda toma de contacto con el estado actual de las aplicaciones móviles en Android. El desarrollo e implantación de nuevos componentes, tales como \textit{Jetpack Compose} y \textit{Material Design 3}, han supuesto un indudable salto de calidad en el sector. Estas herramientas han permitido implementar una interfaz gráfica \gls{responsive} y de cierta calidad; garantizando, entre otras cuestiones, el cumplimiento de ciertas secciones del estándar \gls{wcag} 2.0 de accesibilidad, o el soporte de funciones demandadas por los usuarios, como el modo oscuro.
    
    No obstante, el estado actual de estos componentes está lejos de ser idílico. Al margen de los problemas ya descritos en la Sección \ref{section:problemas}, la ausencia de fabricantes de \glspl{wearable} como Xiaomi, Huawei o claramente Apple en \textit{Salud Conectada} limitan el impacto de este tipo de proyectos. Asimismo, la elevada curva de dificultad y los problemas no plenamente resueltos de \textit{Jetpack Compose} y \textit{Material Design 3} han supuesto un elevado consumo de recursos, impidiendo profundizar en otras áreas.

    Estos contratiempos, aunque esperables en un \gls{tfm}, resultan especialmente dolorosos a juicio del autor, ya que han impedido el desarrollo de todos los requisitos planteados y en general han restringido el impacto que el sistema hubiera tenido en la \gls{etsisi}. Cuestiones como la política de privacidad han bloqueado la puesta en marcha real de este sistema y su distribución en tiendas virtuales, mientras que funcionalidades clave como la detección precoz han tenido que ser limitadas.
    
    En particular, la detección precoz solo ha podido implementarse mediante la realización de cuestionarios. Si bien se han implementado cuestionarios diseñados expresamente por psicólogas, ha quedado pendiente de explotar el enorme potencial de los datos recolectados; o la puesta en marcha de proyectos que permitieran evaluar la efectividad de la solución propuesta. En la Sección \ref{section:lineas_futuras} serán planteadas una serie de mejoras para su implantación en futuros trabajos.

    Paralelamente, el esfuerzo dedicado a la construcción, pruebas y despliegue del componente servidor en general,  y de la \gls{api} en particular, ha sido satisfactorio. Se ha constatado una vez más el excepcional trabajo realizado por los diversos actores en el ecosistema de Python permitiendo implementar con sencillez y elegancia componentes que aportan un gran valor añadido.

    Asimismo, en una capa más personal, este proyecto ha servido para que el autor pudiera experimentar y comprender definitivamente numerosos conceptos relacionados con la \gls{iss} y el desarrollo móvil. Si bien se han sufrido numerosos contratiempos, proyectos de esta envergadura permiten poner en práctica conocimientos o planteamientos que resultan imposibles de realizar en el marco de las asignaturas de grado o de máster. Donde las clases ponen las bases del conocimiento, estos proyectos permiten dar un salto cualitativo que beneficia al conjunto de la sociedad. 

    Continuando en esta línea, se anima especialmente tanto al profesorado como al colectivo de estudiantes de la \gls{etsisi} a realizar proyectos que puedan tener un claro impacto social. La realización de este proyecto ha supuesto para el autor descubrir la importancia de la Salud Mental y del potencial que disponen las \gls{tic} para cambiar la vida de las personas. La tecnología es producto del contexto social y de las necesidades y retos que afronta la sociedad, estando en manos de los profesionales en qué lugar y con qué profundidad aportará valor su conocimiento. 
    
    Por último, a juicio del autor, la tecnología es neutral por definición y el progreso, imparable. Ante la certeza de que \textit{``lo único constante es el cambio''}, ya planteada en la Antigua Grecia por el filósofo Heráclito, la solución no reside en limitar o prohibir el uso de la tecnología. Existe espacio suficiente para que los ingenieros, en colaboración con los profesionales de otras disciplinas, puedan crear soluciones; o para adaptarse el cuerpo normativo para limitar las consecuencias negativas de la tecnología. En última instancia no se debe olvidar que las sociedades son soberanas de decidir su propio destino.
    
\section{Lineas futuras}
    \label{section:lineas_futuras}
    
    En primer lugar, el trabajo futuro más prioritario es la implementación de los requisitos que no han sido cubiertos en este primer desarrollo. Dichos requisitos, junto con sus líneas de actuación propuestas, son:

    \begin{itemize}
        \item \ref{req:no_funcionales:datos_solo_cientificos} \textit{El acceso a los datos anonimizados de los usuarios solo estará permitido para propósitos científicos.}
        
        Para ello se necesita definir una política de acceso a los datos, la cual debe aceptarse por los analistas de datos. En este aspecto es recomendable contar con asesoría legal. Posteriormente, se debería diseñar e implementar un sistema de acceso que permita la administración de los diferentes usuarios.
        
        \item \ref{req:no_funcionales:ddos} \textit{El servidor que aloje los datos anonimizados deberá estar protegido antes ataques de denegación de servicio y accesos no autorizados.}

        Para cumplir con este requisito, se deberá definir e implementar un conjunto de medidas que permitan proteger a dicho servidor, en consonancia con los recursos disponibles en la \gls{etsisi}.
        
        \item \ref{req:no_funcionales:cifrado_comunicaciones} \textit{Las comunicaciones entre aplicación y servidor deberán estar cifradas.}

        La implementación de este requisito consta de las siguientes fases:
        \begin{enumerate}
            \item Generar un certificado \gls{ssl} por parte de una autoridad de certificación reconocida.
            \item Modificar la \gls{api} para usar \gls{https} con el certificado anterior.
            \item Restringir la aplicación para que únicamente acepte conexiones \gls{https}.
        \end{enumerate}
        
        \item \ref{req:no_funcionales:politica_privacidad} \textit{Se deberá definir una política de privacidad de acuerdo al \gls{rgpd} \cite{publications_office_of_the_european_union_reglamento_nodate} y a las directrices de \textit{Salud Conectada} \cite{google_preguntas_nodate}.}

        Se recomienda nuevamente contar con asesoría legal para la conformidad de dicha política. Por otra parte, dicha política debe ser accesible desde fuera de la aplicación, para garantizar que los usuarios puedan leerla antes de comenzar el uso o descarga de la aplicación. 
    \end{itemize}

    Una vez todos los requisitos planteados hayan sido satisfechos, se podría desplegar la aplicación en tiendas virtuales como \textit{Play Store}, dotando al proyecto de mucho mayor alcance y de una retroalimentación mucho más amplia y profuda.
    
    Asimismo, otra línea de trabajo propuesta es la realización de ciertas mejoras en el marco de los requisitos de usuario ya implementados. En el caso de los requisitos de seguimiento (\ref{req:usuario:seguimiento_estres}, \ref{req:usuario:seguimiento_depresion}, \ref{req:usuario:seguimiento_soledad} y \ref{req:usuario:seguimiento_suicidio}), la principal mejora sería la incorporación de un modelo basado en Inteligencia Artificial que permita mejorar el diagnóstico precoz. 
    
    Para ello se podría realizar un estudio científico que explore esta posibilidad en el que se recojan datos de los usuarios, algo que ya permite el estado actual del proyecto. Este estudio podría experimentar con los datos recogidos y/o con los datos de los estudios detallados en las Secciones \ref{sec:estado_arte:biometricos} y \ref{sec:estado_arte:smartphone}. Asimismo, la arquitectura planteada permite un despliegue sencillo, con el cual se podría evaluar la eficacia del mismo.

    En cuanto a los requisitos de consejos (\ref{req:usuario:consejo_estres}, \ref{req:usuario:consejo_depresion}, \ref{req:usuario:consejo_soledad} y \ref{req:usuario:consejo_suicidio}) se anima al resto de estudiantes y a la \gls{etsisi} a ahondar en la colaboración con profesionales de la Psicología para la elaboración de nuevas pautas. El estado actual del proyecto permite a dichos profesionales desplegarlos con muy poco coste, por lo que se podrían realizar trabajos de investigación que exploren el impacto de dichos consejos.

    Finalmente, se enumeran otras líneas de trabajo que son factibles de realizar a largo
    plazo para ampliar y refinar el proyecto:
    
    \begin{enumerate}
        \item Elaboración de un conjunto de pruebas automáticas exahustivas para la aplicación móvil, incluyendo reportes de \textit{coverage} del código.
        \item Comprobar la recolección de datos en otras pulseras, tanto de Fitbit como de otros fabricantes.
        \item Realizar agregación de datos para ciertas variables, como las pulsaciones, para aligerar el volumen de datos.
        \item Explorar cuestiones relacionadas con la infraestructura, como el despliegue de la funcionalidad del servidor mediante contenedores, lo que permitiría una mejor escalabilidad del mismo.
        \item Segmentar la comunidad de usuarios para mejorar la representatividad del conjunto.
        \item Estudiar la imputación de datos nulos con otras técnicas, tales como: regresión, \textit{Last Observation Carried Forward}, \textit{Next Observation Carried Backward}, etc. \cite{gupta_null_nodate}
        \item Añadir \textit{features} relacionadas con los cuestionarios puntuales: gráficas, notificaciones, entre otras.
        \item Diseñar e implementar una interfaz gráfica centrada en las pantallas \textit{expandidas}\footnote{Pantallas 
        que disponen de más de 840dp de ancho o 900dp de alto}.
        \item Trasladar la aplicación a más idiomas.
    \end{enumerate}
