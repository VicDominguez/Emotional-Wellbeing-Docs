\chapter{[En curso] Resultados}
\label{chapter:resultados}

\chapquote{¡Los números Mason! ¿Qué significan?}{Jason Hudson}

\section{Preparación del entorno}
En este apartado se enumeran los pasos seguidos para la preparación del entorno que ha sido utilizado en el desarrollo del sistema.

\subsection{Sistema Operativo}


\subsection{Tecnologia 1}
\subsection{Tecnologia ...}
\subsection{Tecnologia N}

Section{Implementación del caso de estudio} \label{Implementación del caso de estudio}


\section{Resultados del caso de estudio}

En este apartado se muestran algunos ejemplos de los utilización y resultados que podrían obtenerse.

\section{Problemas encontrados}

A continuación se van a detallar todos los problemas encontrados durante el desarrollo
de este Trabajo de Fin de Máster.

\begin{enumerate}
    \item Falta de experiencia del menda en apps móviles, proceso de aprendizaje
    \item Soporte de los fabricantes fuera de Fitbit, Samsung y de mala gana
    \item Madurez en Jetpack Compose y especialmente en Material Design 3
    \item Cambios de Health Connect al estar en testeo
    \item Cosas más avanzadas en vico
    \item Manejo de parámetros en las ventanas
    \item Componentes gráficos que no estaban, como el progreso circular
    \item Diseñar una arquitectura a nivel de implementación que soportarse tantas combinaciones posibles, como cuestionarios numericos y no numericos, el caso especial de suicido que se cierra antes de llegar al final...
    \item Gráficas en Android en general, y más siendo responsives
    \item Poco soporte en Android para elegir manualmente el idioma
    
    \item Dificultades para testeo
    \item Implementaciones diferentes según móvil, especialmente work manager
    
\end{enumerate}