\chapter{Análisis del sistema propuesto}
\label{chapter:analisis}

\chapquote{Cada día me gusta levantarme porque hay otro reto.}{Roger Penske}

\section{Introducción}

    \subsection{Propósito}
        El propósito de este capítulo consiste en la Especificación de Requisitos Software (ERS) correspondiente al proyecto desarrollado en este Trabajo Fin de Máster (TFM). Para la elaboración de la misma se ha tomado como base el estándar IEEE 830.

        Esta especificación recoge todas las características, objetivos, restricciones y suposiciones para el desarrollo del proyecto, definiendo claramente el sistema a desarrollar y sirviendo asimismo de referencia para la verificación y validación de la solución implantada.

        Por otra parte, dado que este proyecto tiene una naturaleza abierta y transparente, en especial al tratar datos sensibles; el público objetivo de esta especificación de requisitos es cualquier persona que por cualquier razón desee conocer las características del sistema y sobre qué criterios, restricciones o supuestos ha sido diseñado, ya que consideramos que esta transparencia es vital para que el conjunto de la sociedad pueda confiar en este proyecto.
        
    \subsection{Alcance}
        Como ya se mencionó en la sección \ref{sec:objetivos}, el objetivo principal de este Trabajo Fin de Master es la creación de un prototipo de un sistema que permita la detección precoz y la mejora de los trastornos de salud mental en la comunidad universitaria. 

        Este proyecto se fundamenta en la alarmante situación psicológica que atraviesa nuestra sociedad, partiendo de la premisa de que en numerosas ocasiones las personas desconocen factores, conductas o síntomas relacionados con problemas de salud mental, pero que mejorando la conciencia de los mismos se podría reducir el número y la gravedad de los casos.

        Para ello, este sistema se encargará de monitorizar las medidas de estrés, soledad y depresión y suicidio, en base a cuestionarios que la persona rellene diariamente. Asimismo, la aplicación recopilará los datos de su dispositivo \textit{wearable} a los que el usuario haya dado acceso, con la finalidad de construir en el marco de la escuela un conjunto de datos anonimizado con el que se pueda desarrollar en el futuro un modelo predictivo de esas medidas, con el que se pueda dar un apoyo más integral al usuario.

        Asimismo, este sistema no está planteado como un reemplazo de los profesionales de la Psicología y Psiquiatría, sino como un complemento para mejorar la sensibilidad de los usuarios sobre la evolución de su salud mental, aportando consejos y remitiendo a un profesional cuando la situación del usuario sea delicada; quedando fuera del proyecto cuestiones como el asesoramiento médico profesional o una atención psicológica personalizada.

        \todo[inline]{Tenemos que hablar aqui de la comunidad}

        Dichos consejos estarán personalizados para el nivel de gravedad (o ausencia de la misma) de cada medida, permitiendo además que el usuario pueda ver claramente cómo se ha sentido durante los últimos días, semanas o meses para cada variable.

        En cuanto a la obtención de los datos de los dispositivos, se utilizará un \textit{framework} ya desarrollado por la industria, por lo que la compatibilidad del sistema estará limitada por este factor. La extracción de datos por otros canales queda fuera del alcance, ya que los cauces legales están fuertemente restringidos. Asimismo, el sistema accederá a esos datos únicamente en modo lectura.

        \todo[inline]{Hablamos aquí de que hemos tenido la colaboración/asesoría de dos psicólogas?}
        
    \subsection{Definiciones, Acrónimos y Abreviaturas}
        TBD
        
    \subsection{Estructura del documento}
        TBD
        
\section{Descripción general del producto}

    \subsection{Perspectiva del producto}
        Bienestar Emocional se constituye bajo dos componentes fundamentales. En primer lugar, una aplicación Android construida desde cero que haga uso del \textit{framework} Salud Conectada, el cual es desde la última versión de Android parte del propio sistema operativo. Este componente nos permite la abstracción de los componentes hardware, delegando a los fabricantes la conexión de sus dispositivos con este sistema.

        En segundo lugar, se desplegará en un servidor de la Universidad un componente que permita tanto la subida de los datos del usuario como la obtención de diversas estadísticas sobre la evolución de las medidas en el resto de la comunidad universitaria.

        Como se comentó anteriormente, este proyecto está planteado como un prototipo, sirviendo como punto de partida para proyectos de futuros alumnos, los cuales puedan ampliar las funcionalidades de esta aplicación sin necesidad de rehacer gran parte del trabajo. Asimismo, se busca que este sistema sea un marco para la investigación en el campo de la Psicología, permitiendo la realización de experimentos que necesiten acceso a datos de salud o cuestionarios.


    \subsection{Características de los usuarios finales}

    TBD

    %Los usuarios finales de este sistema serán los miembros de la comunidad universitaria que deseen monitorizar su salud mental. No obstante, deberán de cumplir ciertas pautas.
    
    %\begin{enumerate}
    %    \item Los usuarios dispondrán de un teléfono con sistema operativo Android, en su versión 8 o superior.
    %    \item Los teléfonos dispondrán de conexión a Internet.
    %    \item El usuario realizará de forma habitual los cuestionarios ofrecidos por la aplicación.
    %    \item El usuario estará dispuesto a que sus respuestas a los cuestionarios sean subidas a un servidor de la universidad, con un identificador de usuario anónimo.
    %    \item Opcionalmente, el usuario dispondrá de un dispositivo \textit{wearable} de los fabricantes Fitbit o Samsung, con la aplicación del fabricante correspondiente instalada y el dispositvo conectado a su teléfono.
        
    %\end{enumerate}
        
    \subsection{Restricciones generales}
        \begin{enumerate}[label=\textbf{RG-\arabic*}]
            \item La aplicación móvil se ejecutará sobre el sistema operativo Android. Concretamente, se tomará Android 14 como referencia.
            \item La aplicación móvil especificará Android 9 como versión mínima, pudiéndose ejecutar la aplicación en cualquier versión entre la mínima y la referente.
            \item La comunicación con los dispositivos wearables se realizará mediante el \textit{framework} de Android \textit{Health Connect}.
            \item La aplicación móvil será implementada utilizando el lenguaje de programación Kotlin.
            \item El componente servidor será implementado mediante el lenguaje de programación Python.
            \item Se utilizará como dispositivo \textit{wearable} la pulsera Fitbit Inspire 2.
        \end{enumerate}
    
    \subsection{Suposiciones}

        \begin{enumerate}[label=\textbf{SUP-\arabic*}]
            \item Se supone que el usuario de la aplicación dispone de conexión a Internet.
            \item Se supone que el usuario de la aplicación es el mismo que utiliza el dispositivo \textit{wearable}.
            \item Se supone que las respuestas del usuario a los cuestionarios son veraces y verídicas.
            \item Se supone que el usuario rellena frecuentemente los cuestionarios.
            \item Se supone que el usuario utiliza con cierta asiduidad el dispositivo \textit{wearable}.
            \item Se supone que el servidor donde se aloja el componente homónimo está disponible.
            \item Se supone que el usuario desactiva la optimización de batería de la aplicación móvil.
            \item Se supone que habrá más de un usuario utilizando la aplicación diariamente.
        \end{enumerate}
        
    \subsection{Dependencias}
        \todo[inline]{Alguna más quizá?}
        \begin{enumerate}[label=\textbf{DEP-\arabic*}]
            \item Para la lectura de los datos del dispositivo \textit{wearable}, el fabricante debe integrar su aplicación móvil con el \textit{framework} \textit{Health Connect}, asegurando la escritura de los datos recogidos.
            \item Para la ejecución de las tareas recurrentes, es necesario que el sistema operativo otorgue recursos de ejecución a la aplicación móvil.
        \end{enumerate}

\section{Requisitos específicos}

    \subsection{Requisitos de usuario}

        \begin{enumerate}[label=\textbf{\texttt{RU-\arabic*}}]
            \item Como usuario, quiero realizar un seguimiento del estrés.
            \item Como usuario, quiero realizar un seguimiento de la depresión.
            \item Como usuario, quiero realizar un seguimiento de la soledad no deseada.\footnote{Para simplificar la redacción de los requisitos, mientras no se especifique lo contrario interpretaremos soledad como soledad no deseada.}
            \item Como usuario, quiero visualizar el resultado de la última medición del estrés.
            \item Como usuario, quiero visualizar el resultado de la última medición de la depresión.
            \item Como usuario, quiero visualizar el resultado de la última medición de soledad.
            \item Como usuario, quiero disponer de al menos un consejo para cada nivel categórico de estrés.
            \item Como usuario, quiero disponer de al menos un consejo para cada nivel categórico de depresión.
            \item Como usuario, quiero disponer de al menos un consejo para cada nivel categórico de soledad.
            \item Como usuario, quiero visualizar la evolución de mis registros de estrés.
            \item Como usuario, quiero visualizar la evolución de mis registros de depresión.
            \item Como usuario, quiero visualizar la evolución de mis registros de soledad.
            \item Como usuario, quiero visualizar la media de estrés del resto de usuarios del día anterior.
            \item Como usuario, quiero visualizar la media de depresión del resto de usuarios del día anterior.
            \item Como usuario, quiero visualizar la media de soledad del resto de usuarios del día anterior.
            \item Como usuario, quiero visualizar la media de estrés del resto de usuarios de los últimos siete días naturales.
            \item Como usuario, quiero visualizar la media de depresión del resto de usuarios de los últimos siete días naturales.
            \item Como usuario, quiero visualizar la media de soledad del resto de usuarios de los últimos siete días naturales.
            \item Como usuario, quiero visualizar, dentro de la semana en curso, la evolución de la media de estrés diario del resto de usuarios.
            \item Como usuario, quiero visualizar, dentro de la semana en curso, la evolución de la media de depresión diaria del resto de usuarios.
            \item Como usuario, quiero visualizar, dentro de la semana en curso, la evolución de la media de soledad diaria del resto de usuarios.
            \item Como usuario, quiero visualizar los datos de actividad física recogidos por la aplicación.
            \item Como usuario, quiero utilizar la aplicación sin necesidad de disponer de Internet en todo momento, por consiguiente estarán disponibles todas las funciones relativas al usuario.
            \item Como analista de datos, quiero poder recibir los datos anonimazados de estrés recogidos por la aplicación móvil.
            \item Como analista de datos, quiero poder recibir los datos anonimazados de depresión recogidos por la aplicación móvil.
            \item Como analista de datos, quiero poder recibir los datos anonimazados de soledad recogidos por la aplicación móvil.
            \item Como analista de datos, quiero poder recibir los datos anonimazados de actividad física recogidos por la aplicación móvil.
        \end{enumerate}
    
    \subsection{Requisitos funcionales}
        TBD

        \subsubsection{Aplicación móvil}

        \begin{enumerate}[label=\textbf{\texttt{RF-\arabic*}}]
            \item El seguimiento del estrés se realizará mediante dos cuestionarios diarios, uno durante la mañana y otro durante la noche, y puntuales.
            \todo[inline]{Decimos aqui que los cuestionarios diarios deben diseñarlos psicólogos?}
            \item El cuestionario puntual para el seguimiento del estrés será el PSS-14.
            \item La medida estrés estará calibrada en tres umbrales: baja, moderada y alta.
            \item Para cada umbral de la medida estrés se dispondrá de al menos un mensaje o recomendación.
            \item El seguimiento de la depresión se realizará mediante dos cuestionarios diarios, uno durante la mañana y otro durante la noche, y puntuales.
            \item El seguimiento de la depresión podrá ser activado y desactivado por el usuario en todo momento.
            \todo[inline]{Decimos aqui que los cuestionarios diarios deben diseñarlos psicólogos?}
            \item El cuestionario puntual para el seguimiento de la depresión será el PHQ-9.
            \item La medida depresión estará calibrada en tres umbrales: baja, moderada y alta.
            \item Para cada umbral de la medida depresión se dispondrá de al menos un mensaje o recomendación.
            \item El seguimiento de la soledad se realizará mediante dos cuestionarios diarios, uno durante la mañana y otro durante la noche, y puntuales.
            \item El seguimiento de la soledad podrá ser activado y desactivado por el usuario en todo momento.
            \todo[inline]{Decimos aqui que los cuestionarios diarios deben diseñarlos psicólogos?}
            \item El cuestionario puntual para el seguimiento de la soledad será el UCLA-20.
            \item La medida soledad estará calibrada en tres umbrales: baja, moderada y alta.
            \item Para cada umbral de la medida soledad se dispondrá de al menos un mensaje o recomendación.

            \item La aplicación dispondrá de un cuestionario diario sobre suicidio. Dicho cuestionario será creado por la noche.
            \todo[inline]{Decimos aqui que los cuestionarios diarios deben diseñarlos psicólogos?}
            \item La medida de riesgo de suicidio estará calibrada en tres umbrales: bajo, moderado y alto.
            \item Para cada umbral de la medida riesgo de suicidio se dispondrá de al menos un mensaje o recomendación.
            \item La aplicación dispondrá de un cuestionario diario denominado "¿Cómo has estado hoy?". Dicho cuestionario será creado por la noche.
	    \todo[inline]{Decimos aqui que los cuestionarios diarios deben diseñarlos psicólogos?}

            \item Durante la realización de un cuestionario, la aplicación permitirá al usuario no completarlo en ese momento si así lo desea.
            \item La aplicación guardará el estado del cuestionario si el usuario decide no completarlo en ese momento.

            \item El usuario podrá autorizar o no el acceso de la aplicación a los datos de distancia recorrida.
            \item El usuario podrá autorizar o no el acceso de la aplicación a los datos de desnivel ganado.
            \item El usuario podrá autorizar o no el acceso de la aplicación a los datos de las sesiones de ejercicio.
            \item El usuario podrá autorizar o no el acceso de la aplicación a los datos del desnivel \todo[inline]{es elevation gained pero suena horrible, mejor traducción?}.
            \item El usuario podrá autorizar o no el acceso de la aplicación a los datos de frecuencia cardíaca.
            \item El usuario podrá autorizar o no el acceso de la aplicación a los datos de sueño.
            \item El usuario podrá autorizar o no el acceso de la aplicación a los datos de pasos realizados.
            \item El usuario podrá autorizar o no el acceso de la aplicación a los datos de calorías quemadas.
            \item El usuario podrá autorizar o no el acceso de la aplicación a los datos de peso.

            \item La aplicación recibirá del servidor la media de los cuestionarios de depresión, soledad y depresión del día anterior.
            \item La aplicación recibirá del servidor la media de los cuestionarios de depresión, soledad y depresión de los últimos siete días.
            \item La aplicación recibirá del servidor la media de los cuestionarios de depresión, soledad y depresión de cada día de la semana actual. 

            \item La aplicación móvil notificará al usuario si necesita instalar el componente \textit{Health Connect} para los dispositivos con versiones de Android inferiores a la 14.

        \end{enumerate}

        \subsubsection{Componente servidor}
    
    \subsection{Requisitos no funcionales}
        TBD
        \subsubsection{Acurracy}
            \todo[inline]{Tiene sentido hablar de precisión en esta app?}
        \subsubsection{Availability}
            \todo[inline]{Brindis al sol de la disponibilidad del servidor?}
        \subsubsection{Eficiencia (Efficiency)}
            \todo[inline]{Queda claro que es necesario pero no suficiente?}
            \begin{enumerate}[label=\textbf{\texttt{RNF-\arabic*}}]
                \item La aplicación móvil sólo intentará enviar datos al servidor cuando se disponga de conexión a Internet.
                \item Si en el momento de envío de datos el dispositivo no dispone de conexión a Internet, se aplazará la subida de los mismos para para realizarse en cuanto se disponga de conexión.
                \item En cada envío de datos la aplicación móvil sólo enviará los datos recogidos desde el último envío.
            \end{enumerate}
        \subsubsection{Extensibilidad (Extensibility)}
            \begin{enumerate}[resume, label=\textbf{\texttt{RNF-\arabic*}}]
                \item El diseño y la aplicación garantizarán, en la medida de lo posible, el soporte a nuevas versiones de Android.
                \item El diseño y la aplicación garantizarán, en la medida de lo posible, la incorporación y/o eliminación de tipos de datos de actividad física.
            \end{enumerate}
        \subsubsection{Interoperabilty}
            \todo[inline]{Hablamos aqui de las unidades de medida de los datos?}
        \subsubsection{Mantenibilidad (Maintainability)}
            \todo[inline]{Podemos añadir algo de la entrega de actualizaciones..}
            \begin{enumerate}[resume, label=\textbf{\texttt{RNF-\arabic*}}]
                \item Se desarrollará el código software siguiendo los principios S.O.L.I.D.
            \end{enumerate}
        \subsubsection{Rendimiento (Performance)}
            \begin{enumerate}[resume, label=\textbf{\texttt{RNF-\arabic*}}]
                \item Los cuestionarios diarios de mañana serán creados a las 9:00 horas.
                \item Los cuestionarios diarios de noche serán creados a las 21:00 horas.
                \item Los cuestionarios puntuales serán creados a las 15:00 horas.
                \item La aplicación subirá al servidor los datos de salud a los que tenga permiso cada 8 horas.
                \item La aplicación subirá al servidor los datos de los cuestionarios diarios terminados cada 8 horas.
                \item La aplicación subirá al servidor los datos de los cuestionarios puntuales terminados cada 8 horas.
    
            \end{enumerate}
        \subsubsection{Portability}
            \begin{enumerate}[resume, label=\textbf{\texttt{RNF-\arabic*}}]
                \item La aplicación deberá poder leer los datos de cualquier dispositivo \textit{wearable} que haya integrado el fabricante del mismo con \textit{Health Connect}.
            \end{enumerate}
        \subsubsection{Recoverability}
            \todo[inline]{Tiene sentido en esta app?}
        \subsubsection{Reusability}
            \todo[inline]{Que podemos hablar aqui? Que los datos se mantendrán entre versiones de la app al definirse una migración? Tolerancia de errores? Tiene sentido en esta app?}
        \subsubsection{Robustness}
            \todo[inline]{Tolerancia de errores? Tiene sentido en esta app?}
        \subsubsection{Safety}
            \begin{enumerate}[resume, label=\textbf{\texttt{RNF-\arabic*}}]
                \item El ID de usuario será generado aleatoriamente por la aplicación y será distinto en cada instalación.
                \item La base de datos de la aplicación estará cifrada con el algoritmo AES, usando el modo de cifrado GCM.
            \end{enumerate}
        \subsubsection{Scalability}
            \begin{enumerate}[resume, label=\textbf{\texttt{RNF-\arabic*}}]
                \item Hola
                \item Mundo
            \end{enumerate}
        \subsubsection{Security / Integrity}
            \begin{enumerate}[resume, label=\textbf{\texttt{RNF-\arabic*}}]
                \item Hola
                \item Mundo
            \end{enumerate}
        \subsubsection{Usability}
            \begin{enumerate}[resume, label=\textbf{\texttt{RNF-\arabic*}}]
                \item Hola
                \item Mundo
            \end{enumerate}

\section{Requisitos de Interfaces Externas}

    \subsection{Interfaces de usuario}
        \begin{enumerate}[label=\textbf{\texttt{RIU-\arabic*}}]
            \item La interfaz de usuario deberá adaptarse al tamaño de la pantalla del dispositivo móvil.
            \item El diseño de la interfaz de usuario contemplará dos ajustes de colores principales: modo claro y oscuro.
            \item La aplicación móvil tomará por defecto el modo de colores general del sistema.
            \item El usuario podrá cambiar el modo de colores de la aplicación en todo momento. 
            \item La aplicación móvil ofrecerá la posibilidad, en los dispositivos móviles que lo soporten, la posibilidad de usar un esquema de colores basado en el fondo de pantalla del dispositivo.
            \todo[inline]{Estos requisitos pertenecen a otra categoría?}
            \item La aplicación móvil estará disponible en el idioma español, siendo el idioma por defecto.
            \item La aplicación móvil estará disponible en el idioma inglés.
            \item El usuario podrá cambiar el modo de colores de la aplicación en todo momento.
        \end{enumerate}

    \subsection{Interfaces Hardware}
        \todo[inline]{Aqui en todo caso hablariamos de que usamos wifi, datos móviles y bluetooth? De verdad hablamos de bluetooth si lo usa health connect y no nosotros?}
    
    \subsection{Interfaces Software}
        \begin{enumerate}[label=\textbf{\texttt{RIS-\arabic*}}]
            \item La comunicación entre aplicación móvil y servidor se implementará mediante una API REST.
            \item La base de datos en el componente servidor será implementada mediante MongoDB.
            \item La base de datos en la aplicación móvil se implementará mediante la librería Room.
            \item Cuando un cuestionario sea creado, la aplicación móvil deberá crear una notificación al usuario.
            \todo[inline]{No me cuadra la redacción del siguiente requisito}
            \item Si existe al menos un cuestionario no completado, deberá notificarse al usuario mediante un aviso dentro de la aplicación.
    
        \end{enumerate}
    
    \subsection{Interfaces de comunicaciones}
        \todo[inline]{Pondríamos aqui los tipos de mensajes que enviamos y recibimos? Lo que tenemos en la api vaya}
    
    \todo[inline]{Restricciones de entorno fisico no aplica verdad?}

\section{Buffer}
\begin{enumerate}
    \item El usuario podrá permitir o no las notificaciones (impuesto por Android)
    \item La aplicación dispondrá de un canal general de notificaciones
    \item La aplicación dispondrá de un canal exclusivo de notificaciones de cuestionarios.

    \item La aplicación dispondrá de una ventana de créditos.
    
    \item Los datos de los cuestionarios serán guardados en una base de datos relacional.
    \item Los ajustes de preferencias serán guardados en memoria persistente.
	
    \item El usuario podrá dar permiso de lectura a cada dato de salud independientemente.
\end{enumerate}
%\section {Análisis de Stakeholders}
