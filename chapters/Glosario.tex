\newglossaryentry{wearable}{
    name={\textit{Wearable}:},
    text={\textit{wearable}},
    plural={\textit{wearables}},
    description={Tipo de dispositivo electrónico, el cual se caracteriza por estar colocado o \textit{vestido} en alguna parte del cuerpo humano; siendo los más conocidos los relojes inteligentes y pulseras de actividad. El lector puede encontrar en la Sección \ref{section:contexto:wearables} una introducción más detallada a estos dispositivos y su uso en el contexto de este proyecto}
}

\newglossaryentry{framework}{
    name={\textit{Framework}:},
    text={\textit{framework}},
    description={Extranjerismo del inglés, que puede traducirse como \textit{marco de trabajo}, utilizado para definir un componente software que actúa como base para desarrollar nuevo software a partir de él, proporcionando herramientas y componentes reutilizables}
}

\newglossaryentry{microframework}{
    name={\textit{Microframework}:},
    text={\textit{microframework}},
    description={Término derivado de \gls{framework}, utilizándose para denominar a aquellos \gls{framework} minimalistas, que normalmente se preocupan de resolver una única necesidad; lo que les permite destacar por su ligereza}
}

\newglossaryentry{sdk}{
    name={SDK},
    description={Siglas en inglés de \textit{Software Development Kit} o \textit{conjunto de herramientas software} en español. Como su nombre indica, se trata de un kit de componentes software, utilizados para interactuar con un dispositivo electrónico. Normalmente estas herramientas son proporcionadas por el propio fabricante del dispositivo, y se suele utilizar para el desarrollo de soluciones software que o bien sean compatibles con dicho dispositivo o interactúen con el de alguna manera.}
}

\newglossaryentry{responsive}{
    name={\textit{Responsive}:},
    text={\textit{responsive}},
    description={Extranjerismo del inglés, que puede traducirse como \textit{adaptable}, que en un contexto informático hace referencia a la capacidad de una interfaz de usuario para ser accesible y usable en cualquier dispositivo. En ese proyecto específicamente, se puede entender como la adaptabilidad de la interfaz gráfica para los diferentes tamaños de las pantallas de los dispositivos compatibles, y sus posibles orientaciones (horizontal o vertical)}
}

\newglossaryentry{widget}{
    name={\textit{Widget}:},
    text={\textit{widget}},
    plural={\textit{widgets}},
    description={En el contexto de Android, se puede entender como elementos de la pantalla de inicio del dispositivo configurables por el usuario que permiten ver información relativa a una aplicación desde la propia ventana de inicio, sin necesidad de acceder a la aplicación}
}

\newglossaryentry{vfc}{
    name={Variabilidad de la frecuencia cardíaca:},
    text={variabilidad de la frecuencia cardíaca},
    description={Simplificadamente, se puede entender como la \textit{``variación de la frecuencia de los latidos del corazón durante un intervalo de tiempo determinado''} \cite{apta_vital_sport_que_2023}, Actualmente se puede medir mediante pulsómetros, los cuales se pueden encontrar en los \glspl{wearable} más conocidos}
}

\newglossaryentry{eda}{
    name={Actividad electrodérmica:},
    text={actividad electrodérmica},
    description={Se puede definir de forma muy simplificada como \textit{``los cambios en la conductancia eléctrica de la piel, debidos a diferentes estímulos y señales''} \cite{universidad_miguel_hernandez_de_elche_actividad_2023}}
}

\newglossaryentry{eeg}{
    name={Electroencefalograma:},
    text={electroencefalograma},
    plural={electroencefalogramas},
    description={Prueba que \textit{``evalúa la actividad eléctrica de las células nerviosas (neuronas) que están situadas en la corteza cerebral (tejido nervioso que recubre el cerebro). Es una prueba neurofisiológica que evalúa y mide la función de las neuronas desde el punto de vista eléctrico, mediante electrodos que se colocan en el cuero cabelludo de una persona''} \cite{iranzo_de_riquer_electroencefalograma_2022}}
}

\newglossaryentry{ecg}{
    name={Electrocardiograma:},
    text={electrocardiograma},
    plural={electrocardiogramas},
    description={Prueba que \textit{``registra la actividad eléctrica del corazón que se produce en cada latido cardiaco. Esta actividad eléctrica se registra desde la superficie corporal del paciente y se dibuja en un papel mediante una representación gráfica o trazado, donde se observan diferentes ondas que representan los estímulos eléctricos de las aurículas y los ventrículos''} \cite{rodriguez_manero_electrocardiograma_nodate}}
}

\newglossaryentry{bvp}{
    name={Volumen de pulso sanguíneo:},
    text={volumen de pulso sanguíneo},
    description={Este concepto hace referencia a \textit{``los cambios relativos de volumen sanguíneo en la venas del dedo índice. Esta medición indica la cantidad de sangre que circula actualmente a través de las venas, lo cual permite calcular la vasoconstricción, la dilatación vascular, la frecuencia cardíaca y la hipovolemia. Si los niveles de volumen de pulso sanguíneo o de vasoconstricción son altos se puede estar en estados de furia o estrés, si los valores se reducen, se puede estar en estado de relajación o tristeza''} \cite{vazquez_pfc_2012}}
}

\newglossaryentry{dataset}{
    name={\textit{Dataset}:},
    text={\textit{dataset}},
    plural={\textit{datasets}},
    description={Extranjerismo del inglés, que puede traducirse como \textit{conjunto organizado de datos}; los cuales son utilizados para realizar análisis de datos o entrenar modelos de aprendizaje automático, entre otras aplicaciones. Dichos conjuntos de datos no tienen por qué estar limitados en cuanto a su naturaleza, pudiendo ser por ejemplo registros de datos, fotografías o archivos de audio}
}

\newglossaryentry{invasiva}{
    name={Medición invasiva:},
    text={medición invasiva},
    plural={mediciones invasivas},
    description={Simplificadamente, se puede considerar como medición invasiva a aquella medición realizada con un sensor insertado dentro del cuerpo o del medio de estudio. Este tipo de sensorización puede proporcionar datos más precisos y fiables, si bien pueden provocar riesgo de infecciones y/o incomodidad o dolor para el paciente. En un contexto médico, dicha penetración puede realizarse mediante la inserción de agujas, sondas, catéteres u otros dispositivos}
}

\newglossaryentry{no-invasiva}{
    name={Medición no invasiva:},
    text={medición no invasiva},
    plural={mediciones no invasivas},
    description={Simplificadamente, se puede considerar como medición no invasiva a aquella medición realizada con un sensor no insertado dentro del cuerpo o del medio de estudio, sin perturbar al cuerpo u objeto. Si bien la fiabilidad y precisión inferior suele ser inferior respecto a un procedimiento invasivo, resulta más fácil de implementar, sin riegos de daño o incomodidad para el usuario. En un contexto informático, se pueden ejemplificar estas mediciones con los dispositivos \glspl{wearable}}
}

\newglossaryentry{smartphone}{
    name={\textit{Smartphone}: },
    text={\textit{smartphone}},
    plural={\textit{smartphones}},
    description={Según el \textit{diccionario panhispánico del español jurídico}, se trata de un \textit{``terminal móvil que ofrece servicios avanzados de comunicaciones (acceso a internet y correo electrónico), así como servicios de agenda y organizador personal con un mayor grado de conectividad que un terminal móvil convencional} \cite{rae_definicion_nodate} }
}

\newglossaryentry{boilerplate}{
    name={\textit{Boilerplate}:},
    text={\textit{boilerplate}},
    description={Término inglés que se utiliza para describir al código que se mantiene igual en todos los lugares donde se utiliza}
}

\newglossaryentry{open-source}{
    name={\textit{Open source}:},
    text={\textit{open source}},
    description={Término inglés que se puede traducir como \textit{código abierto}, el cual hace referencia a un modelo de desarrollo basado en la colaboración abierta, haciendo especial hincapié en el acceso al código fuente y la libertad de modificación del mismo, sin implicar necesariamente la gratuidad del mismo}
}

\newglossaryentry{comorbilidad}{
    name={Comorbilidad:},
    text={comorbilidad},
    description={Presencia de dos o más enfermedades al mismo tiempo en una persona}
}

\newacronym{tfm}{TFM}{Trabajo Fin de Máster}
\newacronym{ers}{ERS}{Especificación de Requisitos Software}
\newacronym{ieee}{IEEE}{\textit{Institute of Electrical and Electronics Engineers}}
\newacronym{wcag}{WCAG}{\textit{Web Content Accessibility Guidelines}}
\newacronym{w3c}{W3C}{\textit{World Wide Web Consortium}}
\newacronym{sysml}{SysML}{\textit{System Modeling Language}}
\newacronym{sgbd}{SGBD}{Sistema Gestor de Base de Datos}
\newacronym{aes}{AES}{\textit{Advanced Encryption Standard}}
\newacronym{gcm}{GCM}{\textit{Galois/Counter Mode}}
\newacronym{rgpd}{RGPD}{Reglamento General de Protección de Datos}
\newacronym{oms}{OMS}{Organización Mundial de la Salud}
\newacronym{ods}{ODS}{Objetivos de Desarrollo Sostenible}
\newacronym{aosp}{AOSP}{\textit{Android Open Source Project}}
\newacronym{gms}{GMS}{\textit{Google Mobile Services}}
\newacronym{cnmc}{CNMC}{Comisión Nacional de los Mercados y la Competencia}
\newacronym{jvm}{JVM}{\textit{Java Virtual Machine}}
\newacronym{api}{API}{\textit{Application Programming Interface}}
\newacronym{dao}{DAO}{\textit{Data Access Object}}
\newacronym{json}{JSON}{\textit{JavaScript Object Notation}}
\newacronym{bson}{BSON}{\textit{Binary \gls{json}}}
\newacronym{sql}{SQL}{\textit{Structured Query Language}}
\newacronym{http}{HTTP}{\textit{Hypertext Transfer Protocol}}
\newacronym{https}{HTTPS}{\textit{Hypertext Transfer Protocol Secure}}
\newacronym{etsisi}{ETSISI}{Escuela Técnica Superior de Ingeniería de Sistemas Informáticos}
\newacronym{xml}{XML}{\textit{eXtensible Markup Language}}
\newacronym{gif}{GIF}{\textit{Graphics Interchange Format}}
\newacronym{svg}{SVG}{\textit{Scalable Vector Graphics}}
\newacronym{agpl}{AGPL}{\textit{GNU Affero General Public License}}
\newacronym{apa}{APA}{\textit{American Psychiatric Association}}
\newacronym{tdah}{TDAH}{Trastorno por Déficit de Atención e Hiperactividad}
\newacronym{pss}{PSS}{\textit{Perceived Stress Scale}}
\newacronym{phq}{PHQ}{\textit{Patient Health Questionnaire}}
\newacronym{ucla}{UCLA}{\textit{University of California, Los Angeles}}
\newacronym{iss}{ISS}{Ingeniería del Software de Sistemas}
\newacronym{iea}{IEA}{Agencia Internacional de la Energía}
\newacronym{tic}{TIC}{Tecnologías de la Información y las Comunicaciones}
\newacronym{ssl}{SSL}{\textit{Secure Sockets Layer}}