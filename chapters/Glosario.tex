\newglossaryentry{wearable}{
    name={\textit{Wearable}:},
    text={\textit{wearable}},
    plural={\textit{wearables}},
    description={Extranjerismo del inglés utilizado para definir aquellos dispositivos electrónicos que son colocados en alguna parte del cuerpo humano. Los dispositivos más conocidos son los relojes inteligentes y pulseras de actividad}
}

\newglossaryentry{framework}{
    name={\textit{Framework}:},
    text={\textit{framework}},
    description={Extranjerismo del inglés, que puede traducirse como \textit{marco de trabajo}, utilizado para definir un componente software que actúa como base para desarrollar nuevo software a partir de él, proporcionando herramientas y componentes reutilizables}
}

\newglossaryentry{sdk}{
    name={SDK},
    description={Siglas de \textit{Software Development Kit}: conjunto de herramientas software,
    normalmente proporcionado por un fabricante para la creación de componentes compatibles con
    su producto}
}

\newglossaryentry{responsive}{
    name={\textit{Responsive}:},
    text={\textit{responsive}},
    description={Extranjerismo del inglés, que puede traducirse como \textit{adaptable}, referido a la capacidad de una interfaz de usuario para ser accesible y usable en cualquier dispositivo. En el contexto de ese proyecto, se refiere a la adaptabilidad para los diferentes tamaño de pantallas y orientaciones de las mismas}
}


\newacronym{tfm}{TFM}{Trabajo Fin de Máster}
\newacronym{ers}{ERS}{Especificación de Requisitos Software}
\newacronym{ieee}{IEEE}{\textit{Institute of Electrical and Electronics Engineers}}
\newacronym{wcag}{WCAG}{\textit{Web Content Accessibility Guidelines}}
\newacronym{w3c}{W3C}{\textit{World Wide Web Consortium}}
\newacronym{sysml}{SysML}{\textit{System Modeling Language}}
\newacronym{sgbd}{SGBD}{Sistema Gestor de Base de Datos}
\newacronym{aes}{AES}{\textit{Advanced Encryption Standard}}
\newacronym{gcm}{GCM}{\textit{Galois/Counter Mode}}