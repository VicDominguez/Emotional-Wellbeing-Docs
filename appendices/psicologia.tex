\chapter{Cuestionarios para el seguimiento diario}
\label{chapter:cuestionarios}
    \section{Inicio del día}
        \subsection{Estrés}
            \begin{enumerate}
                \item Me siento nervioso/a
                \item Me siento angustiado/a
                \item Me siento activo/a
                \item Estoy preocupado/a
            \end{enumerate}
            La respuesta a cada pregunta es un número entero en la escala de 0 a 10.

        \subsection{Depresión}
            \begin{enumerate}
                \item Me siento triste
                \item Me siento vacío/a 
                \item Me siento apático/a
            \end{enumerate}
            La respuesta a cada pregunta es un número entero en la escala de 0 a 10.

        \subsection{Soledad}
            \begin{enumerate}
                \item Me siento solo/a
                \item Me siento incomprendido/a
                \item Me siento exclusivo/a
                \item Me siento poco ayudado/a
            \end{enumerate}
            La respuesta a cada pregunta es un número entero en la escala de 0 a 10.

        \subsection{Suicidio}
            \begin{enumerate}
                \item Tengo pensamientos de suicidio
                \item En los últimos días, ¿has pensado seriamente en suicidarte?
                \item ¿Existe alguna posibilidad de que pienses acabar con tu vida hoy o en los próximos días?
            \end{enumerate}
            Las posibles respuestas a cada pregunta es sí o no.

    \section{Final del día}
        \subsection{Estrés}
            \begin{enumerate}
                \item Me he sentido nervioso/a
                \item Me he sentido angustiado/a
                \item Me he sentido activo/a
                \item He estado preocupado/a
            \end{enumerate}
            La respuesta a cada pregunta es un número entero en la escala de 0 a 10.

        \subsection{Depresión}
            \begin{enumerate}
                \item Me he sentido triste
                \item Me he sentido vacío/a 
                \item Me he sentido apático/a
            \end{enumerate}
            La respuesta a cada pregunta es un número entero en la escala de 0 a 10.

        \subsection{Soledad}
            \begin{enumerate}
                \item Me he sentido solo/a
                \item Me he sentido incomprendido/a
                \item Me he sentido exclusivo/a
                \item Me he sentido poco ayudado/a
            \end{enumerate}
            La respuesta a cada pregunta es un número entero en la escala de 0 a 10.

        \subsection{Suicidio}
            \begin{enumerate}
                \item He tenido pensamientos de suicidio
                \item En el día de hoy, ¿has pensado seriamente en suicidarte?
                \item ¿Existe alguna posibilidad de que pienses acabar con tu vida hoy o en los próximos días?
            \end{enumerate}
            Las posibles respuestas a cada pregunta es sí o no.

        \subsection{Contraste}
            \begin{enumerate}
                \item ¿Has experimentado cambios en el apetito?
                
                Posibles respuestas:
                    \begin{itemize}
                        \item Excesivamente alto  
                        \item Adecuado
                        \item Excesivamente bajo
                    \end{itemize}

                \item ¿Con cuánta energía te has notado?
                
                Posibles respuestas:
                    \begin{itemize}
                        \item Alta
                        \item Moderada
                        \item Baja
                    \end{itemize}

                \item ¿Cuál ha sido tu nivel de descanso?
                
                Posibles respuestas:
                    \begin{itemize}
                        \item Satisfactorio  
                        \item Moderado
                        \item Insuficiente
                    \end{itemize}
                    
                \item Tu nivel de concentración ha sido…
                
                Posibles respuestas:
                    \begin{itemize}
                        \item Satisfactorio  
                        \item Adecuado
                        \item Insuficiente
                    \end{itemize}
                    
                \item ¿Cuál ha sido tu nivel de líbido?
                
                Posibles respuestas:
                    \begin{itemize}
                        \item Satisfactorio  
                        \item Adecuado
                        \item Insuficiente
                    \end{itemize}
                    
                \item ¿Cómo te has encontrado a nivel de dolor?
                
                Posibles respuestas:
                    \begin{itemize}
                        \item Sin dolor 
                        \item Dolor moderado
                        \item Dolor alto
                    \end{itemize}
                    
            \end{enumerate}
\chapter{Recomendaciones}
\label{chapter:recomendaciones}

    \section{Estrés}
        \subsection{Bajo}
            Estupendo, sigue así. 
        \subsection{Moderado}
            \subsubsection{Pauta 1}
                Hemos percibido que estás experimentando niveles moderados de ansiedad o estrés. 
                Por ello, te recomendamos que planifiques un espacio en el día de hoy para hacer ejercicio físico.

                El ejercicio físico puede disminuir el estrés por varias razones:
                \begin{itemize}
                    \item Liberación de endorfinas: Durante el ejercicio, el cuerpo libera endorfinas, que son hormonas que  actúan como analgésicos naturales y generan sensaciones de bienestar. 
                    \item Reducción de la hormona del estrés: El ejercicio regular puede disminuir los niveles de cortisol, la hormona del estrés.
                    \item Mejora del sueño: El ejercicio regular puede promover un sueño más profundo y reparador.
                    \item Distracción y enfoque: Participar en actividades físicas puede distraer la mente de las preocupaciones y tensiones diarias. Cuando te concentras en el ejercicio, tu mente se enfoca en la actividad física en lugar de en los problemas, lo que puede ayudar a reducir el estrés y proporcionar un descanso mental.
                    \item Aumento de la confianza y la autoestima: El ejercicio regular puede ayudar a mejorar la confianza y la autoestima. Al establecer metas y lograr objetivos en el ámbito del ejercicio, puedes desarrollar una mayor sensación de logro y fortaleza personal. Esto puede ayudar a reducir el estrés al 
                    proporcionar una sensación de control y empoderamiento sobre tu vida.
                \end{itemize}

            \subsubsection{Pauta 2}
                En caso de que no dispongas de mucho tiempo, te proponemos una serie de alternativas. 

                \begin{itemize}
                    \item Trata de buscar un momento para ti, libre de estímulos estresantes. Puedes salir a dar un pequeño paseo, darte una ducha relajante, poner música y centrarte en escucharla durante unos minutos, practica unos estiramientos corporales… El objetivo es rebajar de forma rápida los niveles de ansiedad para poder retomar las tareas desde un estado emocional más adecuado. 
                    \item Trata de eliminar algunos estímulos que puedan estar aumentando tu ansiedad: apaga el móvil a partir de determinada hora en la noche para tener unas horas libres de notificaciones antes de dormir, ponte unos cascos con música relajante para no escuchar el ruido de alrededor, si tienes pendiente tomar una decisión o discutir algo con alguien, aplázalo durante unas horas o días y permítete posponer los pensamientos al respecto, etc. El objetivo es eliminar los estímulos que están produciendo estrés para poder rebajar los niveles de ansiedad y así enfrentarnos de forma más adecuada a nuestros problemas o dificultades. 
                \end{itemize}
        \subsection{Alto}
            Te proponemos que realices un ejercicio de respiración abdominal. 

            El objetivo de esta técnica es regular la respiración y, en consecuencia, disminuir la respuesta de activación fisiológica y la sensación de ansiedad. 
            
            Para ello, trata de llevar el aire hasta tu abdomen en cada inspiración para llenar tus pulmones en profundidad. Visualmente, deberías observar cómo tu tripa se hincha al llenarse de aire. Al expulsar el aire durante la espiración, el abdomen debería retornar a su posición habitual. Evita mover el pecho, los hombros o las clavículas, pues esto indica que el aire está llegando únicamente a la parte superior de los pulmones.
            
            Inhala durante la ascensión de la curva y exhala durante el descenso. Trata de no hacerlo de forma demasiado profunda. Puedes repetirte mentalmente una palabra como calma o relax, puedes imaginar que estás en un lugar tranquilo, o centrar tu atención en cómo el aire entra y sale y cómo la tensión se escapa con cada exhalación. 
    
    \section{Depresión}
        \subsection{Baja}
            Estupendo, sigue así. 
        \subsection{Moderada}

            Te proponemos que incluyas en tu  día de hoy alguna actividad agradable o placentera. 
            
            Es posible que sientas que no tienes ganas o energía para hacerlas o, incluso, que aunque las hagas no lo disfrutarás. No obstante, es importante que entiendas que “las ganas se hacen”. 
            
            Esto significa que cuando nuestro estado de ánimo está un poquito bajo, si esperamos a experimentar ganas para hacer las cosas, probablemente nunca las hagamos. Esto a su vez hará que nuestro estado de ánimo disminuya todavía más, y entremos en un circulo vicioso en el que no haremos nada porque no tenemos ganas porque estamos tristes, y como no hacemos nada estaremos aún más tristes. 
            
            Para no caer en esta problemática te sugerimos que realices alguna actividad agradable, que no sea muy costosa y que te permita sentirte mejor. Pueden ser actividades que hagas tú solo/a o acompañado/a. 
            
            Aquí te dejamos algunas sugerencias. 
            \begin{itemize}
                \item Dar un paseo al aire libre. 
                \item Ir a comprar al supermercado y cocinar una receta que te guste. 
                \item Ver una serie o película que te apetezca.
                \item Quedar con un amigo/a tomar algo o pasear.
                \item Leer un libro, escuchar música, dibujar… 
            \end{itemize}

        \subsection{Alta}
            Te sugerimos que busques apoyo en las personas de tu alrededor.

            Es muy beneficioso que puedas expresar cómo te estás sintiendo a otras personas. El mero hecho de contarlo supondrá un desahogo emocional que te hará sentirte mejor.
            
            Además, las personas que te escuchen podrán comprender por lo que estás pasando y mostrar su empatía y apoyo. Es posible también que puedan tratar de ayudarte u ofrecerte consejos.
            
            Para que todo vaya bien, trata de elegir a la persona adecuada en el momento adecuado. Busca a una persona que se encuentre bien, que no esté muy estresada u ocupada, que te haya mostrado su afecto en alguna ocasión…
            
            Si necesitas desahogarte con frecuencia, trata de hacerlo con diferentes personas y no focalizarte solo en una, ya que prestar apoyo emocional en ocasiones puede resultar algo cansado.
           
            Por último, ten cuidado de no caer en la queja: comunica tus emociones tratando de buscar soluciones y formas de sentirte mejor, en lugar de anclarte en el problema que ha sucedido.

    \section{Soledad}
        \subsection{Baja}
            Estupendo, sigue así. 
        \subsection{Moderada}
            Te sugerimos que busques apoyo en las personas de tu alrededor.

            Es posible que consideres que no tienes a nadie con quién hablar. Sin embargo, si lo intentas, seguro que puedes encontrar personas deseosas de conversar contigo.
            
            Busca a tu alrededor: un vecino con quien hayas tenido contacto, el camarero del bar o el restaurante al que vas en ocasiones, un compañero de trabajo…Puedes tratar de entablar una conversación preguntándoles cómo se encuentran y comentando cosas sobre el ambiente (el tiempo, los precios, los horarios, etc.).
            
            Después, puedes intentar contar alguna anécdota o experiencia personal reciente. Por ejemplo, un programa de TV que te haya gustado, algo curioso que hayas visto recientemente en redes sociales, un plan al que tengas ganas de asistir o que hayas disfrutado si ya lo has hecho, etc.
           
            Poco a poco, podrás progresar en la conversación y hablar más a menudo con estas personas.

        \subsection{Alta}
            Te recomendamos que llames a Cruz Roja Te Escucha (900 107 917).

            Cruz Roja te escucha es una iniciativa que trata de ofrecer acompañamiento y apoyo a personas que se encuentran en una situación de soledad no deseada. 
            
            Podrán facilitarte pautas para sentirte mejor y ofrecerte recursos en tu localidad que puedan servirte de apoyo. 
            
            Puedes contactar en el 900 107 917, de lunes a jueves laborales de 10h a 14h y de 16h a 20h (una hora menos en Canarias) y los viernes laborables de 10 a 14h (una hora menos en Canarias). La llamada es gratuita y confidencial.
    
    \section{Riesgo de suicidio}
        \subsection{Bajo}
            Estupendo, sigue así. 
        \subsection{Moderado}
            Recuerda que, si en algún momento tienes pensamientos relacionados con el suicidio, es importante que pidas ayuda.

        \subsection{Alto}
            Por lo que nos has contado, creemos que el riesgo de que puedas hacerte daño o de que acabes con tu vida es alto.

            Por favor, acude cuanto antes a un servicio de emergencias o llama a los teléfonos 112 o 024. 
            
            Allí encontrarás personas que podrán comprender cómo te sientes y ayudarte a sentirte mejor. 
            
            Recuerda que el suicidio es la única opción que no tiene vuelta atrás. Trata de agotar otras posibles soluciones y pide ayuda para conseguirlo.