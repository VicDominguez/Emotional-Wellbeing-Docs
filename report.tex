\documentclass[%
    school=etsisi,%
    type=pfm,%
    degree=61MSSDE,%
    authorsex=m,%
    directorsex=f,%
]{upm-report}

\author{Víctor Manuel Domínguez Rivas}
\title{Sistema para el Bienestar Emocional}
\director{Sandra Gómez Canaval y Gema Bello Orgaz}


\pagestyle{empty}

\abstract{spanish}{
    
%Cuatro o cinco; Expresiones clave; Que resuman; Nuestro proyecto o; Investigación

En 2022 4.227 personas se suicidaron en España, suponiendo la primera causa de muerte no natural en nuestro país; mientras a nivel global esta cifra ascendió a casi 800.000 personas. Asimismo, se estima que el 9\% de la población española tiene actualmente algún problema de salud mental, mientras que el 25\% lo sufrirá a lo largo de su vida. Es un problema especialmente relevante entre los jóvenes, ya que el 48,9\% considera que ha tenido algún problema de salud mental. 

Numerosas investigaciones han establecido correlaciones entre ciertos trastornos de salud mental, como el estrés, la depresión o la soledad; con variables como la variabilidad de la frecuencia cardíaca, los hábitos de sueño o la actividad física. En paralelo, los avances en hardware han permitido que estas variables se puedan medir mediante dispositivos \textit{wearables} de forma no invasiva.

En este Trabajo Fin de Máster se pretende diseñar e implementar una aplicación para dispositivos Android que permita detectar estrés, depresión, soledad y riesgo de suicidio; mediante la realización de cuestionarios y de la extracción de datos de dispositivos \textit{wearables}. Según el estado del usuario la aplicación ofrecerá consejos para mejorar su bienestar emocional, junto a estadísticas anonimizadas del bienestar emocional en la comunidad universitaria; las cuales serán calculadas mediante un componente servidor.

Asimismo, este sistema será planteado para ser utilizado en el marco de futuras investigaciones o extendido por otros estudiantes en sus proyectos fin de estudios; prestando especial atención por cuestiones como la accesibilidad, el diseño adaptativo o la internacionalización, entre otras.   
}

\keywords{spanish}{Salud Mental, Bienestar Emocional, Android, \textit{Wearables}, \textit{Health Connect}}

\abstract{english}{

In 2022, 4,227 people committed suicide in Spain, making it the leading cause of unnatural death in the country, while 800,000 people commited suicide globally. Moreover, it is estimated that 9 percent of the Spanish population currently has some kind of mental health issues, while 25 percent will suffer from it during their lifetime. Also, this topic is a particularly relevant one among young people, as 48.9\% of them consider that they have had some kind of mental health issue. 

Numerous studies have established correlations between certain mental health disorders, such as stress, depression or loneliness; with variables such as heart rate variability, sleep habits or physical activity. At the same time, advances in hardware have achieved the possibility of measuring these variables non-invasively, using wearable devices.

This Master's Thesis aims to design and implement an application for Android devices to detect stress, depression, loneliness and suicide risk through questionnaires and extracting data from wearable devices. Depending on the user's state, the application will offer advices to improve their emotional wellbeing, along with anonymised statistics on the emotional wellbeing status in the university community, which will be computed by a server component.

Moreover, this system will be designed to be used as part of future research, or extended by other students in their final projects; paying special attention to key topics, such as accessibility, adaptive design or internationalisation, among others.   

}
\keywords{english}{Mental Health, Emotional Wellbeing, Android, Wearables, Health Connect}

%Agradecimientos
\acknowledgements{
Hay momentos, épocas y aventuras que no tienen una conclusión clara, ya que solo se pueden expresar con una profunda escala de grises. Para mí este proyecto ha traído momentos muy positivos y a la vez me ha expuesto a momentos que, sin ser necesariamente negativos, han cuestionado cada uno de los cimientos de mi vida. 

No puedo calificar este proyecto como un éxito o como un fracaso, ya que depende del punto de vista con el que se analice. Ha supuesto una historia de aprendizaje brutal, tanto a nivel profesional como personal, pero también lugar para historias de profunda frustración, desasosiego y fracaso. Quizá una conclusión es la importancia de disfrutar el camino, independientemente del destino. 

No obstante, hay lugar para el agradecimiento. Este proyecto, después de todo, ha marcado el final de un camino de ocho años en la escuela. Y a pesar de sus duras y crudas lecciones en lo personal, es un viaje que me ha cambiado profundamente como persona. Quizás haya sido el lugar donde he podido encontrarme como persona, de darme cuenta de mis prioridades o del rumbo a seguir en mi vida. Eso es mucho más importante que cualquier nota o reconocimiento.

Esta aventura no habría tenido ningún tipo de sentido sin mis amigos, por su apoyo en algunos de los momentos más delicados y complicados de mi vida, y sobre todo, por ayudarme a entender que la vida no tiene razón de ser sin las personas que te aportan, te quieren y te cuidan. 

Aunque afortunadamente bastantes más personas han tenido su contribución y la conocen perfectamente, Juan Luis y Rocío se merecen un párrafo especial. Sin vosotros no habría comenzado este máster, y quién sabe qué no habría hecho tanto antes como después. Gracias a vosotros dos, pude entender lo que es el respeto mutuo, la lealtad, la resiliencia, el apoyo, o el cariño. Complementarse gracias a nuestras diferencias, dar para recibir. Encontrar siempre un hogar cálido en mitad de un interminable, áspero y hostil desierto. Por apoyarnos siempre cuando alguno hemos vivido etapas y momentos crueles o descorazonadores. El legado que me dejáis transciende mucho más de lo que el papel pueda aguantar, pero por resumirlo, por haber sido la familia que se elige.

Por otra parte, gracias a mi familia ``de verdad'' por haber estado ahí desde la distancia, a pesar de vivir unos años extremadamente difíciles y convulsos. En los momentos donde todo cambia, en los que asumir las caras más amargas de la vida; es donde aparece la calidad humana de las personas. 

Esto tampoco sería posible sin mis tutoras. A Sandra y a Gema por su apoyo, tanto profesional como especialmente personal. A pesar de ser una época dura para mí han intentando sacar lo mejor de cada momento. Ayudándome a decidir sobre tantas cosas, aportando consejos, entendiendo situaciones delicadas... trayendo ilusión en el día a día. El mejor apoyo que se podía pedir.

En este momento no me quiero olvidar de Sandra Nieto, por la enorme labor que ha realizado \textit{en las sombras} para encontrarme a mi mismo, y por darme innumerables lecciones sobre la vida. También quiero agradecer el trabajo de otras personas, como a Cristina y a Miriam por su increíble labor en los consejos y cuestionarios; o a LevelUp por darle alma a este proyecto. 

Por otra parte, este proyecto no existiría sin los profesionales de la Psicología. Su trabajo ante una emergencia sanitaria invisibilizada es vital para nuestra sociedad. Historias como las que llegan desde los lugares de prevención del suicidio necesitan respuestas, ya que nos va la vida en ello.

Por último, quiero resaltar el trabajo de personas como Aaron Swartz y Alexandra Elbakyan, e iniciativas como la Open Access por facilitar el acceso a la ciencia \cite{salamanca_geopolitica_2023}. El acceso libre y gratuito al conocimiento científico es vital para el desarrollo de cualquier sociedad democrática. Una sociedad que lo promociona es una más formada y menos manipulable; algo vital en nuestros tiempos.
\begin{center}
    \vspace{-10pt}
    \textit{
    Caminante, son tus huellas \\
    el camino y nada más; \\
    Caminante, no hay camino, \\
    se hace camino al andar. \\
    Al andar se hace el camino, \\
    y al volver la vista atrás \\
    se ve la senda que nunca \\
    se ha de volver a pisar. \\
    Caminante no hay camino \\
    sino estelas en la mar.\\
    Antonio Machado, Campos de Castilla (1912).}
\end{center}
}

\begin{document}

\frontmatter


%Para quitar el header fuera en el resumen, abstract y agradecimientos
\pagestyle{schoolheader} 

\tableofcontents
\clearpage
\listoffigures
\clearpage
\listoftables
%\lstlistoflistings

\mainmatter

\input{chapters/Introduccion}
\chapter{Marco teórico y contexto tecnológico}
\label{chapter:marco}

\chapquote{La ciencia amigo, está compuesta de errores; pero son errores que es útil cometer ya que nos acercan poco a poco hacia la verdad.}{Julio Verne}

\section{Marco Teórico}


%En este capítulo se introducen los conceptos teóricos sobre los que se asienta el desarrollo de este proyecto. Con el contenido de este capítulo se espera
%crear un base de conocimiento sobre la que desarrollar el resto de este documento. Además, se realiza un análisis de las herramientas que se van a emplear para desarrollar el sistema y cómo estas se encuadran en el contexto tecnológico actual.
%
%\section{Sistemas IoT}
%
%Tal y como se introdujo de manera informal en el capítulo anterior, \textit{IoT} se define como un sistema global e inteligente
%sistema con conciencia global, transmisión fiable
%y procesamiento inteligente de datos %\cite{noauthor_8_2020}.
%
%En este sentido una arquitectura de alto nivel, basada en el conjunto de capas de todos estos objetos interconectados puede verse representada gráficamente en la Figura \ref{fig:IoT_General}.
%
%\begin{figure}[H]
%  \centering
%  \includegraphics[width=0.7\linewidth]{figures/IoT-Architecture-Layers-and-Components.png}
%  \caption{Diagrama general de un sistema IoT}
%  \label{fig:IoT_General}
%\end{figure}


\section{Contexto tecnológico}

    En esta sección se describirán las tecnologías y herramientas más relevantes que se utilizarán a lo
    largo del desarrollo proyecto. El uso de las mismas se describirá en la sección \ref{chapter:desarrollo}.
    \todo{referenciar seccion correspondiente}

    \subsection{Gestión del proyecto}
        \subsubsection{Notion}
        \subsubsection{Microsoft Teams}
        \subsubsection{Zotero}
        \subsubsection{\LaTeX y Overleaf}
        

    \subsection{Ingeniería del Software}
        \subsubsection{Inyección de dependencias}
        \subsubsection{Integración continua}
        \subsubsection{Control de versiones}

    \subsection{Aplicación móvil}

        \subsubsection{Android}

            A grandes rasgos, Android es un Sistema Operativo orientado a dispositivos móviles basado en el núcleo 
            Linux, diseñado para ser independente de la arquitectura hardware de dichos dispositivos. 
            Si bien originalmente fue planteado para teléfonos móviles, con el avance de la industria ha adopado 
            un enfoque más amplio y es compatible con más dispositivos: tabletas, relojes inteligentes, televisores, 
            pantallas de automóviles... aunque, excepto en el caso de las tabletas, se trata de versiones basadas en
            Android con su propia idionsincrasia. \newline

            \begin{figure}[h]
                \centering
                \includegraphics[width=0.25\textwidth]{figures/Android logo.png}
                \caption[Logo actual de Android.]
                {Logo actual de Android. Imagen extraída de \cite{vulcansphere_english_2019}}
                \label{figure:android:logo}
            \end{figure}

            Normalmente cuando nos referimos a Android, no nos referimos únicamente al sistema operativo, sino a la
            plataforma creada entorno al mismo; como haremos a lo largo de este proyecto. Dicha plataforma o 
            \textit{framework} consta de numerosas capas, siendo el sistema operativo una parte de ellas. El sistema 
            operativo como tal es denominado AOSP o \textit{Android Open Source Project}, siendo su código fuente 
            público. Cualquier persona puede acceder a él, descargarlo y modificiarlo \cite{collado_que_2022}.

            \begin{figure}[h]
                \centering
                \includegraphics[width=0.66\textwidth]{figures/Android capas.jpg}
                \caption[Capas de Android.]
                {Capas de Android. Imagen extraída de \cite{perez_aosp_2019}}
                \label{figure:android:capas}
            \end{figure}

            No obstante, en la inmensa mayoría de los teléfonos móviles el sistema operativo es complementado con,
            entre otros, los GMS (\textit{Google Mobile Services}, o servicios de Google), los cuales solo están 
            disponibles bajo licencia; otorgada a los fabricantes que cumplen con una serie de requisitos. Los GMS 
            se utilizan para tareas como la gestión de notificaciones, servicios de geolocalización... además de para 
            acceder a las herramientas de Google, como la tienda de aplicaciones Play Store. Los fabricantes también 
            pueden personalizar y añadir funciones al sistema operativo, lo que explica que dos terminales con la misma 
            versión puedan verse tan diferentes entre sí. \newline
            

            Por otra parte, Android fue inicialmente desarrollado por la empresa homónima, si bien fue comprada en 2005
            por Google por 50 millones de dólares. La salida del sistema operativo se produciría dos años después, el 5 
            de noviembre de 2007, si bien el primer terminal que lo utilizaba (HTC Dream, también conocido como 
            T-Mobile G1) fue comercializado el 23 de septiembre de 2008 \cite{adeva_android_2023} \cite{marquez_asi_2022}.

            \begin{figure}[h]
                \centering
                \includegraphics[width=0.5\textwidth]{figures/HTC Dream.jpg}
                \caption[HTC Dream en funcionamiento.]{HTC Dream en funcionamiento. Imagen extraída de \cite{oryl_t-mobile_2008}}
                \label{figure:android:htc_dream}
            \end{figure}

            Desde entonces, numerosas versiones de Android han sido lanzadas, siendo la última versión estable Android 
            13; estableciéndose por parte de Google la \textit{costumbre} de lanzar cada año una nueva versión principal. 
            En cada una de ellas se introducen nuevas características, pero esto no significa que todos los dispositivos 
            puedan actualizar. Los fabricantes no están obligados a actualizar sus terminales, lo que en la práctica 
            supone que las nuevas versiones no son utilizadas masivamente y que los programadores deben de tener en 
            cuenta las versiones antiguas en sus aplicaciones. 
            \newline

            Debido a que Google dejó de publicar oficialmente las estadísticas de uso de su sistema operativo, no es
            posible conocer con plena exactitud dichas cifras. La comunidad se ha encargado de estimar dicha 
            información \cite{belinski_android_nodate}; relevando que a fecha de mayo de 2023 sólo el 20\% de los 
            dispositivos tienen la última versión, mientras que las versiones 12,11 y 10 están presentes en el 
            20,8\%, 21,1\% y 16,6\% respectivamente. \newline

            \begin{figure}[H]
                \centering
                \includegraphics[width=1\textwidth]{figures/Android usage.PNG}
                \caption[Estadísiticas acumulativas de las versiones de Android]
                {Estadísiticas acumulativas de las versiones de Android. Imagen extraída de \cite{belinski_android_nodate}}
                \label{figure:android:usage}
            \end{figure}

            Por último, a fecha de marzo de 2023, Android dispone de una cuota de mercado del 71\% en el segmento de sistemas 
            operativos para dispositivos móviles, teniendo su mayor rival,el sistema operativo iOS (propiedad de Apple) 
            un 28\%. Entre ambos acaparan el mercado, con un 99\% de cuota de mercado. En cuanto a España, el 
            porcentaje de Android asciende hasta el 77,73\% por el 21,81 de iOS \cite{press_asi_2023}.

        \subsubsection{Kotlin}

            Durante el diseño de Android se estableció que el lenguaje principal para desarrollar aplicaciones sería 
            Java, si bien incorpora soporte para utilizar código C y C++ \cite{android_developers_como_nodate}. No
            obstante, al ser Java un lenguaje interpretrado sobre una máquina virtual (JVM o \textit{Java Virtual
            Machine}) se abrió la puerta para  utilizar otros lenguajes que utilizasen la JVM. En la conferencia de 
            Google \textit{I/O} de 2017 fue anunciado el soporte oficial y completo 
            de Kotlin dentro de Android. \newline

            Kotlin es un lenguaje de programación desarrollado por JetBrains\footnote{JetBrains es una
            empresa muy reconocida dentro de la industria por crear una serie de entornos de desarrollo muy populares,
            como PHPStorm, CLion o Intellij IDEA. Sobre este último está construido Android Studio, el entorno de 
            desarrollo oficial dentro de Android.} y publicada su primera versión estable el 15 de febrero de 2016.
            El objetivo de este lenguaje fue tan sencillo como ambicioso: crear un lenguaje conciso (permitiendo reducir
            la cantidad de código inútil), con soporte de nuevas funcionalidades; pero sin renunciar a la rapidez de
            compilación de Java ni al todo el código escrito en él 
            \cite{rao_k_history_nodate}. \newline

            Sus principales características son las siguientes \cite{noauthor_kotlin_nodate} \cite{noauthor_enfoque_nodate}:
            \begin{itemize}
                \item Interoperable al 100\% con Java, lo que facilita la reutilización de código ya existentes. 
                Es interoperable en ambos sentidos.
                \item Permite escribir código más seguro, ya que en el diseño del lenguaje se solucionaron problemas
                crónicos de Java como las \textit{Null Pointer Exception}. Según datos internos de Google, las 
                aplicaciones escritas en Kotlin tienen un 20\% de probabilidades menos de fallar.
                \item Soporte nativo y estructurado para la programación concurrente y asíncrona mediante 
                construcciones como las corrutinas y los flujos.
                \item Desarrollo multiplataforma, no solo para Android: aplicaciones web, \textit{backend} e iOS.
                \item Permite desarrollos en varios paradigmas: orientada a objetos, funcional, imperativa...
            \end{itemize}
            
            Asimismo, al convertirse en la \textit{I/O} de 2019 en el lenguaje de referencia para el desarrollo 
            de Android \cite{braun_celebrating_2022}, los desarrollos de librerías y herramientas relacionadas con 
            Android están escritas en este lenguaje, aprovechando al máximo sus nuevas características. Por tanto, si
            bien es interoperable con Java, está recomendado que los nuevos desarrollos lo utilicen 
            \cite{lardinois_kotlin_2019}, como se ha realizado en este proyecto.
            

        \subsubsection{Jetpack Compose}

            Jetpack Compose es un conjunto de herramientas \textit{modernas} de Android para el desarrollo de 
            interfaces gráficas, lanzado en su primera versión estable el 28 de julio de 2021 por Google
            \cite{bellini_jetpack_2021}. Este kit de librerías permite desarrollar en el ecosistema Android de 
            forma nativa interfaces gráficas de manera declarativa como en los sistemas React, Flutter o SwiftUI;
            siguiendo las tendencias actuales de la industria en el desarrollo de aplicaciones móviles. \newline

            Este enfoque declarativo nos permite describir cómo queremos que sea nuestra interfaz gráfica. 
            Asimismo, las interfaces que construimos con este sistema pueden estar interconectadas a un estado 
            que definamos; describiendo cómo será nuestra interfaz para cada posible estado. Cuando ese estado cambie, 
            nuestra interfaz gráfica cambiará automáticamente para mostrar el nuevo estado, simplificando enormemente 
            el desarrollo y reduciendo el código necesario \cite{leiva_que_2021}. \newline

            Hasta la aparición de Jetpack Compose, el desarrollo interfaces gráficas nativas en Android se realizaba 
            con el enfoque conocido como programación imperativa. En este tipo de desarrollo es necesario especificar 
            paso por paso cómo se va a construir dicha interfaz gráfica exahustivamente. En dicho proceso (conocido 
            en Android como sistema de vistas) se codificaba un fichero XML, en el que se describían todos los 
            elementos gráficos (botones, textos...); para en el código Java/Kotlin de la aplicación se accediera 
            a dichos elementos y se le aplicaran manualmente modificaciones y transformaciones 
            \cite{noauthor_programacion_2021}. \newline

            Además, en Jetpack Compose, a diferencia del sistema anterior, los componentes gráficos están desacoplados 
            del sistema operativo; por lo que no dependemos de la versión del terminal para mostrar correctamente 
            nuestra interfaz gráfica. Eso ocurría anteriormente y como ya vimos, la fragmentación en Android es un 
            problema endémico, lo que complicaba bastante el desarrollo. Asimismo, es compatible con los componentes XML 
            del sistema anterior, lo que facilita la migración de los proyectos antiguos a este nuevo paradigma. \newline

            No obstante, como ya vimos en el apartado anterior, está diseñado para ser utilizado desde Kotlin, por lo 
            que en la práctica obliga a usar dicho lenguaje; lo que en algunos casos puede resultar en un pico de 
            dificultad hasta que se domina el lenguaje.
        

        \subsubsection{Material Design 3}
            Material Design 3 (también conocido como \textit{Material You}) es la tercera iteración del 
            conjunto de principios y directrices de diseño de Google, 
            como respuesta a la creciente ubiquidad de Android: móviles con pantallas
            plegables, \textit{smartwatch}, televisores \cite{ramirez_que_2022}... 
            Su primera implementación estable para Jetpack Compose fue lanzada el 
            24 de octubre de 2022 \cite{singh_material_2022}. \newline
            
            Al ser utilizado por Google para la creación de elementos 
            gráficos tanto en sus aplicaciones como en el sistema operativo, es la guía de diseño de facto dentro del
            ecosistema Android. \newline

            Sus principales características son las siguientes \cite{noauthor_material_nodate}:
            \begin{itemize}
                \item Centrado en la personalización de la interfaz gráfica. Los diseñadores definirán tres colores
                principales, los cuales serán utilizados para los elementos gráficos de forma totalmente transparente
                al programador. A partir de dichos colores, se elaborará mediante la herramienta 
                \textit{Material Design Builder} \cite{noauthor_material_nodate-1} una paleta de colores con 
                variantes de los mismos, 
                diseñada para cumplir estándares de accesibilidad\footnote{Describir dichos
                estándares está fuera del alcance del proyecto, pero es un proceso basado en proporciones de 
                luminitancia}, garantizando el nivel de contraste correcto.

                    \begin{figure}[h]
                        \centering
                        \includegraphics[width=0.75\textwidth]{figures/Material Design Builder example.png}
                        \caption[Ejemplo de uso de la herramienta \textit{Material Design Builder}]
                        {Ejemplo de uso de la herramienta \textit{Material Design Builder}. Imagen extraída de \cite{singh_material_2022}}
                        \label{figure:material_design_3:builder}
                    \end{figure}

                Además, si el dispositivo dispone de Android 12 o superior, pueden tomarse dichos
                colores desde el fondo de pantalla del usuario, incrementando exponencialmente la personalización;
                si bien se permite establecer colores \textit{fijos} para ciertos contenidos.
                \item Soporte nativo para categorizar el tamaño de la pantalla del dispositivo, tanto en altura como
                en anchura.
                    \begin{figure}[h]
                        \centering
                        \includegraphics[width=0.75\textwidth]{figures/Tamaños de ventana.png}
                        \caption[Categorías de pantalla según anchura]
                        {Categorías de pantalla según anchura. Imagen extraída de \cite{singh_material_2022}}
                        \label{figure:material_design_3:width_classes}
                    \end{figure}
                \item Sistema de fuentes basado en estilos principales para cada tipo de contenido: desde titulares 
                hasta etiquetas, pasando por títulos, cuerpos de texto...
                \item Soporte nativo para animaciones, las cuales ya son utilizadas en los componentes gráficos nativos,
                como los \textit{switch}.
                \item Evolución de muchos elementos gráficos, como las tarjetas, botones, selectores de fechas...
                    \begin{figure}[h]
                        \centering
                        \includegraphics[width=0.75\textwidth]{figures/Elementos gráficos material design 3.png}
                        \caption[Algunos elementos gráficos de Material Design 3]
                        {Algunos elementos gráficos de Material Design 3. Imagen extraída de \cite{cerda_material_2022}}
                        \label{figure:material_design_3:elementos_graficos}
                    \end{figure}
                \item Sistema de formas o \textit{redondeo} multinivel para modernizar nuestros elementos y hacerlos
                más distingubles entre sí.
        
                \item Mejora el concepto conocido como \textit{elevación}, basándose en colores y no en sombras. Este
                elemento permite superponer elementos y transmitir gráfica y visualmente la importancia de cada 
                uno de ellos.
                \item Soporte nativo para tema claro y oscuro, ya que en los últimos años los temas oscuros 
                se han hecho cada vez más populares en las aplicaciones y no estaba presente previamente de forma nativa.
            \end{itemize}

            En pocas palabras, en esta versión se han dedicado a modernizar el lenguaje de diseño, haciéndolo más 
            atractivo y completo; alejándose de lo puramente funcional, mejorando la accesibilidad y explorando 
            el mundo de la personalización para el usuario.

        \subsubsection{Salud Conectada}
            En la conferencia \textit{I/O} (sí, otra vez) de 2022 se anunció \textit{Health Connect} (o Salud Conectada
            en castellano), una aplicación creada por Google (junto con Samsung \cite{wilk_introducing_2022}) 
            que aglutinará todos los datos relacionados con salud dentro del ecosistema
            Android. La aplicación (desde el 11 de noviembre de 2022 en estado beta) es compatible con Android 9 o 
            superior, mientras que está previsto que para Android 14 venga
            incorporada aplicación de fábrica o preinstalada\footnote{Recordar aquí que esta aplicación será
            parte de los servicios de Google y no del sistema operativo propiamente dicho o AOSP.} 
            \cite{pandey_health_2023}. \newline

            \begin{figure}[h]
                \centering
                \includegraphics[width=0.25\textwidth]{figures/Health connect logo.png}
                \caption[Logo de Salud Conectada.]
                {Logo de Salud Conectada. Imagen extraída de \cite{noauthor_health_nodate}}
                \label{figure:health_connect:logo}
            \end{figure}

            Esta herramienta viene a solucionar una probelmática importante y es la ausencia del reaprovechamiento de 
            los datos, ya que hasta ese momento, la posición casi unánime dentro la industria se basaba en
            que los datos recoletados por sus dispositivos \textit{wearables} o aplicación se quedaran en su
            ecosistema \cite{ramirez_android_2022} \cite{rahman_android_2023}. \newline
            
            La única manera utilizar dichos datos 
            en otras aplicaciones estaba restringuida a sistemas propietarios como Google Fit, cuyos datos se 
            almacenaban enla nube con los problemas de privacidad asociados, o proyectos \textit{open source} como 
            Gadget Bridge \cite{freeyourgadget_gadgetbridge_nodate}; que accedían a datos mediante ingeniería inversa 
            de pulseras como las Xiaomi Mi Band. \newline

            A fecha de mayo de 2023, se estima que más de 100 aplicaciones han integrado Salud Conectada, incluyendo
            las aplicaciones de las pulseras Fitbit y Samsung, Peloton, Oura \cite{malik_googles_2023}... \newline

            Centrándonos en Salud Conectada, se trata tanto de una plataforma como una API para los desarrolladores.
            La plataforma puede registrar datos como la actividad física, el sueño, la nutrición o incluso el ciclo
            menstrual, siendo un intermediario entre las aplicaciones que generan o escriben dichos datos y las 
            que quieren acceder a esos datos. \newline

            \begin{figure}[h]
                \centering
                \includegraphics[width=0.33\textwidth]{figures/Arquitectura básica de Health Connect.png}
                \caption[Arquitectura básica de Salud Conectada.]
                {Arquitectura básica de  Salud Conectada. Imagen extraída de \cite{wilk_introducing_2022}}
                \label{figure:health_connect:logo}
            \end{figure}
            
            Asimismo, está diseñada con la privacidad en mente: los datos se almacenan localmente, mientras que el 
            acceso a los mismos esta fuertemente granularizado: en la que el usuario puede
            decidir qué aplicaciones tienen acceso (tanto lectura como escritura) a cada tipo de registro 
            \cite{saez_google_2022}.

            

        \subsubsection{Room}

    
    \subsection{Servidor}
        \subsubsection{Python}

        \subsubsection{Flask}

        \subsubsection{MongoDB}

\chapter{[En curso] Estado del arte}
\label{chapter:estado_arte}

\chapquote{Solo sé que no se nada.}{Sócrates}


En este capítulo se introduce una revisión del estado del arte y del estado de la cuestión en lo que respecta al marco teórico del proyecto que se propone en este \gls{tfm}, así como una descripción sobre los sistemas que existen en el mercado o que han sido reportados en la literatura, los cuales presentan aspectos comunes con la solución propuesta.
%\section{Energías renovables: ¿Qué son?}


\section{Análisis de la situación actual}

En esta sección, se pone en valor y/o contraste la información introducida en las secciones anteriores, de tal forma que pueda realizarse un análisis del entorno en el que se desarrolla este proyecto, así como también se pueda poner de manifiesto la relevancia y la idoneidad del desarrollo propuesto. 


\todo[inline]{Aqui si hay que introducir los wearbles}

\subsection{Proyectos relacionados}

Tenemos aqui el proyecto DemonicSalmon de 2018 de 72 universitarios entre 18 y 23 años. Datos no wearables: movilidad (que mejor la de la pulsera), actividad (aqui igual), patrones de comunicación.

Student life es el mitico de 2014, el de las 10 semanas con cuestionarios. Aqui usaban datos como micro, sensor de luz, gps, bluetooth y acelerometro para determinar actividad, conversacion,sueño y localizacion.

Sus cuestionarios de contraste con PSS, PHQ9 y UCLA pero solo ANTES y DESPUES de acabar el estudio, no diarios. Tirar con las correlaciones desde aqui.

El paper de Smart Devices andWearable Technologies to Detect and
Monitor Mental Health Conditions and Stress:
A Systematic Review sirve para ver que datos valen para algo: variabilidad del ritmo cardiaco,  electroencefalograma (nope). Mirar los resultados para eso. Es de 2021.

Dataset wesad, 2018 alemania, pero si son invasivos. Ahi tiramos la trama para desmarcarnos de eso. Lo mismo con PASS, diciembre 2020 pero también pilla ritmo caridaco.

\subsection{Contribución de la solución propuesta}

Usamos datos de wearables compatibles con varios fabricantes.
Cuestionarios diarios además de los de 12 semanas, tanto mañana y noche.
\chapter{[Por revisar] Metodología}
\label{chapter:metodologia}

\chapquote{Caer está permitido, levantarse es obligatorio.}{Proverbio ruso}

En primer lugar, se puede definir como metodología de desarrollo software al conjunto de técnicas, prácticas y métodos que se utilizan para diseñar e implementar un sistema basado en \textit{software}, con la finalidad de organizar de la mejor manera posible los equipos de trabajo involucrados en el desarrollo software \cite{santander_universidades_metodologias_2020}.

La elección de la metodología para un proyecto de desarrollo software depende de factores como el tamaño del equipo de desarrollo del proyecto, los requisitos del mismo y su posible cambio, las restricciones de tiempo... Asimismo, dicha elección impacta notablemente en ámbitos como la gestión de los recursos, la planificación de las actividades, la comunicación entre los miembros del equipo o a la evaluación de la calidad del producto.

\section{Metodologías de desarrollo del \textit{software} de Sistemas}

    Las metodologías para el desarrollo de software han sufrido una constante evolución desde su aparición en los años 60 del siglo pasado y presentan grandes diferencias entre sí. En esta sección serán introducidas algunas de las metodologías más relevantes, remitiendo al lector a las publicaciones citadas si desea profundizar en las mismas.

    \subsection{Metodologías Tradicionales}
    
        Estas metodologías se basan en una estructura lineal, donde se completan las fases del desarrollo en secuencia. El calificativo tradicional se debe a su creación en las primeras etapas de la Ingeniería del Software.
    
        \begin{itemize}
        \item Cascada: este modelo, según  \cite{sommerville_software_2011}, separa las actividades fundamentales tales como la especificación, desarrollo, validación y evolución, en una secuencia de fases: definición de requerimientos, diseño del sistema software, implementación y prueba de unidad, integración y pruebas del sistema y, por último, operación y mantenimiento. El desarrollo se lleva a cabo completando cada una de estas fases de forma secuencial, pudiendo regresar a cualquiera de las fases anteriores solo tras haber completado previamente la secuencia completa.
    
        Por otra parte, existen modificaciones de esta metodología, como \textit{cascada con retroalimentación}, que añaden la posibilidad de volver a la fase anterior de desarrollo en caso de identificarse problemas, sin necesidad de finalizar todas las etapas; como puede verse en la figura \ref{fig:metodologia:cascada_retroalimentada}.
    
        \begin{figure}[h]
            \centering
            \includegraphics[width=0.66\textwidth]{figures/cascada retroalimentada.JPG}
            \caption[Modelo en cascada con retroalimentación, extraído de \cite{sommerville_software_2011}]{Modelo en cascada con retroalimentación.}
            \label{fig:metodologia:cascada_retroalimentada}
        \end{figure}
        
        \item Modelo en V: considerado como una variante de la metodología en cascada, se representan las acciones a seguir en una V, como se puede ver en la Figura \ref{fig:metodologia:modelo_v}. Según \cite{pressman_software_2005}, a medida que el equipo de software avanza hacia abajo desde el lado izquierdo de la V, los requerimientos básicos del problema mejoran hacia representaciones técnicas cada vez más detalladas del problema y de su solución. 
        
        Una vez que se ha generado el código, el equipo sube por el lado derecho de la V, y en esencia ejecuta una serie de pruebas (acciones para asegurar la calidad) que validan cada uno de los modelos creados cuando el equipo fue hacia abajo por el lado izquierdo donde se aprecia la relación entre las acciones para  el aseguramiento de la calidad y aquellas asociadas con la comunicación, modelado y construcción temprana. 
    
        \begin{figure}[h]
            \centering
            \includegraphics[width=0.5\textwidth]{figures/en v.JPG}
            \caption[Modelo en V, extraído de \cite{pressman_software_2005}]{Modelo en V.}
            \label{fig:metodologia:modelo_v}
        \end{figure}
    \end{itemize}

    \subsection{Metodologías Ágiles}

        Este conjunto de métodos deben su nombre al \textit{Manifiesto Ágil} \cite{varios_autores_manifiesto_2001}, donde se reivindica un enfoque que apueste por ``individuos e interacciones sobre procesos y herramientas, software funcionando sobre documentación extensiva, colaboración con el cliente sobre negociación contractual y respuesta ante el cambio sobre seguir un plan``. Algunas de las metodologías que implementan estos principios son:
    
        \begin{itemize}
            \item Scrum, si bien es previa al \textit{Manifiesto Ágil}, sus principios son congurentes con el manifiesto. En esta aproximación se utiliza un enfoque iterativo para el desarrollo, bajo el concepto de \textit{sprint}, tal y como se puede visualizar en la Figura \ref{fig:metodologia:scrum} . Un \textit{sprint} es una unidad de trabajo de duración fija (normalmente dos semanas) en la cual no se pueden introducir cambios, permitiendo cierta estabilidad a la vez que se abraza el cambio \cite{pressman_software_2005}. 
    
            \begin{figure}[h]
                \centering
                \includegraphics[width=0.66\textwidth]{figures/scrum.JPG}
                \caption[El proceso de Scrum, extraído de \cite{sommerville_software_2011}]{El proceso de Scrum}
                \label{fig:metodologia:scrum}
            \end{figure}
            
            Asimismo, en esta unidad de tiempo se incorporan una serie de reuniones o ceremonias, tales la reunión diaria o \textit{daily} ,en la que en 15 minutos cada integrante del equipo expresa qué ha hecho desde la reunión anterior, sus planes de cara a la siguiente y los obstáculos que está encontrando; la \textit{Sprint Planning}, en la que el equipo de desarrollo decide las tareas a realizar en el siguiente sprint, o la \textit{Sprint Retrospective}, donde se analiza la ejecución del sprint anterior para plantear posibles mejoras. 
    
            Por otra parte, se definen roles en el equipo tales como \textit{Scrum Master}, el cual se encarga de la comunicación con otros equipos y clientes, modera las reuniones y registra las decisiones entre otras cuestiones, o el \textit{Product Owner}, siendo el responsable del proyecto en cuanto al desarrollo, mantenimiento y priorización de las tareas \cite{valtx_metodologias_2023}.
    
            \item Kanban es una metodología ágil creada por Toyota, la cual tiene como elemento diferenciador el uso de tarjetas visuales (kanban en japonés) \cite{pzt_metodologias_nodate}, representadas en forma de un tablero donde se refleja el flujo de los procesos en un trabajo designado, permitiendo a cada responsable mover sus tareas libremente según los avances. Ese flujo de trabajo puede ser algo tan sencillo como ``Por hacer``, ``En curso`` y ``Terminado`` \cite{atlassian_que_nodate-1}, como se puede ver en la figura \ref{fig:metodologia:kanban}.
    
            \begin{figure}[h]
                \centering
                \includegraphics[width=0.66\textwidth]{figures/kanban.png}
                \caption[Tablero Kanban, extraído de \cite{stsepanets_metodo_2024}]{Tablero Kanban}
                \label{fig:metodologia:kanban}
            \end{figure}
            
    
            Según \cite{valtx_metodologias_2023}, algunos de sus principios son: 
    
            \begin{enumerate}
                \item Visualizar lo que hace. La visualización por parte de todos los responsables del proyecto en el flujo de las tareas permite mantenerse atentos sobre el desarrollo del proyecto.
                \item Limitar la cantidad de trabajo en proceso. El establecimiento de metas alcanzables permite al grupo un equilibrio en el flujo del trabajo y previene el exceso de procesos centralizados en pocos responsables.
                \item Realizar seguimiento del tiempo. El manejo del tiempo de forma organizada permite obtener resultados precisos en el proyecto.
                \item Lectura fácil de indicadores visuales. La visualización de los tipos de trabajo, prioridades, fechas y demás detalles empodera al equipo en el desarrollo de soluciones ajustadas a la realidad.
            \end{enumerate}
            
        \end{itemize}

\section{Metodología de desarrollo seleccionada}

    Para el desarrollo de este proyecto se ha elegido la metodología \textbf{en cascada con retroalimentación}, justificándose dicha elección con las siguientes razones:
    
    \begin{itemize}
        \item Equipo reducido: el equipo de desarrollo de este proyecto se ha compuesto únicamente por el autor, con la supervisión de sus dos directoras. Como se describió anteriormente, metodologías como Scrum definen una serie de roles claramente diferenciados; por lo que no se adaptan a un equipo tan reducido.
        \item Estabilidad de los requisitos: los requisitos que fundamentan este proyecto son definidos claramente desde el inicio, siendo poco propensos a ser modificados durante la ejecución del mismo. 
        \item Definición clara y sencilla de las fases: se busca una metodología que permita trabajar eficientemente, con la menor sobrecarga posible mientras se garantice la calidad. Al asumirse un cambio inexistente o relativamente pequeño, se puede contemplar el uso de metodologías tradicionales en lugar de las ágiles, al no necesitarse la sobrecarga de roles y ceremonias. 
        \item Retroalimentación periódica: si bien no se contemplan cambios significativos durante el desarrollo del proyecto, se desea cierta flexibilidad que permita descubrir y corregir errores de forma temprana. Metodologías tradicionales, como en cascada sin retroalimentación, muestran un enfoque muy rígido para la identificación y resolución de problemas durante el transcurso del proyecto.
    \end{itemize}
\chapter{Análisis del sistema propuesto}
\label{chapter:analisis}

\chapquote{Cada día me gusta levantarme porque hay otro reto.}{Roger Penske}

\section{Introducción}

    \subsection{Propósito}
        El propósito de este capítulo consiste en la Especificación de Requisitos Software (ERS) correspondiente al proyecto desarrollado en este Trabajo Fin de Máster (TFM). Para la elaboración de la misma se ha tomado como base el estándar IEEE 830.

        Esta especificación recoge todas las características, objetivos, restricciones y suposiciones para el desarrollo del proyecto, definiendo claramente el sistema a desarrollar y sirviendo asimismo de referencia para la verificación y validación de la solución implantada.

        Por otra parte, dado que este proyecto tiene una naturaleza abierta y transparente, en especial al tratar datos sensibles; el público objetivo de esta especificación de requisitos es cualquier persona que por cualquier razón desee conocer las características del sistema y sobre qué criterios, restricciones o supuestos ha sido diseñado, ya que consideramos que esta transparencia es vital para que el conjunto de la sociedad pueda confiar en este proyecto.
        
    \subsection{Alcance}
        Como ya se mencionó en la sección \ref{sec:objetivos}, el objetivo principal de este Trabajo Fin de Master es la creación de un prototipo de un sistema que permita la detección precoz y la mejora de los trastornos de salud mental en la comunidad universitaria. 

        Este proyecto se fundamenta en la alarmante situación psicológica que atraviesa nuestra sociedad, partiendo de la premisa de que en numerosas ocasiones las personas desconocen factores, conductas o síntomas relacionados con problemas de salud mental, pero que mejorando la conciencia de los mismos se podría reducir el número y la gravedad de los casos.

        Para ello, este sistema se encargará de monitorizar las medidas de estrés, soledad y depresión y suicidio, en base a cuestionarios que la persona rellene diariamente. Asimismo, la aplicación recopilará los datos de su dispositivo \textit{wearable} a los que el usuario haya dado acceso, con la finalidad de construir en el marco de la escuela un conjunto de datos anonimizado con el que se pueda desarrollar en el futuro un modelo predictivo de esas medidas, con el que se pueda dar un apoyo más integral al usuario.

        Asimismo, este sistema no está planteado como un reemplazo de los profesionales de la Psicología y Psiquiatría, sino como un complemento para mejorar la sensibilidad de los usuarios sobre la evolución de su salud mental, aportando consejos y remitiendo a un profesional cuando la situación del usuario sea delicada; quedando fuera del proyecto cuestiones como el asesoramiento médico profesional o una atención psicológica personalizada.

        \todo[inline]{Tenemos que hablar aqui de la comunidad}

        Dichos consejos estarán personalizados para el nivel de gravedad (o ausencia de la misma) de cada medida, permitiendo además que el usuario pueda ver claramente cómo se ha sentido durante los últimos días, semanas o meses para cada variable.

        En cuanto a la obtención de los datos de los dispositivos, se utilizará un \textit{framework} ya desarrollado por la industria, por lo que la compatibilidad del sistema estará limitada por este factor. La extracción de datos por otros canales queda fuera del alcance, ya que los cauces legales están fuertemente restringidos. Asimismo, el sistema accederá a esos datos únicamente en modo lectura.

        \todo[inline]{Hablamos aquí de que hemos tenido la colaboración/asesoría de dos psicólogas?}
        
    \subsection{Definiciones, Acrónimos y Abreviaturas}
        TBD
        
    \subsection{Estructura del documento}
        TBD
        
\section{Descripción general del producto}

    \subsection{Perspectiva del producto}
        Bienestar Emocional se constituye bajo dos componentes fundamentales. En primer lugar, una aplicación Android construida desde cero que haga uso del \textit{framework} Salud Conectada, el cual es desde la última versión de Android parte del propio sistema operativo. Este componente nos permite la abstracción de los componentes hardware, delegando a los fabricantes la conexión de sus dispositivos con este sistema.

        En segundo lugar, se desplegará en un servidor de la Universidad un componente que permita tanto la subida de los datos del usuario como la obtención de diversas estadísticas sobre la evolución de las medidas en el resto de la comunidad universitaria.

        Como se comentó anteriormente, este proyecto está planteado como un prototipo, sirviendo como punto de partida para proyectos de futuros alumnos, los cuales puedan ampliar las funcionalidades de esta aplicación sin necesidad de rehacer gran parte del trabajo. Asimismo, se busca que este sistema sea un marco para la investigación en el campo de la Psicología, permitiendo la realización de experimentos que necesiten acceso a datos de salud o cuestionarios.


    \subsection{Características de los usuarios finales}

    TBD

    %Los usuarios finales de este sistema serán los miembros de la comunidad universitaria que deseen monitorizar su salud mental. No obstante, deberán de cumplir ciertas pautas.
    
    %\begin{enumerate}
    %    \item Los usuarios dispondrán de un teléfono con sistema operativo Android, en su versión 8 o superior.
    %    \item Los teléfonos dispondrán de conexión a Internet.
    %    \item El usuario realizará de forma habitual los cuestionarios ofrecidos por la aplicación.
    %    \item El usuario estará dispuesto a que sus respuestas a los cuestionarios sean subidas a un servidor de la universidad, con un identificador de usuario anónimo.
    %    \item Opcionalmente, el usuario dispondrá de un dispositivo \textit{wearable} de los fabricantes Fitbit o Samsung, con la aplicación del fabricante correspondiente instalada y el dispositvo conectado a su teléfono.
        
    %\end{enumerate}
        
    \subsection{Restricciones generales}
        \begin{enumerate}[label=\textbf{RG-\arabic*}]
            \item La aplicación móvil se ejecutará sobre el sistema operativo Android. Concretamente, se tomará Android 14 como referencia.
            \item La aplicación móvil especificará Android 9 como versión mínima, pudiéndose ejecutar la aplicación en cualquier versión entre la mínima y la referente.
            \item La comunicación con los dispositivos wearables se realizará mediante el \textit{framework} de Android \textit{Health Connect}.
            \item La aplicación móvil será implementada utilizando el lenguaje de programación Kotlin.
            \item El componente servidor será implementado mediante el lenguaje de programación Python.
            \item Se utilizará como dispositivo \textit{wearable} la pulsera Fitbit Inspire 2.
        \end{enumerate}
    
    \subsection{Suposiciones}

        \begin{enumerate}[label=\textbf{SUP-\arabic*}]
            \item Se supone que el usuario de la aplicación dispone de conexión a Internet.
            \item Se supone que el usuario de la aplicación es el mismo que utiliza el dispositivo \textit{wearable}.
            \item Se supone que las respuestas del usuario a los cuestionarios son veraces y verídicas.
            \item Se supone que el usuario rellena frecuentemente los cuestionarios.
            \item Se supone que el usuario utiliza con cierta asiduidad el dispositivo \textit{wearable}.
            \item Se supone que el servidor donde se aloja el componente homónimo está disponible.
            \item Se supone que el usuario desactiva la optimización de batería de la aplicación móvil.
            \item Se supone que habrá más de un usuario utilizando la aplicación diariamente.
        \end{enumerate}
        
    \subsection{Dependencias}
        \todo[inline]{Alguna más quizá?}
        \begin{enumerate}[label=\textbf{DEP-\arabic*}]
            \item Para la lectura de los datos del dispositivo \textit{wearable}, el fabricante debe integrar su aplicación móvil con el \textit{framework} \textit{Health Connect}, asegurando la escritura de los datos recogidos.
            \item Para la ejecución de las tareas recurrentes, es necesario que el sistema operativo otorgue recursos de ejecución a la aplicación móvil.
        \end{enumerate}

\section{Requisitos específicos}

    \subsection{Requisitos de usuario}

        \begin{enumerate}[label=\textbf{\texttt{RU-\arabic*}}]
            \item Como usuario, quiero realizar un seguimiento del estrés.
            \item Como usuario, quiero realizar un seguimiento de la depresión.
            \item Como usuario, quiero realizar un seguimiento de la soledad no deseada.\footnote{Para simplificar la redacción de los requisitos, mientras no se especifique lo contrario interpretaremos soledad como soledad no deseada.}
            \item Como usuario, quiero visualizar el resultado de la última medición del estrés.
            \item Como usuario, quiero visualizar el resultado de la última medición de la depresión.
            \item Como usuario, quiero visualizar el resultado de la última medición de soledad.
            \item Como usuario, quiero disponer de al menos un consejo para cada nivel categórico de estrés.
            \item Como usuario, quiero disponer de al menos un consejo para cada nivel categórico de depresión.
            \item Como usuario, quiero disponer de al menos un consejo para cada nivel categórico de soledad.
            \item Como usuario, quiero visualizar la evolución de mis registros de estrés.
            \item Como usuario, quiero visualizar la evolución de mis registros de depresión.
            \item Como usuario, quiero visualizar la evolución de mis registros de soledad.
            \item Como usuario, quiero visualizar la media de estrés del resto de usuarios del día anterior.
            \item Como usuario, quiero visualizar la media de depresión del resto de usuarios del día anterior.
            \item Como usuario, quiero visualizar la media de soledad del resto de usuarios del día anterior.
            \item Como usuario, quiero visualizar la media de estrés del resto de usuarios de los últimos siete días naturales.
            \item Como usuario, quiero visualizar la media de depresión del resto de usuarios de los últimos siete días naturales.
            \item Como usuario, quiero visualizar la media de soledad del resto de usuarios de los últimos siete días naturales.
            \item Como usuario, quiero visualizar, dentro de la semana en curso, la evolución de la media de estrés diario del resto de usuarios.
            \item Como usuario, quiero visualizar, dentro de la semana en curso, la evolución de la media de depresión diaria del resto de usuarios.
            \item Como usuario, quiero visualizar, dentro de la semana en curso, la evolución de la media de soledad diaria del resto de usuarios.
            \item Como usuario, quiero visualizar los datos de actividad física recogidos por la aplicación.
            \item Como usuario, quiero utilizar la aplicación sin necesidad de disponer de Internet en todo momento, por consiguiente estarán disponibles todas las funciones relativas al usuario.
            \item Como analista de datos, quiero poder recibir los datos anonimazados de estrés recogidos por la aplicación móvil.
            \item Como analista de datos, quiero poder recibir los datos anonimazados de depresión recogidos por la aplicación móvil.
            \item Como analista de datos, quiero poder recibir los datos anonimazados de soledad recogidos por la aplicación móvil.
            \item Como analista de datos, quiero poder recibir los datos anonimazados de actividad física recogidos por la aplicación móvil.
        \end{enumerate}
    
    \subsection{Requisitos funcionales}
        TBD

        \subsubsection{Aplicación móvil}

        \begin{enumerate}[label=\textbf{\texttt{RF-\arabic*}}]
            \item El seguimiento del estrés se realizará mediante dos cuestionarios diarios, uno durante la mañana y otro durante la noche, y puntuales.
            \todo[inline]{Decimos aqui que los cuestionarios diarios deben diseñarlos psicólogos?}
            \item El cuestionario puntual para el seguimiento del estrés será el PSS-14.
            \item La medida estrés estará calibrada en tres umbrales: baja, moderada y alta.
            \item Para cada umbral de la medida estrés se dispondrá de al menos un mensaje o recomendación.
            \item El seguimiento de la depresión se realizará mediante dos cuestionarios diarios, uno durante la mañana y otro durante la noche, y puntuales.
            \item El seguimiento de la depresión podrá ser activado y desactivado por el usuario en todo momento.
            \todo[inline]{Decimos aqui que los cuestionarios diarios deben diseñarlos psicólogos?}
            \item El cuestionario puntual para el seguimiento de la depresión será el PHQ-9.
            \item La medida depresión estará calibrada en tres umbrales: baja, moderada y alta.
            \item Para cada umbral de la medida depresión se dispondrá de al menos un mensaje o recomendación.
            \item El seguimiento de la soledad se realizará mediante dos cuestionarios diarios, uno durante la mañana y otro durante la noche, y puntuales.
            \item El seguimiento de la soledad podrá ser activado y desactivado por el usuario en todo momento.
            \todo[inline]{Decimos aqui que los cuestionarios diarios deben diseñarlos psicólogos?}
            \item El cuestionario puntual para el seguimiento de la soledad será el UCLA-20.
            \item La medida soledad estará calibrada en tres umbrales: baja, moderada y alta.
            \item Para cada umbral de la medida soledad se dispondrá de al menos un mensaje o recomendación.

            \item La aplicación dispondrá de un cuestionario diario sobre suicidio. Dicho cuestionario será creado por la noche.
            \todo[inline]{Decimos aqui que los cuestionarios diarios deben diseñarlos psicólogos?}
            \item La medida de riesgo de suicidio estará calibrada en tres umbrales: bajo, moderado y alto.
            \item Para cada umbral de la medida riesgo de suicidio se dispondrá de al menos un mensaje o recomendación.
            \item La aplicación dispondrá de un cuestionario diario denominado "¿Cómo has estado hoy?". Dicho cuestionario será creado por la noche.
	    \todo[inline]{Decimos aqui que los cuestionarios diarios deben diseñarlos psicólogos?}

            \item Durante la realización de un cuestionario, la aplicación permitirá al usuario no completarlo en ese momento si así lo desea.
            \item La aplicación guardará el estado del cuestionario si el usuario decide no completarlo en ese momento.

            \item El usuario podrá autorizar o no el acceso de la aplicación a los datos de distancia recorrida.
            \item El usuario podrá autorizar o no el acceso de la aplicación a los datos de desnivel ganado.
            \item El usuario podrá autorizar o no el acceso de la aplicación a los datos de las sesiones de ejercicio.
            \item El usuario podrá autorizar o no el acceso de la aplicación a los datos del desnivel \todo[inline]{es elevation gained pero suena horrible, mejor traducción?}.
            \item El usuario podrá autorizar o no el acceso de la aplicación a los datos de frecuencia cardíaca.
            \item El usuario podrá autorizar o no el acceso de la aplicación a los datos de sueño.
            \item El usuario podrá autorizar o no el acceso de la aplicación a los datos de pasos realizados.
            \item El usuario podrá autorizar o no el acceso de la aplicación a los datos de calorías quemadas.
            \item El usuario podrá autorizar o no el acceso de la aplicación a los datos de peso.

            \item La aplicación recibirá del servidor la media de los cuestionarios de depresión, soledad y depresión del día anterior.
            \item La aplicación recibirá del servidor la media de los cuestionarios de depresión, soledad y depresión de los últimos siete días.
            \item La aplicación recibirá del servidor la media de los cuestionarios de depresión, soledad y depresión de cada día de la semana actual. 

            \item La aplicación móvil notificará al usuario si necesita instalar el componente \textit{Health Connect} para los dispositivos con versiones de Android inferiores a la 14.

        \end{enumerate}

        \subsubsection{Componente servidor}
    
    \subsection{Requisitos no funcionales}
        TBD
        \subsubsection{Acurracy}
            \todo[inline]{Tiene sentido hablar de precisión en esta app?}
        \subsubsection{Availability}
            \todo[inline]{Brindis al sol de la disponibilidad del servidor?}
        \subsubsection{Eficiencia (Efficiency)}
            \todo[inline]{Queda claro que es necesario pero no suficiente?}
            \begin{enumerate}[label=\textbf{\texttt{RNF-\arabic*}}]
                \item La aplicación móvil sólo intentará enviar datos al servidor cuando se disponga de conexión a Internet.
                \item Si en el momento de envío de datos el dispositivo no dispone de conexión a Internet, se aplazará la subida de los mismos para para realizarse en cuanto se disponga de conexión.
                \item En cada envío de datos la aplicación móvil sólo enviará los datos recogidos desde el último envío.
            \end{enumerate}
        \subsubsection{Extensibilidad (Extensibility)}
            \begin{enumerate}[resume, label=\textbf{\texttt{RNF-\arabic*}}]
                \item El diseño y la aplicación garantizarán, en la medida de lo posible, el soporte a nuevas versiones de Android.
                \item El diseño y la aplicación garantizarán, en la medida de lo posible, la incorporación y/o eliminación de tipos de datos de actividad física.
            \end{enumerate}
        \subsubsection{Interoperabilty}
            \todo[inline]{Hablamos aqui de las unidades de medida de los datos?}
        \subsubsection{Mantenibilidad (Maintainability)}
            \todo[inline]{Podemos añadir algo de la entrega de actualizaciones..}
            \begin{enumerate}[resume, label=\textbf{\texttt{RNF-\arabic*}}]
                \item Se desarrollará el código software siguiendo los principios S.O.L.I.D.
            \end{enumerate}
        \subsubsection{Rendimiento (Performance)}
            \begin{enumerate}[resume, label=\textbf{\texttt{RNF-\arabic*}}]
                \item Los cuestionarios diarios de mañana serán creados a las 9:00 horas.
                \item Los cuestionarios diarios de noche serán creados a las 21:00 horas.
                \item Los cuestionarios puntuales serán creados a las 15:00 horas.
                \item La aplicación subirá al servidor los datos de salud a los que tenga permiso cada 8 horas.
                \item La aplicación subirá al servidor los datos de los cuestionarios diarios terminados cada 8 horas.
                \item La aplicación subirá al servidor los datos de los cuestionarios puntuales terminados cada 8 horas.
    
            \end{enumerate}
        \subsubsection{Portability}
            \begin{enumerate}[resume, label=\textbf{\texttt{RNF-\arabic*}}]
                \item La aplicación deberá poder leer los datos de cualquier dispositivo \textit{wearable} que haya integrado el fabricante del mismo con \textit{Health Connect}.
            \end{enumerate}
        \subsubsection{Recoverability}
            \todo[inline]{Tiene sentido en esta app?}
        \subsubsection{Reusability}
            \todo[inline]{Que podemos hablar aqui? Que los datos se mantendrán entre versiones de la app al definirse una migración? Tolerancia de errores? Tiene sentido en esta app?}
        \subsubsection{Robustness}
            \todo[inline]{Tolerancia de errores? Tiene sentido en esta app?}
        \subsubsection{Safety}
            \begin{enumerate}[resume, label=\textbf{\texttt{RNF-\arabic*}}]
                \item El ID de usuario será generado aleatoriamente por la aplicación y será distinto en cada instalación.
                \item La base de datos de la aplicación estará cifrada con el algoritmo AES, usando el modo de cifrado GCM.
            \end{enumerate}
        \subsubsection{Scalability}
            \begin{enumerate}[resume, label=\textbf{\texttt{RNF-\arabic*}}]
                \item Hola
                \item Mundo
            \end{enumerate}
        \subsubsection{Security / Integrity}
            \begin{enumerate}[resume, label=\textbf{\texttt{RNF-\arabic*}}]
                \item Hola
                \item Mundo
            \end{enumerate}
        \subsubsection{Usability}
            \begin{enumerate}[resume, label=\textbf{\texttt{RNF-\arabic*}}]
                \item Hola
                \item Mundo
            \end{enumerate}

\section{Requisitos de Interfaces Externas}

    \subsection{Interfaces de usuario}
        \begin{enumerate}[label=\textbf{\texttt{RIU-\arabic*}}]
            \item La interfaz de usuario deberá adaptarse al tamaño de la pantalla del dispositivo móvil.
            \item El diseño de la interfaz de usuario contemplará dos ajustes de colores principales: modo claro y oscuro.
            \item La aplicación móvil tomará por defecto el modo de colores general del sistema.
            \item El usuario podrá cambiar el modo de colores de la aplicación en todo momento. 
            \item La aplicación móvil ofrecerá la posibilidad, en los dispositivos móviles que lo soporten, la posibilidad de usar un esquema de colores basado en el fondo de pantalla del dispositivo.
            \todo[inline]{Estos requisitos pertenecen a otra categoría?}
            \item La aplicación móvil estará disponible en el idioma español, siendo el idioma por defecto.
            \item La aplicación móvil estará disponible en el idioma inglés.
            \item El usuario podrá cambiar el modo de colores de la aplicación en todo momento.
        \end{enumerate}

    \subsection{Interfaces Hardware}
        \todo[inline]{Aqui en todo caso hablariamos de que usamos wifi, datos móviles y bluetooth? De verdad hablamos de bluetooth si lo usa health connect y no nosotros?}
    
    \subsection{Interfaces Software}
        \begin{enumerate}[label=\textbf{\texttt{RIS-\arabic*}}]
            \item La comunicación entre aplicación móvil y servidor se implementará mediante una API REST.
            \item La base de datos en el componente servidor será implementada mediante MongoDB.
            \item La base de datos en la aplicación móvil se implementará mediante la librería Room.
            \item Cuando un cuestionario sea creado, la aplicación móvil deberá crear una notificación al usuario.
            \todo[inline]{No me cuadra la redacción del siguiente requisito}
            \item Si existe al menos un cuestionario no completado, deberá notificarse al usuario mediante un aviso dentro de la aplicación.
    
        \end{enumerate}
    
    \subsection{Interfaces de comunicaciones}
        \todo[inline]{Pondríamos aqui los tipos de mensajes que enviamos y recibimos? Lo que tenemos en la api vaya}
    
    \todo[inline]{Restricciones de entorno fisico no aplica verdad?}

\section{Buffer}
\begin{enumerate}
    \item El usuario podrá permitir o no las notificaciones (impuesto por Android)
    \item La aplicación dispondrá de un canal general de notificaciones
    \item La aplicación dispondrá de un canal exclusivo de notificaciones de cuestionarios.

    \item La aplicación dispondrá de una ventana de créditos.
    
    \item Los datos de los cuestionarios serán guardados en una base de datos relacional.
    \item Los ajustes de preferencias serán guardados en memoria persistente.
	
    \item El usuario podrá dar permiso de lectura a cada dato de salud independientemente.
\end{enumerate}
%\section {Análisis de Stakeholders}

\chapter{Diseño del sistema propuesto}
\label{chapter:disenio}

\chapquote{Si tienes tanto miedo al fracaso, nunca tendrás éxito. Tienes que arriesgarte.}{Mario Andretti}

\section{Descripción del caso de estudio} \label{section:CasoEstudio}


\section{Diseño del sistema}

    \subsection{Casos de uso}
    \subsection{Diagramas de secuencia}
    \subsection{Interfaz de usuario}

\section{Buffer}

\subsection{Casos de uso}

    \subsubsection{Caso de uso 1: Extracción de datos de un \gls{wearable}}


    \begin{table}[h]
        \centering
        \begin{tabularx}{\textwidth}{|c|X|X|}
            \hline
            Evento activador & \multicolumn{2}{|c|}{X} \\
            \hline
            Actor primario & \multicolumn{2}{|c|}{X} \\
            \hline
            Precondición & \multicolumn{2}{|c|}{X} \\
            \hline
            \multirow{2}{*}{Flujo normal} & Paso & Acción \\
            \cline{2-3} & X & I'have been waiting for the weekend to begin \\
            \hline
            \multirow{2}{*}{Flujo alternativo} & Paso & Acción \\
            \cline{2-3} & I'have been waiting for the weekend to begin & X \\
            \hline
            Postcondición & \multicolumn{2}{|c|}{X} \\
            \hline
            \multirow{2}{*}{Excepciones}  & Paso & Acción \\
            \cline{2-3} & X & X \\
            \hline
        \end{tabularx}
        \caption{Especificación del caso de uso 1: Extracción de datos de un \gls{wearable}}
        \label{tabla:caso_uso_1}
    \end{table}
    
    \subsubsection{Caso de uso 2: Seguimiento individual}

    \begin{table}[h]
        \centering
        \begin{tabularx}{\textwidth}{|c|X|X|}
            \hline
            Evento activador & \multicolumn{2}{|c|}{X} \\
            \hline
            Actor primario & \multicolumn{2}{|c|}{X} \\
            \hline
            Precondición & \multicolumn{2}{|c|}{X} \\
            \hline
            \multirow{2}{*}{Flujo normal} & Paso & Acción \\
            \cline{2-3} & X & I'have been waiting for the weekend to begin \\
            \hline
            \multirow{2}{*}{Flujo alternativo} & Paso & Acción \\
            \cline{2-3} & I'have been waiting for the weekend to begin & X \\
            \hline
            Postcondición & \multicolumn{2}{|c|}{X} \\
            \hline
            \multirow{2}{*}{Excepciones}  & Paso & Acción \\
            \cline{2-3} & X & X \\
            \hline
        \end{tabularx}
        \caption{Especificación del caso de uso 2: Seguimiento individual}
        \label{tabla:caso_uso_2}
    \end{table}
    
    \subsubsection{Caso de uso 3: Seguimiento conjunto}

    \begin{table}[h]
        \centering
        \begin{tabularx}{\textwidth}{|c|X|X|}
            \hline
            Evento activador & \multicolumn{2}{|c|}{X} \\
            \hline
            Actor primario & \multicolumn{2}{|c|}{X} \\
            \hline
            Precondición & \multicolumn{2}{|c|}{X} \\
            \hline
            \multirow{2}{*}{Flujo normal} & Paso & Acción \\
            \cline{2-3} & X & I'have been waiting for the weekend to begin \\
            \hline
            \multirow{2}{*}{Flujo alternativo} & Paso & Acción \\
            \cline{2-3} & I'have been waiting for the weekend to begin & X \\
            \hline
            Postcondición & \multicolumn{2}{|c|}{X} \\
            \hline
            \multirow{2}{*}{Excepciones}  & Paso & Acción \\
            \cline{2-3} & X & X \\
            \hline
        \end{tabularx}
        \caption{Especificación del caso de uso 3: Seguimiento conjunto}
        \label{tabla:caso_uso_3}
    \end{table}
    
    \subsubsection{Caso de uso 4: Recopilación del histórico?}

    \begin{table}[h]
        \centering
        \begin{tabularx}{\textwidth}{|c|X|X|}
            \hline
            Evento activador & \multicolumn{2}{|c|}{X} \\
            \hline
            Actor primario & \multicolumn{2}{|c|}{X} \\
            \hline
            Precondición & \multicolumn{2}{|c|}{X} \\
            \hline
            \multirow{2}{*}{Flujo normal} & Paso & Acción \\
            \cline{2-3} & X & I'have been waiting for the weekend to begin \\
            \hline
            \multirow{2}{*}{Flujo alternativo} & Paso & Acción \\
            \cline{2-3} & I'have been waiting for the weekend to begin & X \\
            \hline
            Postcondición & \multicolumn{2}{|c|}{X} \\
            \hline
            \multirow{2}{*}{Excepciones}  & Paso & Acción \\
            \cline{2-3} & X & X \\
            \hline
        \end{tabularx}
        \caption{Especificación del caso de uso 4: Recopilación del histórico?}
        \label{tabla:caso_uso_4}
    \end{table}

\subsection{Diagramas de secuencia}

    \subsubsection{Diagramas de secuencia 1: Extracción de datos de un \gls{wearable}}
        La secuencia lógica de la extracción de datos de un \gls{wearable} es la siguiente (Figura X):

        \begin{enumerate}
            \item Hola
            \item Mundo
            \item Bifurcación
            \begin{enumerate}
                \item Primario
                \item Alternativo
            \end{enumerate}
        \end{enumerate}
    
    \subsubsection{Diagramas de secuencia 2: Seguimiento individual}
        La secuencia lógica del seguimiento individual es la siguiente (Figura X):

        \begin{enumerate}
            \item Hola
            \item Mundo
            \item Bifurcación
            \begin{enumerate}
                \item Primario
                \item Alternativo
            \end{enumerate}
        \end{enumerate}
    
    \subsubsection{Diagramas de secuencia 3: Seguimiento conjunto}
        La secuencia lógica del seguimiento conjunto es la siguiente (Figura X):

        \begin{enumerate}
            \item Hola
            \item Mundo
            \item Bifurcación
            \begin{enumerate}
                \item Primario
                \item Alternativo
            \end{enumerate}
        \end{enumerate}
        
    \subsubsection{Diagramas de secuencia 4: Recopilación del histórico?}
        La secuencia lógica de la recopilación del histórico es la siguiente (Figura X):

        \begin{enumerate}
            \item Hola
            \item Mundo
            \item Bifurcación
            \begin{enumerate}
                \item Primario
                \item Alternativo
            \end{enumerate}
        \end{enumerate}

    \subsubsection{[WIP] Diagramas de secuencia 5: Post cuestionario}
        La secuencia lógica de la recopilación del histórico es la siguiente (Figura X):

        \begin{enumerate}
            \item Android invoca a la tarea \textit{Upload}
            \item Se comprobará si hay datos de estrés pendientes de enviar.
            \item Si hay datos pendientes de est
            \item Bifurcación
            \begin{enumerate}
                \item Primario
                \item Alternativo
            \end{enumerate}
        \end{enumerate}

Diagrama ER

\begin{sidewaysfigure}[h]
    \centering
    \includegraphics[width=1\textwidth]{figures/bd/ER simple.png}
    \caption[Diagrama ER]{Diagrama ER. Elaboración propia}
    \label{figure:disenio:diagrama_er}
\end{sidewaysfigure}

Diagrama colección datos usuario (\textit{user data})

\begin{figure}[h]
    \centering
    \includegraphics[width=0.75\textwidth]{figures/bd/Servidor user data.png}
    \caption[Diagrama colección datos usuario]{Diagrama colección datos usuario. Elaboración propia}
    \label{figure:disenio:diagrama_user_data}
\end{figure}

Diagrama colección cuestionarios diarios (\textit{daily questionnaires})

\begin{figure}[h]
    \centering
    \includegraphics[width=0.33\textwidth]{figures/bd/Servidor daily questionnaires.png}
    \caption[Diagrama colección cuestionarios diarios]{Diagrama colección cuestionarios diarios. Elaboración propia}
    \label{figure:disenio:diagrama_daily}
\end{figure}

Diagrama colección cuestionarios puntuales(\textit{one off questionnaires})

\begin{figure}[h]
    \centering
    \includegraphics[width=0.33\textwidth]{figures/bd/Servidor one off questionnaires.png}
    \caption[Diagrama colección cuestionarios puntuales]{Diagrama colección cuestionarios puntuales. Elaboración propia}
    \label{figure:disenio:diagrama_one_off}
\end{figure}
\chapter{[En curso] Desarrollo del sistema}
    \label{chapter:desarrollo}

    \chapquote{¿Quién dijo miedo habiendo hospitales?}{Sabiduría popular de la ETSISI}
    
    \section{Setup del proyecto}

    \section{Implementación de la aplicación móvil}

    \section{Implementación de la API del servidor}

        \subsection{Subida de datos de usuario}

        \subsection{Subida de los datos de los cuestionarios}

        \subsection{Datos de la comunidad}

    \subsection{Grafos de navegación}
    \begin{figure}[h]
        \centering
        \begin{tikzpicture}[
                > = stealth, % estilo de la cabeza de la flecha
                shorten > = 1pt, % para que la flecha no toque el nodo
                node distance = 3cm, % distancia entre nodos
            ]
    
            \tikzstyle{every state}=[
                draw = black,
                thick,
                fill = white,
            ]

            \node[rectangle,draw] (click) {\textit{El usuario entra en la app}};
            \node[state] [below of=click] (splash) {Splash};
            \node[state] [below of=splash] (bienvenida) {Bienvenida};
            \node[state] [below of=bienvenida] (permisos) {Permisos};
            \node[state] [below of=permisos] (inicio) {\textit{Inicio}};
            
            \path[->] (click) edge node {} (splash);
            \path[->] (splash) edge node {} (bienvenida);
            \path[->] (bienvenida) edge node {} (permisos);
            \path[->] (permisos) edge node {} (inicio);
    
        \end{tikzpicture}
        \caption{Grafo de navegación en el primer uso de la app}
        \label{figure:disenio:grafo_primer_uso}
    \end{figure}
    
    \begin{figure}[h]
        \centering
        \begin{tikzpicture}[
                font=\small, % fuente de letra pequeña
                > = stealth, % estilo de la cabeza de la flecha
                shorten > = 1pt, % para que la flecha no toque el nodo
                node distance = 3.3cm, % distancia entre nodos
            ]
    
            \tikzstyle{every state}=[
                draw = black,
                thick,
                fill = white,
            ]

            \node[rectangle,draw] (click) {E\textit{l usuario entra en la app}};
            \node[state] [below of=click] (splash) {Splash};
            \node[state] [below of=splash] (inicio) {Inicio};
    
            \node[state] [below left of=inicio] (historial) {Historial};
            \node[state] [below right of=inicio] (comunidad) {Comunidad};
            \node[state, align=center] [right of=inicio] (incompletos) {Cuestionarios\\incompletos};
            \node[state] [below right of=historial] (ajustes) {Ajustes};
            \node[state] [left of=inicio] (consejo) {Consejo};
            \node[state] [left of=historial] (medida) {Medida};
        
            \node[state, align=center] [above right of=incompletos] (diaria) {\textit{Ronda}\\\textit{diaria}};
            \node[state, align=center] [below right of=incompletos] (puntual) {\textit{Ronda}\\\textit{puntual}};
            
            \node[state] [below of=ajustes] (bienvenida) {Bienvenida};
            \node[state] [left of=bienvenida] (mis_datos) {Mis datos};
            \node[state] [left of=mis_datos] (privacidad) {Privacidad};
            \node[state] [right of=bienvenida] (acerca) {Acerca de};
            \node[state] [right of=acerca] (créditos) {Créditos};

            \path[->] (click) edge node {} (splash);
            \path[->] (splash) edge node {} (inicio);
            
            \path[<->] (inicio) edge node {} (historial);
            \path[<->] (inicio) edge node {} (comunidad);
            \path[<->] (inicio) edge node {} (ajustes);
            \path[<->] (inicio) edge node {} (medida);
            \path[<->] (inicio) edge node {} (consejo);
            \path[<->, dashed] (inicio) edge node {} (incompletos);
    
            \path[<->] (historial) edge node {} (medida);
            \path[<->] (historial) edge node {} (comunidad);
            \path[<->] (historial) edge node {} (ajustes);
            
            \path[<->] (ajustes) edge node {} (comunidad);
            \path[<->] (ajustes) edge node {} (bienvenida);
            \path[<->] (ajustes) edge node {} (mis_datos);
            \path[<->] (ajustes) edge node {} (privacidad);
            \path[<->] (ajustes) edge node {} (acerca);
            \path[<->] (ajustes) edge node {} (créditos);
            
            \path[->] (incompletos) edge node {} (diaria);
            \path[->] (incompletos) edge node {} (puntual);
    
        \end{tikzpicture}
        \caption{Grafo de navegación principal de la app}
        \label{figure:disenio:grafo_principal}
    \end{figure}

    \begin{figure}[h]
        \centering
        \begin{tikzpicture}[
                font=\small, % fuente de letra pequeña
                > = stealth, % estilo de la cabeza de la flecha
                shorten > = 1pt, % para que la flecha no toque el nodo
                node distance = 3cm, % distancia entre nodos
            ]

            \tikzstyle{every state}=[
                draw = black,
                thick,
                fill = white,
            ]

            \node[state, align=center] (diaria) {Ronda\\diaria}; 

            \node[rectangle,draw, align=center] [above left of=diaria] (notificacion) {\textit{El usuario pulsa}\\\textit{en la notificación}};
            \node[rectangle,draw, align=center] [above right of=diaria] (incompletos) {\textit{El usuario pulsa en un} \\ \textit{cuestionario incompleto}};
            
            \node[state] [right of=diaria] (inicio) {\textit{Inicio}}; 
            
            \node[state, align=center] [below of=diaria] (suicidio_diario) {Suicidio\\diario};
            \node[state, align=center] [left of=suicidio_diario] (depresion_diario) {Depresión\\diario};
            \node[state, align=center] [left of=depresion_diario] (estres_diario) {Estrés\\diario};
            \node[state, align=center] [right of=suicidio_diario] (soledad_diario) {Soledad\\diario};
            \node[state, align=center] [right of=soledad_diario] (contraste_diario) {Contraste\\ diario};
    
            \node[state] [below of=suicidio_diario] (consejo) {Consejo};

            \path[->] (notificacion) edge node {} (diaria);
            \path[->] (incompletos) edge node {} (diaria);
            \path[->] (diaria) edge node {} (inicio);
            \path[<->,dashed] (diaria) edge node {} (estres_diario);
            \path[<->,dashed] (diaria) edge node {} (depresion_diario);
            \path[<->,dashed] (diaria) edge node {} (soledad_diario);
            \path[<->,dashed] (diaria) edge node {} (suicidio_diario);
            \path[<->,dashed] (diaria) edge node {} (contraste_diario);
    
            
            \path[<->] (consejo) edge node {} (estres_diario);
            \path[<->] (consejo) edge node {} (depresion_diario);
            \path[<->] (consejo) edge node {} (soledad_diario);
            \path[<->] (consejo) edge node {} (suicidio_diario);

        \end{tikzpicture}
        \caption{Grafo de navegación en los cuestionarios diarios de la app}
        \label{figure:disenio:grafo_diario}
    \end{figure}

    \begin{figure}[h]
        \centering
        \begin{tikzpicture}[
                font=\small, % fuente de letra pequeña
                > = stealth, % estilo de la cabeza de la flecha
                shorten > = 1pt, % para que la flecha no toque el nodo
                node distance = 3.5cm, % distancia entre nodos
            ]

            \tikzstyle{every state}=[
                draw = black,
                thick,
                fill = white,
            ]
            
            \node[state, align=center] (puntual) {Ronda\\puntual}; 
            
            \node[rectangle,draw, align=center] [above left of=diaria] (notificacion) {\textit{El usuario pulsa}\\\textit{en la notificación}};
            \node[rectangle,draw, align=center] [above right of=diaria] (incompletos) {\textit{El usuario pulsa en un} \\ \textit{cuestionario incompleto}};
            
            \node[state] [right of=diaria] (inicio) {\textit{Inicio}}; 
            
            \node[state, align=center] [below of=puntual] (depresion_puntual) {Depresión\\puntual};
            \node[state, align=center] [left of=depresion_puntual] (estres_puntual) {Estrés\\puntual};
            \node[state, align=center] [right of=depresion_puntual] (soledad_puntual) {Soledad\\puntual};
    
            \node[state] [below of=depresion_puntual] (consejo) {Consejo};
    
            \path[->] (puntual) edge node {} (inicio);
            \path[->] (notificacion) edge node {} (puntual);
            \path[->] (incompletos) edge node {} (puntual);
            
            \path[<->,dashed] (puntual) edge node {} (estres_puntual);
            \path[<->,dashed] (puntual) edge node {} (depresion_puntual);
            \path[<->,dashed] (puntual) edge node {} (soledad_puntual);
    
            \path[<->] (consejo) edge node {} (estres_puntual);
            \path[<->] (consejo) edge node {} (depresion_puntual);
            \path[<->] (consejo) edge node {} (soledad_puntual);

        \end{tikzpicture}
        \caption{Grafo de navegación en los cuestionarios puntuales de la app}
        \label{figure:disenio:grafo_puntuales}
    \end{figure}
\chapter{Pruebas del sistema}
\label{chapter:pruebas}
    \section{Aplicación móvil}
    \section{API del servidor}
        \subsection{Pruebas unitarias}
        \subsection{Pruebas de integración}

\chapter{[En curso] Resultados}
\label{chapter:resultados}

\chapquote{¡Los números Mason! ¿Qué significan?}{Jason Hudson}

\section{Resultados del caso de estudio}

    En este apartado se muestran algunos ejemplos de los utilización y resultados que podrían obtenerse.

\section{Problemas encontrados}

    A lo largo de esta sección serán detallados los problemas encontrados durante este \gls{tfm}.
    
    \begin{enumerate}
        \item \textbf{Falta de experiencia del autor en la \gls{iss}}: el autor es plenamente consciente de la dificultad de la elaboración de los planes de estudios para cubrir todas las áreas relevantes de conocimiento, pero quiere poner de manifiesto que la \gls{iss} solo se ha cubierto mediante dos asignaturas, una de ellas en el \textit{Doble Grado en Ingeniería de Computadores y Tecnologías para la Sociedad de la Información} y la otra en este máster.
        
        Para interiorizar plenamente los conceptos de la \gls{iss}, el autor considera que es necesaria cierta experiencia práctica; por lo que se introdujo como un objetivo fundamental de este \gls{tfm} desde el inicio. Esta casuística ha ralentizado el ritmo de realización de este proyecto, si bien ha supuesto un enorme aprendizaje para el autor.
        
        \item \textbf{Falta de experiencia del autor en el desarrollo de aplicaciones móviles}: se trata de un problema muy similar al anterior. Si bien el autor había desarrollado dos aplicaciones pequeñas en sendas asignaturas, se ha tratado del primer desarrollo a mayor escala; especialmente en cuanto al diseño e implementación de interfaces gráficas.
    
        Esta circunstancia ha supuesto numerosos retos para el autor, tales como el aprendizaje de nuevos lenguajes de programación, el uso de patrones de diseño como inyección de dependencias o factoría o la realización de diseños \gls{responsive}. Como en el caso anterior, ha frenado el ritmo de desarrollo, pero a cambio ha supuesto una incalculable experiencia que ha permitido el aprendizaje de numerosos conceptos y/o técnicas.
    
        \item \textbf{Agresivas optimizaciones de batería en algunos dispositivos}: este problema fue descubierto realizando pruebas en dispositivos reales al componente \textit{Work Manager}. En dichas comprobaciones se constató en que algunos dispositivos, como el \textit{Pocophone F1} del autor, el sistema operativo nunca ejecutaba en segundo plano las tareas planificadas, realizándose todas ellas de golpe únicamente cuando el usuario abría la aplicación.
    
        Dado que este comportamiento no está documentado oficialmente, una investigación fue realizada para tratar de descubrir el problema de fondo y buscar una solución. Al parecer se trata de un problema reportado desde al menos el año 2017 \cite{mathew_why_2017}, en el cual ciertos fabricantes de \glspl{smartphone}, tales como Xiaomi, Samsung, Oppo o Redmi, realizan modificaciones en sus versiones de Android que rompen con el cumplimiento de las \gls{api} \cite{android_public_tracker_chinese_nodate} \cite{android_public_tracker_request_nodate}, levantando polémica ya que según las políticas de Android no está permitido \cite{cylon999_workmanager_2018}.
    
        Con la finalidad de ahorrar batería, en estas versiones el sistema cierra forzosamente ciertas aplicaciones, cancelando todas las tareas en segundo plano \cite{manohar_is_2020}. Las aplicaciones más utilizadas por los usuarios, como \textit{WhatsApp}, parecen estar en una \textit{lista blanca}, lo que genera un enorme agravio comparativo; ya que el usuario debe desactivar manualmente esta optimización de batería \cite{shukla_work_2021}.
    
        Sobre este problema se pueden encontrar iniciativas como \textit{Don't kill my app} \cite{dont_kill_my_app_our_nodate} que están arrojando visibilidad a este fenómeno, a través de charlas y de la evaluación de estas políticas según cada fabricante.
    
        \item \textbf{Escaso soporte oficial de los fabricantes de \glspl{wearable} a \textit{Salud Conectada}}: desde el anuncio de \textit{Salud Conectada}, únicamente dos fabricantes soportan dicho \gls{framework}: Samsung y la propia Fitbit (propiedad de la propia Google desde 2019). 
        
        En el caso de Samsung, se realizaron pruebas con una \textit{Samsung Galaxy Fit 2}, descubriéndose que la pulsera únicamente era capaz de escribir en \textit{Salud Conectada} datos de sueño, como se puede ver en la Figura \ref{figure:problemas:samsung_sueño}.
    
        \begin{figure}[h]
            \centering
            \includegraphics[width=0.45\textwidth]{figures/escritura_solo_sueño.jpg}
            \caption{Registro de lectura/escritura con una \textit{Samsung Galaxy Fit 2}}
            \label{figure:problemas:samsung_sueño}
        \end{figure}
    
        Estos problemas de escritura fueron reportados por la comunidad en los casos de pulsaciones \cite{jauwaadshams_heart_2023}, pasos \cite{ste1603137554_samsung_2022} o directamente, no funcionando en Android 14 \cite{muzzas_samsung_2023}.
    
        Para resolver estas cuestiones, se apostó por utilizar únicamente pulseras Fitbit para el desarrollo, pero en cualquier caso estos problemas tienen un impacto directo en el uso de otras pulseras por parte de los usuarios.
        
        \item \textbf{Inestabilidad del ecosistema \textit{Salud Conectada}}. A fecha de julio de 2024, el \gls{sdk} ofrecido por Google continúa aún en fase \textit{alpha}, y durante la realización de este \gls{tfm} se han producido numerosos cambios en dicho componente. Las \gls{api} ofrecidas para los desarrolladores no se han mantenido estables, por lo que ha supuesto un consumo adicional de recursos mantener la aplicación actualizada.
    
        \item \textbf{Falta de madurez y adopción de \textit{Jetpack Compose} y \textit{Material Design 3}}: se ha tratado de un problema sistemático y persistente a lo largo de todo el proyecto, que ha supuesto un sinfín de retos y de dificultades técnicas.
    
        En el caso de \textit{Jetpack Compose} lleva aproximadamente tres años en el mercado; aún presenta deficiencias a resolver. El caso más paradigmático encontrado en este proyecto ha sido el procedimiento de paso de parámetros en la navegación entre ventanas, el cual ha tenido que \textit{resolver} la comunidad mediante librerías \cite{costa_overview_nodate}.  
        
        Por otra parte, la aún escasa adopción de este \gls{framework} ha supuesto una clara desventaja en el acceso a documentación y recursos actualizados; provocando un incremento de dificultad más que notable de tareas tan habituales como la automatización de pruebas.
    
        Un caso similar ha sucedido con \textit{Material Design 3}, con la salvedad que su primera versión estable fue lanzada en octubre de 2022; pocas semanas después del comienzo de este proyecto. La ausencia de adopción masiva es comprensible, pero no la falta de madurez de la misma. 
        
        Un caso paradigmático es el componente \textit{date picker}, utilizado para que el usuario pueda seleccionar fechas. Siete mes después de la primera versión estable se introdujo dicho componente en un estado experimental o inestable, el cual sigue sin haber abandonado casi dos años después de su especificación, como se puede ver en la Figura \ref{figure:problemas:date_picker}. Para resolver estos problemas, nuevamente la comunidad ha tomado la alternativa y ha propuesto implementaciones de acuerdo a esta especificación. \cite{keppeler_sheets-compose-dialogs_nodate}.
    
        \begin{figure}[h]
            \centering
            \includegraphics[width=0.75\textwidth]{figures/date_picker_status.JPG}
            \caption[Información oficial del elemento \textit{date picker}]
            {Información oficial del elemento \textit{date picker}. Imagen extraída de \cite{material_design_3_date_nodate}.}
            \label{figure:problemas:date_picker}
        \end{figure}
    
        \item \textbf{Dificultades notables para realizar pruebas}: los siguientes factores han complicado exponencialmente el proceso de pruebas, especialmente de cara a la automatización:
    
        \begin{itemize}
            \item El uso de \textit{Salud Conectada}, el cual aparece como una aplicación externa en versiones anteriores a la 14. Google no ha provisto aún una estrategia para realizar pruebas automáticas en aplicaciones que la integren.
            \item Ausencia de procedimientos oficiales o componentes para probar módulos que interactúan con el sistema operativo, tales como las notificaciones o la planificación de tareas.
            \item La implementación de los mecanismos de cambios de idioma, paletas de colores o modos.
        \end{itemize}
    
        \item \textbf{Ausencia de soporte para numerosas cuestiones}: las siguientes características implementadas en este proyecto disponen de un soporte muy limitado o inexistente, teniendo que acudir a librerías de terceros o afrontando numerosas dificultades técnicas:
            \begin{itemize}
                \item Cambio de idioma manual por parte del usuario \cite{berezanskyi_how_2020}.
                \item Menú y diálogos de ajustes \cite{borras_paronella_alormacompose-settings_nodate}.
                \item Diseño de nuevos componentes gráficos, como el elemento de progreso circular.
                \item Elaboración de gráficas \gls{responsive} para móviles en posición apaisada.
                \item Modificación del eje X de las gráficas para mostrar el día de la semana y/o el día del año.
            \end{itemize}
        \item \textbf{Implementación de los cuestionarios}: debido a la naturaleza de los cuestionarios de la aplicación, fue particularmente complejo diseñar e implementar una arquitectura que permitiera implementar cuestionarios con diferentes tipos de preguntas (categóricas y numéricas), presencia o ausencia de mecanismos de puntuación, terminación temprana en el caso del cuestionario de suicidio, etc.
    \end{enumerate}
\chapter{[Por revisar] Impacto social y medioambiental}
\label{chapter:aspectos}

\chapquote{Si tienes tanto miedo al fracaso, nunca tendrás éxito.}{Mario Andretti}

En este capítulo se recogen los beneficios que la implantación del proyecto desarrollado podría generar tanto a nivel social como medioambiental.

\section{Aspectos éticos, sociales y económicos}
    El sistema propuesto tiene un impacto considerable en estas cuestiones, al buscar contribuir en la mejora de la salud mental de las personas. En primer lugar, se puede observar el impacto de este proyecto en base a los \gls{ods} respecto a los siguientes objetivos globales:

    \begin{itemize}
        \item \textbf{Salud y bienestar}: el sistema contribuye claramente a la visibilización y concienciación acerca de la salud mental, a través de la detección de trastornos en personas que posiblemente lo desconozcan y comunicando recomendaciones para evitar en la medida que dichos trastornos se agraven\footnote{Recordar una vez más que el sistema no pretende desplazar a los profesionales de la psicología, sino ser un apoyo para los mismos. En casos graves la solución pasa necesariamente por especialistas.}. No obstante, este tipo de iniciativas deben ir acompañadas de políticas públicas que permitan resolver cuestiones como las listas de espera o de la ausencia de profesionales, ya que un diagnóstico precoz de un trastorno grave es poco útil si no va acompañado de un tratamiento de calidad por parte de profesionales.
        \item \textbf{Educación de calidad}: al proporcionar consejos y pautas avaladas por profesionales de la psicología, los usuarios pueden acceder a información respaldada científicamente; la cual puede mejorar su calidad de vida. No obstante, como se mencionó en el punto anterior, estos consejos son una parte de la solución a estos problemas, pero no pueden resolverlos por sí solos.
        \item \textbf{Reducción de las desigualdades}: como se vió en la Sección \ref{sec:justificacion}, existen numerosas desigualdades en el acceso a la atención psicolópgica. En este ámbito, los requisitos técnicos del sistema son alcanzables por \glspl{smartphone} baratos, facilitando el acceso a la información psicológica independientemente del contexto socio-económico.
        \item \textbf{Producción y consumo responsable}: el sistema no requiere hardware específico, limitándose a hacer uso de los recursos que disponga el usuario. Asimismo, el sistema no marca restricciones adicionales a las ya impuestas por sus dependencias, evitando en general promover el consumo nuevos productos.
    \end{itemize}

    En cuanto a las cuestiones de privacidad, cabe destacar que el uso de componentes como \textit{Salud Conectada} y el uso de un servidor de la \gls{etsisi}. Estas circunstancias permiten mejorar la privacidad de los usuarios frente a otras soluciones como \textit{Google Fit}, ya que los datos únicamente están o bien en el dispositivo del usuario o anonimizados dentro de la Escuela. 
    
    Con esto se logra evitar el trasvase de información a servidores alojados en países fuera de la Unión Europea, donde las normativas de protección de datos son más laxas; como es el caso de los Estados Unidos. 

    Por otra parte, a nivel ético existen numerosas preocupaciones en el ámbito de las \gls{tic}, ya que las prácticas deshonestas expuestas en la Sección \ref{sec:estado_arte:apps} no se limitan a las aplicaciones de salud mental.
    
    Algunas plataformas digitales muy conocidas están diseñadas específicamente para explotar ciertos resortes psicológicos del ser humano. Redes sociales como \textit{Instagram} o \textit{TikTok} utilizan fenómenos como la gratificación instantánea o los mecanismos de recompensa \cite{noauthor_dopamina_2022} para espolear la liberación de dopamina\footnote{Se trata de un neurotransmisor asociado con funciones cerebrales, tales como la motivación, el aprendizaje y la recompensa, entre otras \cite{gil_que_2023}.}, pudiendo provocar desde la adicción a las mismas redes, hasta problemas de ansiedad o de autoestima cuando no se reciben dichas recompensas \cite{ina_impacto_2023}.
    
    Por otra parte, este mecanismo de gratificación instantánea también es explotado por algunos videojuegos a través de las \textit{lootboxes} (o \textit{cajas botín} en castellano), las cuales provocan grandes réditos económicos al provocar graves problemas de adicción, especialmente entre los menores. Debido a este contexto, las instituciones están tomando medidas para limitar técnicas como el \textit{scroll infinito} \cite{alconchel_prohibir_2023} o de limitar el acceso a las \textit{lootboxes} \cite{ministerio_de_derechos_sociales_consumo_y_agenda_2030_ministerio_2024} \cite{garcia_espanda_2024}. 
    
    En este marco cabe recalcar la apuesta firme de este proyecto por la psicología que permita ayudar a las personas a afrontar sus problemas de salud mental, sin fines ocultos o funcionalidades perversas. Como ya se reflejó a lo largo de este documento, la transparencia y la privacidad de los usuarios son baluartes fundamentales de esta iniciativa.

\section{Contexto medioambiental}

    Antes de comenzar con esta sección, cabe destacar que este proyecto no está diseñado para interactuar directamente con el medioambiente, por lo que el impacto del mismo en el entorno se puede considerar como indirecto.

    Al tratarse de un sistema que busca ante todo un diagnóstico precoz de las enfermedades de salud mental, existen potenciales beneficios en esa línea. De finalizarse este prototipo e implementarse en la comunidad universitaria, podrían detectarse estos estos trastornos en etapas más tempranas, lo cual podría reducir la gravedad de los mismos, redundando en tratamientos médicos más cortos.

    Para lograr este cometido, el sistema necesita realizar una serie de tareas, las cuales tienen cierto impacto en el medio ambiente. Una de estas actuaciones es la extracción de datos de los \glspl{wearable}. El sistema propuesto únicamente consume, de existir, los datos ya leídos por el fabricante del dispositivo; los cuales se alojan únicamente en el dispositivo del usuario. Por tanto, el impacto de estas lecturas es muy pequeño, ya que no se obliga al usuario a comprar un \gls{wearable} ni se realiza comunicación redudante con el aparato.

    Por otra parte, en cuanto a la comunicación con el servidor, al utilizarse un servidor de la \gls{etsisi}, el envío de los datos a través de la red es relativamente ligero. Asimismo, en la implementación se realizaron una serie de optimizaciones para enviar únicamente los datos necesarios cada 8 horas; por lo que no se trata de un sistema que realice un elevado consumo de la red y de energía.

    No obstante, el principal escollo en cuanto al medio ambiente sería la adición de un modelo para mejorar sensiblemente la detección de los niveles de salud mental; que si bien está fuera de la implementación actual, su incoporación futura podría ser muy útil. 
    
    La popularización de modelos de Inteligencia Artificial, como \textit{ChatGPT}, ha provocado que hasta la \gls{iea} haya convocado una conferencia sobre este tema \cite{perez_demanda_2024}, ya que se estima que el consumo de energía de los centros de datos oscile entre el 3 y el 4\% del consumo global \cite{gijon_inteligencia_2024}; mientras que para 2027, el consumo de electricidad de estos sistemas a nivel mundial podría aumentar entre 85 y 134 TWh anuales, cantidad comparable al consumo anual de electricidad de países como los Países Bajos, Argentina y Suecia \cite{redaccion_inteligencia_2023}.
    
    En ese sentido, para paliar el potencial impacto medioambiental se podría continuar con el enfoque de localizar estos servidores en la \gls{etsisi}; y apostar por el uso de energías limpias, como la solar; para alinearse con los \gls{ods} y en particular, con el número 7: \textit{Energía asequible y no contaminante}. El objetivo de estos movimientos sería reducir el consumo energético en la medida de lo posible, y basar ese consumo en fuentes de energía que tengan un impacto lo más reducido posible en el medioambiente.
\chapter{Conclusiones}
\label{chapter:conclusiones}

\chapquote{El éxito es realmente la libertad de hacer lo que quieras.}{Cita atribuida a Magnus Walker}
%\input{chapters/licencia}

\printbibliography

% Si se quieren imprimir antes que los anexos
\printglossary

\appendix
\chapter{Cuestionarios para el seguimiento diario}
\label{chapter:cuestionarios_diarios}
    \section{Inicio del día}
        \subsection{Estrés}
            \begin{enumerate}
                \item Me siento nervioso/a
                \item Me siento angustiado/a
                \item Me siento activo/a
                \item Estoy preocupado/a
            \end{enumerate}
            La respuesta a cada pregunta es un número entero en la escala de 0 a 10.

        \subsection{Depresión}
            \begin{enumerate}
                \item Me siento triste
                \item Me siento vacío/a 
                \item Me siento apático/a
            \end{enumerate}
            La respuesta a cada pregunta es un número entero en la escala de 0 a 10.

        \subsection{Soledad}
            \begin{enumerate}
                \item Me siento solo/a
                \item Me siento incomprendido/a
                \item Me siento exclusivo/a
                \item Me siento poco ayudado/a
            \end{enumerate}
            La respuesta a cada pregunta es un número entero en la escala de 0 a 10.

        \subsection{Suicidio}
            \begin{enumerate}
                \item Tengo pensamientos de suicidio
                \item En los últimos días, ¿has pensado seriamente en suicidarte?
                \item ¿Existe alguna posibilidad de que pienses acabar con tu vida hoy o en los próximos días?
            \end{enumerate}
            Las posibles respuestas a cada pregunta es sí o no.

    \section{Final del día}
        \subsection{Estrés}
            \begin{enumerate}
                \item Me he sentido nervioso/a
                \item Me he sentido angustiado/a
                \item Me he sentido activo/a
                \item He estado preocupado/a
            \end{enumerate}
            La respuesta a cada pregunta es un número entero en la escala de 0 a 10.

        \subsection{Depresión}
            \begin{enumerate}
                \item Me he sentido triste
                \item Me he sentido vacío/a 
                \item Me he sentido apático/a
            \end{enumerate}
            La respuesta a cada pregunta es un número entero en la escala de 0 a 10.

        \subsection{Soledad}
            \begin{enumerate}
                \item Me he sentido solo/a
                \item Me he sentido incomprendido/a
                \item Me he sentido exclusivo/a
                \item Me he sentido poco ayudado/a
            \end{enumerate}
            La respuesta a cada pregunta es un número entero en la escala de 0 a 10.

        \subsection{Suicidio}
            \begin{enumerate}
                \item He tenido pensamientos de suicidio
                \item En el día de hoy, ¿has pensado seriamente en suicidarte?
                \item ¿Existe alguna posibilidad de que pienses acabar con tu vida hoy o en los próximos días?
            \end{enumerate}
            Las posibles respuestas a cada pregunta es sí o no.

        \subsection{Contraste}
            \begin{enumerate}
                \item ¿Has experimentado cambios en el apetito?
                
                Posibles respuestas:
                    \begin{itemize}
                        \item Excesivamente alto  
                        \item Adecuado
                        \item Excesivamente bajo
                    \end{itemize}

                \item ¿Con cuánta energía te has notado?
                
                Posibles respuestas:
                    \begin{itemize}
                        \item Alta
                        \item Moderada
                        \item Baja
                    \end{itemize}

                \item ¿Cuál ha sido tu nivel de descanso?
                
                Posibles respuestas:
                    \begin{itemize}
                        \item Satisfactorio  
                        \item Moderado
                        \item Insuficiente
                    \end{itemize}
                    
                \item Tu nivel de concentración ha sido…
                
                Posibles respuestas:
                    \begin{itemize}
                        \item Satisfactorio  
                        \item Adecuado
                        \item Insuficiente
                    \end{itemize}
                    
                \item ¿Cuál ha sido tu nivel de líbido?
                
                Posibles respuestas:
                    \begin{itemize}
                        \item Satisfactorio  
                        \item Adecuado
                        \item Insuficiente
                    \end{itemize}
                    
                \item ¿Cómo te has encontrado a nivel de dolor?
                
                Posibles respuestas:
                    \begin{itemize}
                        \item Sin dolor 
                        \item Dolor moderado
                        \item Dolor alto
                    \end{itemize}
                    
            \end{enumerate}
\chapter{Cuestionarios puntuales}
\label{chapter:cuestionarios_puntuales}
    \section{Estrés (PSS 10)}
    \label{cuestionarios:pss_10}
        \begin{enumerate}
            \item ¿Con qué frecuencia ha estado afectado por algo que ha ocurrido inesperadamente?
            \item ¿Con qué frecuencia se ha sentido incapaz de controlar las cosas importantes en su vida?
            \item ¿Con qué frecuencia se ha sentido nervioso o estresado?
            \item ¿Con qué frecuencia ha estado seguro sobre su capacidad para manejar sus problemas personales?
            \item ¿Con qué frecuencia ha sentido que las cosas le van bien?
            \item ¿Con qué frecuencia ha sentido que no podía afrontar todas las cosas que tenía que hacer?
            \item ¿Con qué frecuencia ha podido controlar las dificultades de su vida?
            \item ¿Con que frecuencia se ha sentido que tenia todo bajo control?
            \item ¿Con qué frecuencia ha estado enfadado porque las cosas que le han ocurrido estaban fuera de su control?
            \item ¿Con qué frecuencia ha sentido que las dificultades se acumulan tanto que no puede superarlas?
        \end{enumerate}

        Las preguntas 4, 5, 7 y 8 siguen el esquema de puntuación invertida.
        
        Por otra parte, la respuesta a cada pregunta es una de las siguientes opciones:
        \begin{itemize}
            \item Nunca
                \begin{itemize}
                    \item Puntuación por defecto: 0
                    \item Puntuación en las preguntas de puntuación invertida: 4
                \end{itemize}
            \item Casi nunca
                \begin{itemize}
                    \item Puntuación por defecto: 1
                    \item Puntuación en las preguntas de puntuación invertida: 3
                \end{itemize}
            \item De vez en cuando
                \begin{itemize}
                    \item Puntuación por defecto: 2
                    \item Puntuación en las preguntas de puntuación invertida: 2
                \end{itemize}
            \item A menudo
                \begin{itemize}
                    \item Puntuación por defecto: 3
                    \item Puntuación en las preguntas de puntuación invertida: 1
                \end{itemize}
            \item Muy a menudo
                \begin{itemize}
                    \item Puntuación por defecto: 4
                    \item Puntuación en las preguntas de puntuación invertida: 0
                \end{itemize}
        \end{itemize}

        Por último, en cuanto a las puntuaciones, se disponen de los siguientes valores:
        \begin{itemize}
            \item Puntuación mínima: 0
            \item Puntuación máxima: 40
            \item Posibles niveles:
                \begin{itemize}
                    \item Nivel bajo entre 0 y 43 puntos.
                    \item Nivel medio entre 14 y 26 puntos.
                    \item Nivel alto entre 27 y 40 puntos.
                \end{itemize}
        \end{itemize} 

    \section{Depresión (PHQ 9)}
    \label{cuestionarios:phq_9}
        \begin{enumerate}
            \item Poco interés o placer en hacer cosas
            \item Se ha sentido decaído(a), deprimido(a) o sin esperanzas 
            \item Ha tenido dificultad para quedarse o permanecer dormido(a), o ha dormido demasiado 
            \item Se ha sentido cansado(a) o con poca energía 
            \item Sin apetito o ha comido en exceso
            \item Se ha sentido mal con usted mismo(a) – o que es un fracaso o que ha quedado mal con usted mismo(a) o con su familia
            \item Ha tenido dificultad para concentrarse en ciertas actividades, tales como leer el periódico o ver la televisión
            \item ¿Se ha movido o hablado tan lento que otras personas podrían haberlo notado? o lo contrario – muy inquieto(a) o agitado(a) que ha estado moviéndose mucho más de lo normal
            \item Pensamientos de que estaría mejor muerto(a) o de lastimarse de alguna manera
        \end{enumerate}
        
        Por otra parte, la respuesta a cada pregunta es una de las siguientes opciones:
        \begin{itemize}
            \item Nunca
                \begin{itemize}
                    \item Puntuación: 0
                \end{itemize}
            \item Alguna vez
                \begin{itemize}
                    \item Puntuación: 1
                \end{itemize}
            \item De vez en cuando
                \begin{itemize}
                    \item Puntuación: 2
                \end{itemize}
            \item A menudo
                \begin{itemize}
                    \item Puntuación: 3
                \end{itemize}
        \end{itemize}

        Por último, en cuanto a las puntuaciones, se disponen de los siguientes valores:
        \begin{itemize}
            \item Puntuación mínima: 0
            \item Puntuación máxima: 27
            \item Posibles niveles:
                \begin{itemize}
                    \item Nivel mínimo entre 0 y 4 puntos.
                    \item Nivel leve entre 5 y 9 puntos.
                    \item Nivel moderado entre 10 y 14 puntos.
                    \item Nivel moderadamente severo entre 15 y 19 puntos.
                    \item Nivel severo entre 20 y 27 puntos.
                \end{itemize}
        \end{itemize} 

    \section{Soledad (UCLA 20)}
        \label{cuestionarios:ucla_20}
        \begin{enumerate}
            \item Sintonizo (me llevo bien) con la gente que me rodea
            \item Me falta compañía
            \item No tengo a nadie con quien yo pueda contar
            \item Me siento solo/a
            \item Me siento parte de un grupo de amigos/as
            \item Tengo muchas cosas en común con la gente que me rodea
            \item No tengo confianza con nadie
            \item Mis intereses e ideas no son compartidos por las personas que me rodean
            \item Soy una persona abierta (extrovertida)
            \item Me siento cercano/a de algunas personas
            \item Me siento excluido/a, olvidado/a por los demás
            \item Mis relaciones sociales son superficiales
            \item Pienso que realmente nadie me conoce bien
            \item Me siento aislado/a de los demás
            \item Puedo encontrar compañía cuando lo necesito
            \item Hay personas que realmente me comprenden
            \item Me siento infeliz de estar tan aislado/a
            \item La gente está a mi alrededor pero no siento que esté conmigo
            \item Hay personas con las que puedo charlar y comunicarme
            \item Hay personas a las que puedo recurrir
        \end{enumerate}

        Las preguntas 1, 5, 6, 9, 10, 15, 16, 19 y 20 siguen el esquema de puntuación invertida.
        
        Por otra parte, la respuesta a cada pregunta es una de las siguientes opciones:
        \begin{itemize}
            \item Nunca
                \begin{itemize}
                    \item Puntuación por defecto: 1
                    \item Puntuación en las preguntas de puntuación invertida: 4
                \end{itemize}
            \item Alguna vez
                \begin{itemize}
                    \item Puntuación por defecto: 2
                    \item Puntuación en las preguntas de puntuación invertida: 3
                \end{itemize}
            \item De vez en cuando
                \begin{itemize}
                    \item Puntuación por defecto: 3
                    \item Puntuación en las preguntas de puntuación invertida: 2
                \end{itemize}
            \item A menudo
                \begin{itemize}
                    \item Puntuación por defecto: 4
                    \item Puntuación en las preguntas de puntuación invertida: 1
                \end{itemize}
        \end{itemize}

        Por último, en cuanto a las puntuaciones, se disponen de los siguientes valores:
        \begin{itemize}
            \item Puntuación mínima: 20
            \item Puntuación máxima: 80
            \item Posibles niveles:
                \begin{itemize}
                    \item Nivel bajo entre 20 y 40 puntos.
                    \item Nivel medio entre 41 y 60 puntos.
                    \item Nivel alto entre 61 y 80 puntos.
                \end{itemize}
        \end{itemize} 
\chapter{Recomendaciones}
\label{chapter:recomendaciones}

    \section{Estrés}
        \subsection{Bajo}
            Estupendo, sigue así. 
        \subsection{Moderado}
            \subsubsection{Pauta 1}
                Hemos percibido que estás experimentando niveles moderados de ansiedad o estrés. 
                Por ello, te recomendamos que planifiques un espacio en el día de hoy para hacer ejercicio físico.

                El ejercicio físico puede disminuir el estrés por varias razones:
                \begin{itemize}
                    \item Liberación de endorfinas: Durante el ejercicio, el cuerpo libera endorfinas, que son hormonas que  actúan como analgésicos naturales y generan sensaciones de bienestar. 
                    \item Reducción de la hormona del estrés: El ejercicio regular puede disminuir los niveles de cortisol, la hormona del estrés.
                    \item Mejora del sueño: El ejercicio regular puede promover un sueño más profundo y reparador.
                    \item Distracción y enfoque: Participar en actividades físicas puede distraer la mente de las preocupaciones y tensiones diarias. Cuando te concentras en el ejercicio, tu mente se enfoca en la actividad física en lugar de en los problemas, lo que puede ayudar a reducir el estrés y proporcionar un descanso mental.
                    \item Aumento de la confianza y la autoestima: El ejercicio regular puede ayudar a mejorar la confianza y la autoestima. Al establecer metas y lograr objetivos en el ámbito del ejercicio, puedes desarrollar una mayor sensación de logro y fortaleza personal. Esto puede ayudar a reducir el estrés al 
                    proporcionar una sensación de control y empoderamiento sobre tu vida.
                \end{itemize}

            \subsubsection{Pauta 2}
                En caso de que no dispongas de mucho tiempo, te proponemos una serie de alternativas. 

                \begin{itemize}
                    \item Trata de buscar un momento para ti, libre de estímulos estresantes. Puedes salir a dar un pequeño paseo, darte una ducha relajante, poner música y centrarte en escucharla durante unos minutos, practica unos estiramientos corporales… El objetivo es rebajar de forma rápida los niveles de ansiedad para poder retomar las tareas desde un estado emocional más adecuado. 
                    \item Trata de eliminar algunos estímulos que puedan estar aumentando tu ansiedad: apaga el móvil a partir de determinada hora en la noche para tener unas horas libres de notificaciones antes de dormir, ponte unos cascos con música relajante para no escuchar el ruido de alrededor, si tienes pendiente tomar una decisión o discutir algo con alguien, aplázalo durante unas horas o días y permítete posponer los pensamientos al respecto, etc. El objetivo es eliminar los estímulos que están produciendo estrés para poder rebajar los niveles de ansiedad y así enfrentarnos de forma más adecuada a nuestros problemas o dificultades. 
                \end{itemize}
        \subsection{Alto}
            Te proponemos que realices un ejercicio de respiración abdominal. 

            El objetivo de esta técnica es regular la respiración y, en consecuencia, disminuir la respuesta de activación fisiológica y la sensación de ansiedad. 
            
            Para ello, trata de llevar el aire hasta tu abdomen en cada inspiración para llenar tus pulmones en profundidad. Visualmente, deberías observar cómo tu tripa se hincha al llenarse de aire. Al expulsar el aire durante la espiración, el abdomen debería retornar a su posición habitual. Evita mover el pecho, los hombros o las clavículas, pues esto indica que el aire está llegando únicamente a la parte superior de los pulmones.
            
            Inhala durante la ascensión de la curva y exhala durante el descenso. Trata de no hacerlo de forma demasiado profunda. Puedes repetirte mentalmente una palabra como calma o relax, puedes imaginar que estás en un lugar tranquilo, o centrar tu atención en cómo el aire entra y sale y cómo la tensión se escapa con cada exhalación. 
    
    \section{Depresión}
        \subsection{Baja}
            Estupendo, sigue así. 
        \subsection{Moderada}

            Te proponemos que incluyas en tu  día de hoy alguna actividad agradable o placentera. 
            
            Es posible que sientas que no tienes ganas o energía para hacerlas o, incluso, que aunque las hagas no lo disfrutarás. No obstante, es importante que entiendas que “las ganas se hacen”. 
            
            Esto significa que cuando nuestro estado de ánimo está un poquito bajo, si esperamos a experimentar ganas para hacer las cosas, probablemente nunca las hagamos. Esto a su vez hará que nuestro estado de ánimo disminuya todavía más, y entremos en un circulo vicioso en el que no haremos nada porque no tenemos ganas porque estamos tristes, y como no hacemos nada estaremos aún más tristes. 
            
            Para no caer en esta problemática te sugerimos que realices alguna actividad agradable, que no sea muy costosa y que te permita sentirte mejor. Pueden ser actividades que hagas tú solo/a o acompañado/a. 
            
            Aquí te dejamos algunas sugerencias. 
            \begin{itemize}
                \item Dar un paseo al aire libre. 
                \item Ir a comprar al supermercado y cocinar una receta que te guste. 
                \item Ver una serie o película que te apetezca.
                \item Quedar con un amigo/a tomar algo o pasear.
                \item Leer un libro, escuchar música, dibujar… 
            \end{itemize}

        \subsection{Alta}
            Te sugerimos que busques apoyo en las personas de tu alrededor.

            Es muy beneficioso que puedas expresar cómo te estás sintiendo a otras personas. El mero hecho de contarlo supondrá un desahogo emocional que te hará sentirte mejor.
            
            Además, las personas que te escuchen podrán comprender por lo que estás pasando y mostrar su empatía y apoyo. Es posible también que puedan tratar de ayudarte u ofrecerte consejos.
            
            Para que todo vaya bien, trata de elegir a la persona adecuada en el momento adecuado. Busca a una persona que se encuentre bien, que no esté muy estresada u ocupada, que te haya mostrado su afecto en alguna ocasión…
            
            Si necesitas desahogarte con frecuencia, trata de hacerlo con diferentes personas y no focalizarte solo en una, ya que prestar apoyo emocional en ocasiones puede resultar algo cansado.
           
            Por último, ten cuidado de no caer en la queja: comunica tus emociones tratando de buscar soluciones y formas de sentirte mejor, en lugar de anclarte en el problema que ha sucedido.

    \section{Soledad}
        \subsection{Baja}
            Estupendo, sigue así. 
        \subsection{Moderada}
            Te sugerimos que busques apoyo en las personas de tu alrededor.

            Es posible que consideres que no tienes a nadie con quién hablar. Sin embargo, si lo intentas, seguro que puedes encontrar personas deseosas de conversar contigo.
            
            Busca a tu alrededor: un vecino con quien hayas tenido contacto, el camarero del bar o el restaurante al que vas en ocasiones, un compañero de trabajo…Puedes tratar de entablar una conversación preguntándoles cómo se encuentran y comentando cosas sobre el ambiente (el tiempo, los precios, los horarios, etc.).
            
            Después, puedes intentar contar alguna anécdota o experiencia personal reciente. Por ejemplo, un programa de TV que te haya gustado, algo curioso que hayas visto recientemente en redes sociales, un plan al que tengas ganas de asistir o que hayas disfrutado si ya lo has hecho, etc.
           
            Poco a poco, podrás progresar en la conversación y hablar más a menudo con estas personas.

        \subsection{Alta}
            Te recomendamos que llames a Cruz Roja Te Escucha (900 107 917).

            Cruz Roja te escucha es una iniciativa que trata de ofrecer acompañamiento y apoyo a personas que se encuentran en una situación de soledad no deseada. 
            
            Podrán facilitarte pautas para sentirte mejor y ofrecerte recursos en tu localidad que puedan servirte de apoyo. 
            
            Puedes contactar en el 900 107 917, de lunes a jueves laborales de 10h a 14h y de 16h a 20h (una hora menos en Canarias) y los viernes laborables de 10 a 14h (una hora menos en Canarias). La llamada es gratuita y confidencial.
    
    \section{Riesgo de suicidio}
        \subsection{Bajo}
            Estupendo, sigue así. 
        \subsection{Moderado}
            Recuerda que, si en algún momento tienes pensamientos relacionados con el suicidio, es importante que pidas ayuda.

        \subsection{Alto}
            Por lo que nos has contado, creemos que el riesgo de que puedas hacerte daño o de que acabes con tu vida es alto.

            Por favor, acude cuanto antes a un servicio de emergencias o llama a los teléfonos 112 o 024. 
            
            Allí encontrarás personas que podrán comprender cómo te sientes y ayudarte a sentirte mejor. 
            
            Recuerda que el suicidio es la única opción que no tiene vuelta atrás. Trata de agotar otras posibles soluciones y pide ayuda para conseguirlo.
%\chapter{Código fuente del proyecto}
\label{chapter:codigo}
% \input{appendices/escuelas-y-titulos}
% \input{appendices/ampliar}
% \input{appendices/paquetes}

\end{document}
