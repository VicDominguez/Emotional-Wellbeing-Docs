% Para impresión, poner twoside. Para PDF, poner oneside.
\documentclass[a4paper,oneside,12pt,titlepage]{report}

%%%%%%%%%%%%%%%%%%%%%%%%%%%%%%%%%%%%%%%%%%%%%%%%%%%
%% Importación de librerías                        
%%%%%%%%%%%%%%%%%%%%%%%%%%%%%%%%%%%%%%%%%%%%%%%%%%%

\usepackage{graphicx}
\usepackage[utf8]{inputenc}
% Para tener los elementos de texto en español. El flag noshorthands arregla el bug de Argument of \language@active@arg" has an extra }.
% http://algunostutoriales.blogspot.com/2014/03/arreglar-un-extrano-error-de-latex.html
\usepackage[english,spanish,es-noshorthands]{babel}
% Hipervinculos para todo lo útil
\usepackage{hyperref}
% Evitamos que rompa las palabras
\usepackage[htt]{hyphenat}
% Soporte para anexos
\usepackage{appendix}
% Paquetes para figuras
\usepackage{caption}
\usepackage{subcaption}
\usepackage{lscape}
% Insertar codigo en memoria
\usepackage{listings}
% Posicionamiento
\usepackage{float}
% Tablas
\usepackage{xltabular}
% Tablas de más de una página
\usepackage{longtable} 
% Metadatos
\usepackage{metainfo}
% Condicionales (para tener distintas versiones del documento)
\usepackage{etoolbox}
% Fracciones
\usepackage{xfrac}
% Márgenes reducidos
\usepackage[pdftex,scale={.8,.8}]{geometry}


% Bibiografia
\usepackage[style=ieee]{biblatex}
\bibliography{contenido/tfm.bib}

\usepackage{csquotes}
\usepackage[dvipsnames]{xcolor}
\usepackage{titlesec}


% Importamos ajustes y metadatos
%%%%%%%%%%%%%%%%%%%%%%%%%
%% Ajustes generales
%%%%%%%%%%%%%%%%%%%%%%%%%

% Ruta en la que pondremos las imágenes para el documento
\graphicspath{{imagenes/}}

% Encabezados y pies de página
% posiciones [] indica cada a la izquierda (par), {} a la derecha (impar). Si es una cara, se aplica derecha
\newpagestyle{long}
{
	\sethead[\sectiontitle][][\chaptername\ \thechapter:\ \chaptertitle]{\chaptername\ \thechapter:\ \chaptertitle}{}{\sectiontitle}
    \headrule % para que haya barra en el encabezado
	\setfoot[\thepage][][]{}{}{\thepage}
}
\newpagestyle{short}
{
	\setfoot[\thepage][][]{}{}{\thepage}
}

% Para que salga el número a un lado en la primera página de un capítulo
\patchcmd{\chapter}{plain}{short}{}{} 

%%%%%%%%%%%%%%%%%%%%%%%%%
%% Nuevos comandos
%%%%%%%%%%%%%%%%%%%%%%%%%

% Definimos el comando de página en blanco
\newcommand{\paginablanco}
{   
    \null
    \thispagestyle{empty}
    \newpage
    \addtocounter{page}{-1}
}

%mayor ancuhura en las fracciones
\newcommand{\ddfrac[2]}
{
    \frac{\displaystyle #1}{\displaystyle #2}
}

%comando escribir TFG
\newcommand{\tfg}{Trabajo Fin de Grado}

%%%%%%%%%%%%%%%%%%%%%%%%%%%
%% Redefinición de comandos
%%%%%%%%%%%%%%%%%%%%%%%%%%%

%Redifinimos el nombre que asigna babel spanish a las tablas (cuadro) a tabla
\renewcommand{\spanishtablename}{Tabla}
\renewcommand{\spanishlisttablename}{Índice de tablas}
\renewcommand{\spanishcontentsname}{Índice}
\renewcommand{\appendixname}{Anexos}
\renewcommand{\appendixtocname}{Anexos}
\renewcommand{\appendixpagename}{Anexos}

%mayor anchura en las tablas
\renewcommand{\arraystretch}{1.5}

%indice de codigo
\renewcommand*{\lstlistingname}{Código}
\renewcommand{\lstlistlistingname}{Índice de código}


%%%%%%%%%%%%%%%%%%%%%%%%%
%% Colores
%%%%%%%%%%%%%%%%%%%%%%%%%

\definecolor{darkgreen}{rgb}{0.0, 0.4, 0.0}
\definecolor{darkorange}{rgb}{1.0, 0.55, 0.0}
\definecolor{eclipseStrings}{RGB}{42,0.0,255}
\definecolor{eclipseKeywords}{RGB}{127,0,85}
\colorlet{numb}{magenta!60!black}

%%%%%%%%%%%%%%%%%%%%%%%%%
%% Ajustes del código
%%%%%%%%%%%%%%%%%%%%%%%%%

% Basado en el estilo de Javier Alonso (@javinator9889)
\lstset
{ 
  backgroundcolor=\color{white},        % Indica el color de fondo; necesita que se añada \usepackage{color} o \usepackage{xcolor}
  breakatwhitespace=false,              % Activarlo para que los saltos automáticos solo se apliquen en los espacios en blanco
  inputencoding=utf8,                   % Codificación de entrada
  captionpos=b,                         % Establece la posición de la leyenda del cuadro de código
  commentstyle=\color{darkgreen},       % Color de los comentarios
  extendedchars=true,                   % Permite utilizar caracteres extendidos no-ASCII; solo funciona para codificaciones de 8-bits; para UTF-8 no funciona. En xelatex necesita estar a true para que funcione.
  keepspaces=true,                      % Mantiene los espacios en el texto. Es útil para mantener la indentación del código(puede necesitar columns=flexible).
  columns=flexible,                     % Columnas a diferente ancho
  keywordstyle=\color{darkorange},      % Estilo de las palabras reservadas
  rulecolor=\color{black},              % Si no se activa, el color del marco puede cambiar en los saltos de línea entre textos que sea de otro color, por ejemplo, los comentarios, que están en verde en este ejemplo
  showspaces=false,                     % Si se activa, muestra los espacios con guiones bajos; sustituye a 'showstringspaces'
  showstringspaces=false,               % Subraya solamente los espacios que estén en una cadena de esto
  showtabs=false,                       % Muestra las tabulaciones que existan en cadenas de texto con guión bajo
  stringstyle=\color{darkgreen},        % Color de las cadenas de texto
  identifierstyle=\color{blue},         % Color del código en general
  tabsize=4,	                        % Establece el salto de las tabulaciones
  breaklines=true,                      % Romper lineas para que no se salga del folio
  basicstyle=\footnotesize\ttfamily,    % Fija el tamaño del tipo de letra utilizado para el código
  postbreak=\mbox{\textcolor{red}{$\hookrightarrow$}\space}, % Marca de la ruptura de la línea
  literate={á}{{\'a}}1 {é}{{\'e}}1 {í}{{\'i}}1 {ó}{{\'o}}1 {ú}{{\'u}}1 % Mapeo de acentos
  {Á}{{\'A}}1 {É}{{\'E}}1 {Í}{{\'I}}1 {Ó}{{\'O}}1 {Ú}{{\'U}}1
  {à}{{\`a}}1 {è}{{\`e}}1 {ì}{{\`i}}1 {ò}{{\`o}}1 {ù}{{\`u}}1
  {À}{{\`A}}1 {È}{{\'E}}1 {Ì}{{\`I}}1 {Ò}{{\`O}}1 {Ù}{{\`U}}1
  {ä}{{\"a}}1 {ë}{{\"e}}1 {ï}{{\"i}}1 {ö}{{\"o}}1 {ü}{{\"u}}1
  {Ä}{{\"A}}1 {Ë}{{\"E}}1 {Ï}{{\"I}}1 {Ö}{{\"O}}1 {Ü}{{\"U}}1
  {â}{{\^a}}1 {ê}{{\^e}}1 {î}{{\^i}}1 {ô}{{\^o}}1 {û}{{\^u}}1
  {Â}{{\^A}}1 {Ê}{{\^E}}1 {Î}{{\^I}}1 {Ô}{{\^O}}1 {Û}{{\^U}}1
  {œ}{{\oe}}1 {Œ}{{\OE}}1 {æ}{{\ae}}1 {Æ}{{\AE}}1 {ß}{{\ss}}1
  {ç}{{\c c}}1 {Ç}{{\c C}}1 {ø}{{\o}}1 {å}{{\r a}}1 {Å}{{\r A}}1
  {€}{{\EUR}}1 {£}{{\pounds}}1 {ñ}{{\~n}}1 {Ñ}{{\~N}}1
}

%%%%%%%%%%%%%%%%%%%%%%%%%%%%%
%% Lenguajes de código custom
%%%%%%%%%%%%%%%%%%%%%%%%%%%%%

% Basado en la de https://github.com/cansik/kotlin-latex-listing, sin sobrrescribir colores
\lstdefinelanguage{Kotlin}
{
  comment=[l]{//},
  emph={filter, first, firstOrNull, forEach, lazy, map, mapNotNull, println},
  emphstyle={\color{OrangeRed}},
  keywords={!in, !is, abstract, actual, annotation,  break, catch, class, const, constructor, continue, crossinline, do, dynamic, else, enum, expect, external, false, field, file, final, finally, for, fun, if, import,infix, inline, inner, interface, internal,lateinit, noinline, object, open, operator, override, package, param, private, property, protected, public, receiveris, reified, return, return@, sealed, setparam, suspend, tailrec, throw, true, try, typealias, typeof, val, var, vararg, when, where, while},
  morecomment=[s]{/*}{*/},
  morestring=[b]",
  morestring=[s]{"""*}{*"""},
  ndkeywords={?,@Deprecated, @JvmField, @JvmName, @JvmOverloads, @JvmStatic, @JvmSynthetic, Array, Byte, Double, Float, Int, Integer, Iterable, Long, Runnable, Short, String, Any, Unit, Nothing, as, as?, by, companion, data, delegate, get, in, init, is, it, lazy, let, null, out, this, set, super},
  ndkeywordstyle={\color{OrangeRed}\bfseries},
  sensitive=true,
}

\lstdefinelanguage{json}
{
    basicstyle=\small\ttfamily,
    commentstyle=\color{eclipseStrings}, % style of comment
    stringstyle=\color{eclipseKeywords}, % style of strings
    numbers=left,
    numberstyle=\scriptsize,
    stepnumber=1,
    numbersep=8pt,
    showstringspaces=false,
    frame=single,
    string=[s]{"}{"},
    comment=[l]{:\ "},
    morecomment=[l]{:"}
}

%%%%%%%%%%%%%%%%%%%%%%%%%%%%%%%%%%%
%% Estilo anexos (mismo para todos)
%%%%%%%%%%%%%%%%%%%%%%%%%%%%%%%%%%%

\lstdefinestyle{PythonAnexos}
{
    language=Python,
    numbers=left,                       % Posición de los números de línea (none, left, right).
    numbersep=5pt,                      % Distancia de los números de línea al código
    numberstyle=\small\color{gray},     % Estilo para los números de línea
    frame=single	                    % Añade un marco al código
}

\lstdefinestyle{KotlinAnexos}
{
    language=Kotlin,
    numbers=left,                    
    numbersep=5pt,                   
    numberstyle=\small\color{gray}, 
    frame=single
}

\lstdefinestyle{jsonAnexos}
{
    language=json,
    numbers=left,                   
    numbersep=5pt,                   
    numberstyle=\small\color{gray}, 
    frame=single
}
% Comandos de metadatos

\def\University{Universidad Politécnica de Madrid\\ Escuela Técnica Superior de Ingeniería de Sistemas Informáticos}
\def\Course{Máster Universitario en Software de Sistemas Distribuidos y Empotrados}
\def\Module{Proyecto Fin de Máster}
\def\ModuleMayus{PROYECTO FIN DE MÁSTER}

\def\Docent{Sandra Gómez Canaval, Gema Bello Orgaz }
\def\DocentMulti{Sandra Gómez Canaval \\ Gema Bello Orgaz }

\def\Title{Sistema para el Bienestar Emocional}
\def\Authors{Victor Manuel Domínguez Rivas}
\def\Shortname{V. Dominguez}
\def\EndDate{Julio de 2023}

%Metadatos PDF
\hypersetup
{
	pdfinfo=
	{
		Title={\Title},
		Author={\Authors},
		Subject={\Module - \Course}
	}
}


% Comenzamos el documento
\begin{document}
    
    % Hacemos la portada
    \begin{titlepage}
  % logo de la etsisi a la izquierda
  \begin{flushleft}
    \includegraphics[width=.4\linewidth]{imagenes/etsisi.png}
  \end{flushleft}
  
  % resto de la portada a mitad de folio
  \begin{center}
    \vspace*{1cm}

    \Huge
    \Course

    \vspace{3.5cm}
    \huge
    \textit{\Title}

    \vfill

    \Large
    \textsc{\ModuleMayus}

    \vspace{2.5cm}

    \Large
    \textit{\Authors}

    \vspace{2cm}

    \EndDate
  \end{center}

  \newpage
  
  \thispagestyle{empty}
  
  \begin{flushleft}
    \includegraphics[width=.4\linewidth]{imagenes/etsisi.png}
  \end{flushleft}

  \begin{center}
    \vspace*{1cm}

    \Huge
    \Course

    \vspace{3.5cm}
    \huge
    \textit{\Title}

    \vfill

    \Large
    \textsc{\ModuleMayus}

    \vspace{2.5cm}

    \Large
    \begin{flushright}
      \textit{Autor: \Authors} \\
      \vspace{0.5cm}
      \textit{Directoras: \Docent} \\
    \end{flushright}

    \vspace{2cm}

    \EndDate
  \end{center}

  \newpage
\end{titlepage}

    % Para comenzar la numeracion de paginas en numeros romanos
    \pagenumbering{Roman} 

    % Agradecimientos
    \thispagestyle{empty}
% Para alinear verticalmente al medio
\vspace*{\fill}

\begin{flushright}
    \textit{Tus agradecimientos aquí.}
\end{flushright}

\vspace*{\fill}

    % Hacemos los índices
    \tableofcontents
    \newpage
    \listoftables
    \newpage
    \listoffigures
    \newpage
    \lstlistoflistings
    \newpage

    % Después de los índices ya podemos poner el espacio entre párrafos igual al interlineado
    \setlength{\parskip}{\baselineskip} 
    \pagestyle{long}

    % contenido adicionales antes del inicio
    \chapter*{Glosario}
    \begin{itemize}
    \item \textbf{Android}: Sistema Operativo para dispositivos móviles basado en Linux. Está desarrollado por Google.
\end{itemize}
    \newpage
    \chapter*{Resumen}
    Resumen aqui.

\textbf{Palabras clave}: Palabras clave
    \chapter*{Abstract}
    Abstract here.

\textbf{Keywords}: Keywords
    
    % Ya tenemos las páginas de puro contenido, pasamos al formato normal
    
    \pagenumbering{arabic}

    %Importamos los capítulos
    \chapter{Introducción}
    \section{Motivación}
    
    \cite{DominguezRivas2020}
\section{Objetivos}
    
\section{Metodologia}
    
\section{Estructura del documento}
    
    \newpage

    \chapter{Contexto}
    \section{Estado del arte}
    
\section{Tecnologías utilizadas}
    
    \newpage

    \chapter{Marco teórico}
    \input{contenido/teorico.tex}
    \newpage

    \chapter{Especificación de requisitos}
    % Secciones de IEEE 830: https://www.fdi.ucm.es/profesor/gmendez/docs/is0809/ieee830.pdf
\section{Introducción}
    \subsection{Propósito}
        
    \subsection{Alcance}
        
\section{Descripción}
    \subsection{Perspectiva del Producto}
        
    \subsection{Funciones del Producto}
        
    \subsection{Características de los Usuarios}
        
    \subsection{Restricciones}
        
    \subsection{Suposiciones y Dependencias}
        
    \subsection{Requisitos Futuros}

\section{Requisitos Específicos}
    \subsection{Interfaces Externas}
        
    \subsection{Funciones}
        
    \subsection{Requisitos de Rendimiento}
        
    \subsection{Restricciones de Diseño}
        
    \subsection{Atributos del Sistema}
        
    \subsection{Otros Requisitos}
        
    \newpage

    \chapter{Diseño del sistema}
    \input{contenido/diseno.tex}
    \newpage

    \chapter{Implementación del sistema}
    \input{contenido/implementacion.tex}
    \newpage

    \chapter{Pruebas del sistema}
    \input{contenido/pruebas.tex}
    \newpage

    \chapter{Resultados}
    \input{contenido/resultados.tex}
    \newpage

    \chapter{Gestión del proyecto}
    

\section{Planificación}
    \subsection{Diagramas de Gantt}
    
\section{Seguimiento}
    \section{Línea temporal del proyecto}
        \begin{vtimeline}[.85]{Línea temporal del proyecto.}
            14/03/2022 & Se concurre a las ayudas para el desarrollo de Trabajos Fin de Titulación que requieran un subsistema hardware \\
            xx/09/2022 & Inicio del proyecto. \\
            14/03/2022 & Se concurre a las ayudas para el desarrollo de Trabajos Fin de Titulación que requieran un subsistema hardware \\
            xx/09/2022 & Inicio del proyecto. \\
            14/03/2022 & Se concurre a las ayudas para el desarrollo de Trabajos Fin de Titulación que requieran un subsistema hardware \\
            xx/09/2022 & Inicio del proyecto. \\
            14/03/2022 & Se concurre a las ayudas para el desarrollo de Trabajos Fin de Titulación que requieran un subsistema hardware \\
            xx/09/2022 & Inicio del proyecto. \\
            14/03/2022 & Se concurre a las ayudas para el desarrollo de Trabajos Fin de Titulación que requieran un subsistema hardware \\
            xx/09/2022 & Inicio del proyecto. \\
            14/03/2022 & Se concurre a las ayudas para el desarrollo de Trabajos Fin de Titulación que requieran un subsistema hardware \\
            xx/09/2022 & Inicio del proyecto. \\
            14/03/2022 & Se concurre a las ayudas para el desarrollo de Trabajos Fin de Titulación que requieran un subsistema hardware \\
            xx/09/2022 & Inicio del proyecto. \\
            14/03/2022 & Se concurre a las ayudas para el desarrollo de Trabajos Fin de Titulación que requieran un subsistema hardware \\
            xx/09/2022 & Inicio del proyecto. \\
            14/03/2022 & Se concurre a las ayudas para el desarrollo de Trabajos Fin de Titulación que requieran un subsistema hardware \\
            xx/09/2022 & Inicio del proyecto. \\
        \end{vtimeline}
    \section{Métricas}
    
\section{Presupuesto}
    \section{Materiales}
    \newpage

    \chapter{Conclusiones}
    \section{Conclusiones técnicas}
    
\section{Conclusiones sociales}
    
\section{Balance del aprendizaje}
    
\section{Reflexión final}
    
    \newpage

    \chapter{Lineas futuras}
    \input{contenido/futuras.tex}
    \newpage

    % Bibilografía
    \printbibliography[heading=bibintoc]

    % Anexos
    \appendix
    \clearpage

    % Anexo de código
    \chapter{Código fuente del proyecto} 
    \section{App}
    \subsection{Demo.kt}
        \lstinputlisting[label={src:Demo.kt},caption={Demo.kt}, style=KotlinAnexos]{codigo/Demo.kt}
    \newpage

\end{document}