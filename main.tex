%Definimos la clase del documento y cargamos las librerías
\documentclass[a4paper]{report}
\usepackage{graphicx}
\usepackage[utf8]{inputenc}
%paquetes para la portada azul
\usepackage{afterpage}
\usepackage{xcolor}
%Para tener los elementos de texto en español. El flag noshorthands arregla el bug de Argument of \language@active@arg" has an extra }.
%http://algunostutoriales.blogspot.com/2014/03/arreglar-un-extrano-error-de-latex.html
\usepackage[spanish,es-noshorthands]{babel}
%Hipervinculos para todo lo útil
\usepackage{hyperref}
%evitamos que rompa las palabras
\usepackage[htt]{hyphenat}
%soporte para anexos
\usepackage{appendix}
%paquetes para figuras
\usepackage{caption}
\usepackage{subcaption}
\usepackage{lscape}
%extracto codigo en memoria
\usepackage{listings}
%posicionamiento
\usepackage{float}
%tablas
\usepackage{xltabular}
%bibiografia
\usepackage[style=ieee]{biblatex}
\usepackage{csquotes}
\bibliography{./secciones/bibliografia.bib}

% formato de javi
\lstset{ %
  backgroundcolor=\color{white},   % Indica el color de fondo; necesita que se añada \usepackage{color} o \usepackage{xcolor}
  breakatwhitespace=false,         % Activarlo para que los saltos automáticos solo se apliquen en los espacios en blanco
  inputencoding=utf8,
  captionpos=b,                    % Establece la posición de la leyenda del cuadro de código
  commentstyle=\color{darkgreen},    % Estilo de los comentarios
  extendedchars=true,              % Permite utilizar caracteres extendidos no-ASCII; solo funciona para codificaciones de 8-bits; para UTF-8 no funciona. En xelatex necesita estar a true para que funcione.
  keepspaces=true,                 % Mantiene los espacios en el texto. Es útil para mantener la indentación del código(puede necesitar columns=flexible).
  columns=flexible,
  keywordstyle=\color{darkorange},       % estilo de las palabras clave
  rulecolor=\color{black},         % Si no se activa, el color del marco puede cambiar en los saltos de línea entre textos que sea de otro color, por ejemplo, los comentarios, que están en verde en este ejemplo
  showspaces=false,                % Si se activa, muestra los espacios con guiones bajos; sustituye a 'showstringspaces'
  showstringspaces=false,          % subraya solamente los espacios que estén en una cadena de esto
  showtabs=false,                  % muestra las tabulaciones que existan en cadenas de texto con guión bajo
  stringstyle=\color{darkgreen},     % Estilo de las cadenas de texto
  identifierstyle=\color{blue},
  tabsize=4,	                   % Establece el salto de las tabulaciones a 2 espacios
  breaklines=true,
  basicstyle=\footnotesize\ttfamily,        % Fija el tamaño del tipo de letra utilizado para el código
  postbreak=\mbox{\textcolor{red}{$\hookrightarrow$}\space},
  literate={á}{{\'a}}1 {é}{{\'e}}1 {í}{{\'i}}1 {ó}{{\'o}}1 {ú}{{\'u}}1
  {Á}{{\'A}}1 {É}{{\'E}}1 {Í}{{\'I}}1 {Ó}{{\'O}}1 {Ú}{{\'U}}1
  {à}{{\`a}}1 {è}{{\`e}}1 {ì}{{\`i}}1 {ò}{{\`o}}1 {ù}{{\`u}}1
  {À}{{\`A}}1 {È}{{\'E}}1 {Ì}{{\`I}}1 {Ò}{{\`O}}1 {Ù}{{\`U}}1
  {ä}{{\"a}}1 {ë}{{\"e}}1 {ï}{{\"i}}1 {ö}{{\"o}}1 {ü}{{\"u}}1
  {Ä}{{\"A}}1 {Ë}{{\"E}}1 {Ï}{{\"I}}1 {Ö}{{\"O}}1 {Ü}{{\"U}}1
  {â}{{\^a}}1 {ê}{{\^e}}1 {î}{{\^i}}1 {ô}{{\^o}}1 {û}{{\^u}}1
  {Â}{{\^A}}1 {Ê}{{\^E}}1 {Î}{{\^I}}1 {Ô}{{\^O}}1 {Û}{{\^U}}1
  {œ}{{\oe}}1 {Œ}{{\OE}}1 {æ}{{\ae}}1 {Æ}{{\AE}}1 {ß}{{\ss}}1
  {ç}{{\c c}}1 {Ç}{{\c C}}1 {ø}{{\o}}1 {å}{{\r a}}1 {Å}{{\r A}}1
  {€}{{\EUR}}1 {£}{{\pounds}}1 {ñ}{{\~n}}1 {Ñ}{{\~N}}1
}
\lstdefinestyle{CAnexos}{
    language=C,
    otherkeywords={inline, \#ifdef, \#endif, \#ifndef},
    numbers=left,                    % Posición de los números de línea (none, left, right).
    numbersep=5pt,                   % Distancia de los números de línea al código
    numberstyle=\small\color{gray}, % Estilo para los números de línea
    frame=single	                   % Añade un marco al código
}

\lstdefinestyle{PythonAnexos}{
    language=Python,
    numbers=left,                    % Posición de los números de línea (none, left, right).
    numbersep=5pt,                   % Distancia de los números de línea al código
    numberstyle=\small\color{gray}, % Estilo para los números de línea
    frame=single	                   % Añade un marco al código
}

\lstdefinestyle{cppAnexos}{
    language=C++,
    numbers=left,                    % Posición de los números de línea (none, left, right).
    numbersep=5pt,                   % Distancia de los números de línea al código
    numberstyle=\small\color{gray}, % Estilo para los números de línea
    frame=single
}

\lstdefinestyle{bashAnexos}{
    language=bash,
    numbers=left,                    % Posición de los números de línea (none, left, right).
    numbersep=5pt,                   % Distancia de los números de línea al código
    numberstyle=\small\color{gray}, % Estilo para los números de línea
    frame=single
}

\lstdefinestyle{txtAnexos}{
    numbers=left,                    % Posición de los números de línea (none, left, right).
    numbersep=5pt,                   % Distancia de los números de línea al código
    numberstyle=\small\color{gray}, % Estilo para los números de línea
    frame=single
}

\lstdefinestyle{C}{
    language=C,
    otherkeywords={inline, \#ifdef, \#endif, \#ifndef}
}

\lstdefinestyle{Python}{
    language=Python
}

\lstdefinestyle{C++}{
    language=C++,
    otherkeywords={inline, \#ifdef, \#endif, \#ifndef}
}

\lstdefinelanguage{json}{
    basicstyle=\small\ttfamily,
    commentstyle=\color{eclipseStrings}, % style of comment
    stringstyle=\color{eclipseKeywords}, % style of strings
    numbers=left,
    numberstyle=\scriptsize,
    stepnumber=1,
    numbersep=8pt,
    showstringspaces=false,
    frame=single,
    string=[s]{"}{"},
    comment=[l]{:\ "},
    morecomment=[l]{:"}
}

\lstdefinestyle{jsonAnexos}{
    language=json,
    numbers=left,                    % Posición de los números de línea (none, left, right).
    numbersep=5pt,                   % Distancia de los números de línea al código
    numberstyle=\small\color{gray}, % Estilo para los números de línea
    frame=single
}

\usepackage{xcolor}
\definecolor{pblue}{rgb}{0.13,0.13,1}
\definecolor{pgreen}{rgb}{0,0.5,0}
\definecolor{pred}{rgb}{0.9,0,0}
\definecolor{pgrey}{rgb}{0.46,0.45,0.48}
\definecolor{darkgreen}{rgb}{0.0, 0.4, 0.0}
\definecolor{darkorange}{rgb}{1.0, 0.55, 0.0}
\definecolor{eclipseStrings}{RGB}{42,0.0,255}
\definecolor{eclipseKeywords}{RGB}{127,0,85}
\colorlet{numb}{magenta!60!black}

%paquete para poner la bibliografia en la table of contents
\usepackage[nottoc,notlot,notlof]{tocbibind}

%Cambiamos el espacio en blanco enorme que por defecto hay en las páginas inicio de capítulo
%\usepackage{titlesec}
%\titleformat{\chapter}[display]   
%{\normalfont\huge\bfseries}{\chaptertitlename\ \thechapter}{20pt}{\Huge}   
%\titlespacing*{\chapter}{0pt}{-50pt}{40pt}

%para crear subsubsubsection pero que no salgan los numeros de paragraph
%\setcounter{secnumdepth}{2}
%\titleformat{\paragraph}
%{\normalfont\normalsize\bfseries}{\theparagraph}{1em}{}
%\titlespacing*{\paragraph}
%{0pt}{3.25ex plus 1ex minus .2ex}{1.5ex plus .2ex}

%definimos el azul de la portada
\definecolor{AzulPortada}{RGB}{135, 206, 250}

%Ruta en la que pondremos las imágenes para el documento
\graphicspath{ {imagenes/} }
%Redifinimos el nombre que asigna babel spanish a las tablas (cuadro) a tabla
\renewcommand{\spanishtablename}{Tabla}
\renewcommand{\spanishlisttablename}{Índice de tablas}
\renewcommand{\spanishcontentsname}{Índice}
\renewcommand{\appendixname}{Anexos}
\renewcommand{\appendixtocname}{Anexos}
\renewcommand{\appendixpagename}{Anexos}

% Definimos el comando de página en blanco
\newcommand\paginablanco{%
    \null
    \thispagestyle{empty}%
    \newpage}

% Definimos el comando de página en blanco sin avanzar numeracion
\newcommand\paginablancosin{%
    \paginablanco{}
    \addtocounter{page}{-1}}

%mayor anchura en las tablas
\renewcommand{\arraystretch}{1.5}
%mayor ancuhura en las fracciones
\newcommand\ddfrac[2]{\frac{\displaystyle #1}{\displaystyle #2}}
%coamdno escribir TFG
\newcommand{\tfg}{Trabajo Fin de Grado }

%indice de codigo
\renewcommand*{\lstlistingname}{Código}
\renewcommand{\lstlistlistingname}{Índice de código}

%javi margenes
\usepackage[pdftex,scale={.8,.8}]{geometry} %%The pack­age pro­vides an easy and flex­i­ble user in­ter­face to cus­tomize page lay­out, im­ple­ment­ing auto-cen­ter­ing and auto-bal­anc­ing mech­a­nisms so that the users have only to give the least de­scrip­tion for the page lay­out. For ex­am­ple, if you want to set each mar­gin 2cm with­out header space, what you need is just \usep­a­ck­age[mar­gin=2cm,no­head]{ge­om­e­try}.

%repetir imagenes
\usepackage{caption}

\newcommand{\repeatcaption}[2]{%
  \renewcommand{\thefigure}{\ref{#1}}%
  \captionsetup{list=no}%
  \caption{#2 (Figura repetida, original en la página \pageref{#1})}%
  \addtocounter{figure}{-1}% So that next figure after the repeat gets the right number.
}

\usepackage{titlesec}
%Encabezados bonitos
\newpagestyle{long}{%
	\sethead[\sectiontitle][][\chaptername\ \thechapter:\ \chaptertitle]
    {\chaptername\ \thechapter:\ \chaptertitle}{}{\sectiontitle}
    \headrule
	\setfoot[\thepage][][]{}{}{\thepage}
}
\newpagestyle{short}{
	\setfoot[\thepage][][]{}{}{\thepage}
}
\patchcmd{\chapter}{plain}{short}{}{} %$ <-- the header on chapter 1

% Metadatos

\def\University{Universidad Politécnica de Madrid\\ Escuela Técnica Superior de Ingeniería de Sistemas Informáticos}
\def\Course{Máster Universitario en Software de Sistemas Distribuidos y Empotrados}
\def\Module{Proyecto Fin de Máster}
\def\ModuleMayus{PROYECTO FIN DE MÁSTER}

\def\Docent{Sandra Gómez Canaval, Gema Bello Orgaz }
\def\DocentMulti{Sandra Gómez Canaval \\ Gema Bello Orgaz }
\def\Assistant{}

\def\Title{Sistema para el Bienestar Emocional}
\def\Authors{Victor Manuel Domínguez Rivas}
\def\Shortname{V. Dominguez}
\def\EndDate{Julio de 2023}


%\Institute\\ \Course\\ \Module\\
% \subject{\Institute \\ \Course \\ \Module}
\title{\Title}
% \title{\Institute\\\Course\\\Module\\\Title}
% \title{\Institute\\ \Course\\ \Module\\ \\ \Title}
\author{\Authors}
\date{Madrid, \today{}}

%%%%%%%%%%%%%%%%%%%%%%%%%%%%%%%%%%%%%%%%%%%%%%%%%%%%%%%%%%%%%%%%%%%%%%%%%%%%%%%%%
%% Creation of pdf information
%%%%%%%%%%%%%%%%%%%%%%%%%%%%%%%%%%%%%%%%%%%%%%%%%%%%%%%%%%%%%%%%%%%%%%%%%%%%%%%%%
\hypersetup{pdfinfo={
			Title={\Title},
			Author={\Authors},
			Subject={\Module -- \Course}
		}}

% Comenzamos el documento
\begin{document}
% Hacemos la portada

\begin{titlepage}
  % logo de la etsisi a la izquierda
  \begin{flushleft}
    \includegraphics[width=.4\linewidth]{imagenes/etsisi.png}
  \end{flushleft}
  
  % resto de la portada a mitad de folio
  \begin{center}
    \vspace*{1cm}

    \Huge
    \Course

    \vspace{3.5cm}
    \huge
    \textit{\Title}

    \vfill

    \Large
    \textsc{\ModuleMayus}

    \vspace{2.5cm}

    \Large
    \textit{\Authors}

    \vspace{2cm}

    \EndDate
  \end{center}

  \newpage
  
  \thispagestyle{empty}
  
  \begin{flushleft}
    \includegraphics[width=.4\linewidth]{imagenes/etsisi.png}
  \end{flushleft}

  \begin{center}
    \vspace*{1cm}

    \Huge
    \Course

    \vspace{3.5cm}
    \huge
    \textit{\Title}

    \vfill

    \Large
    \textsc{\ModuleMayus}

    \vspace{2.5cm}

    \Large
    \begin{flushright}
      \textit{Autor: \Authors} \\
      \vspace{0.5cm}
      \textit{Directoras: \Docent} \\
    \end{flushright}

    \vspace{2cm}

    \EndDate
  \end{center}

  \newpage
\end{titlepage}

\pagenumbering{Roman} % para comenzar la numeracion de paginas en numeros romanos

% Agradecimientos
\chapter*{}
\thispagestyle{empty}
\addcontentsline{toc}{section}{Agradecimientos} % si queremos que aparezca en el índice
\begin{flushright}
\textit{Tus agradecimientos aquí.}

\end{flushright}

\thispagestyle{empty}
\chapter*{Resumen} % si no queremos que añada la palabra "Capitulo"
\addcontentsline{toc}{section}{Resumen} % si queremos que aparezca en el índice

Tu resumen aquí. \newline


\textbf{Palabras clave}: Palabras clave


\thispagestyle{empty}
\chapter*{Abstract} % si no queremos que añada la palabra "Capitulo"
\addcontentsline{toc}{section}{Abstract} % si queremos que aparezca en el índice

Your abstract here. \newline

\textbf{Keywords}: Keywords

\pagestyle{short}
\tableofcontents
\newpage
\listoftables %ver si hay alguna tabla al final
\newpage
\listoffigures
\newpage
\lstlistoflistings
\newpage

\paginablanco{}

\pagenumbering{arabic}
\pagestyle{long}

\chapter{Introducción}
\chapter{Introducción}
\label{chapter:introduccion}

\chapquote{Si he logrado ver más lejos ha sido porque he subido a hombros de gigantes.}{Isaac Newton}

El proyecto desarollado consiste, a grandes rasgos, en sistema que permita al usuario conocer su estado mental, en 
particular de tres variables: estrés, depresión y soledad. Para


Un párrafo o dos como los de Clara

\section{Contexto}

Aqui vendemos la motito del tema en cuestión

\section{Motivación}

\section{Justificación}

Por qué es importante para el mundo (más a nivel político, stats)

\section{Objetivos}

    \subsection{Objetivo general}

    El objetivo de todo el proyecto

    \subsection{Objetivos específicos}


    Los objetivos como tal


        El objetivo de un \gls{pfg}, \gls{pfm} y  \gls{td} es una de las piezas clave a plantear, y a su vez una de las más complicadas. Se considera \textbf{la finalidad} del proyecto en cuestión a realizar y suele encajar dentro de una de las siguientes categorías:

        \begin{itemize}
            \item \textbf{Contraste} o validación de una hipótesis. Este es típico de \glspl{td}, aunque algunos \glspl{pfm} y (muy raramente) \glspl{pfg} pueden caer dentro de esta categoría.
            \item \textbf{Desarrollo} o diseño de algo (e.g.~Software, hardware, sistema, edificio). Suele ser el más común en la rama de la ingeniería, tanto \glspl{pfm} como \glspl{pfg}.
            \item \textbf{Estudio} de un tema que deduce o descubre nuevo conocimiento. Éste suele ser más común en las ramas de las ciencias puras y humanidades, tanto \glspl{pfm} como \glspl{pfg}.
        \end{itemize}

        Decimos que es una pieza clave porque sirve como primer indicador de la consecución del proyecto. Si nos planteamos un objetivo, en las conclusiones podemos indicar si se ha cumplido o no el objetivo planteado. Por eso es necesario que el objetivo esté bien definido, porque si se acepta como objetivo válido en un proyecto, y éste se concluye como cumplido, el proyecto habrá sido ejecutado correctamente.

        Ahora bien, ¿cómo determinamos que el objetivo se ha cumplido? pues intentando definirlo para que se pueda cumplir, es decir, intentando que sea:

        \begin{itemize}
            \item \textbf{Acotado en el tiempo}, así es más fácil establecer un marco temporal para su realización y programar temporalmente las partes de las que se compone.
            \item \textbf{Medible}, para saber cómo de lejos estamos de llegar a un resultado aceptable.
            \item \textbf{Específico}, de manera que esté bien acotado y sea difícil embarcarse en tareas que no nos acerquen a su consecución.
            \item \textbf{Alcanzable}, porque si no lo es, por mucha intención y esfuerzo que le pongamos no se va a terminar.
            \item \textbf{Relevante}, porque si, en un \gls{pfg} para Ingeniería del Software, desarrollamos un producto mecánico para sexar pollos, pues por muy importante que sea, poco tiene que ver con lo que se ha estudiado durante todos estos años.
        \end{itemize}

        Y sí, para acordarnos de cuáles son estas características podemos usar el acrónimo %\textit{AMEAR}.

\section{Estructura del documento}
    La
\newpage

\chapter{Objetivos}
\input{secciones/objetivos}
\newpage

\chapter{Metodologia}
\input{secciones/metodologia}
\newpage

\chapter{Especificación de requisitos}
% Secciones de IEEE 830: https://www.fdi.ucm.es/profesor/gmendez/docs/is0809/ieee830.pdf
\section{Introducción}
    \subsection{Propósito}
        
    \subsection{Alcance}
        
\section{Descripción}
    \subsection{Perspectiva del Producto}
        
    \subsection{Funciones del Producto}
        
    \subsection{Características de los Usuarios}
        
    \subsection{Restricciones}
        
    \subsection{Suposiciones y Dependencias}
        
    \subsection{Requisitos Futuros}

\section{Requisitos Específicos}
    \subsection{Interfaces Externas}
        
    \subsection{Funciones}
        
    \subsection{Requisitos de Rendimiento}
        
    \subsection{Restricciones de Diseño}
        
    \subsection{Atributos del Sistema}
        
    \subsection{Otros Requisitos}
        
\newpage

\chapter{Diseño del sistema}
\input{secciones/diseno}
\newpage

\chapter{Implementación del sistema}
\input{secciones/implementacion}
\newpage

\chapter{Pruebas del sistema}
\input{secciones/pruebas}
\newpage

\chapter{Resultados}
\input{secciones/resultados}
\newpage

\chapter{Impacto social y medioambiental}
\input{secciones/impacto}
\newpage

\chapter{Presupuesto}
\input{secciones/presupuesto}
\newpage

\chapter{Conclusiones}
\section{Conclusiones técnicas}
    
\section{Conclusiones sociales}
    
\section{Balance del aprendizaje}
    
\section{Reflexión final}
    
\newpage

\chapter{Lineas futuras}
\input{secciones/futuras}
\newpage

%\chapter*{Bibliografía}
%\addcontentsline{toc}{chapter}{Bibliografía}
\printbibliography[heading=bibintoc]

\appendix
\clearpage
\addappheadtotoc
\appendixpage

\chapter{Código fuente del proyecto} %uno o dos anexos??
\section{App}
    \subsection{Demo.kt}
        \lstinputlisting[label={src:Demo.kt},caption={Demo.kt}, style=KotlinAnexos]{codigo/Demo.kt}
\newpage
\paginablanco{}

\end{document}