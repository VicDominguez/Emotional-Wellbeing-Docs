% Basada en las de Javier Alonso (@javinator9889) y Bernard (tex.stackexchange)
% https://github.com/Javinator9889/VIMS-Memory/blob/main/base/vtimeline.tex
% https://tex.stackexchange.com/a/196808

%%%%%%%%%%%%%%%%%%%%%%%%%
%% Importación paquetes
%%%%%%%%%%%%%%%%%%%%%%%%%
\usepackage{environ}
\usepackage{fourier, heuristica}
\usepackage{array, booktabs}
%\usepackage[x11names,table]{xcolor} % Si no se importa xcolor en main, si se importa hay que añadir estos paquetes

%%%%%%%%%%%%%%%%%%%%%%%%%
%% Timeline
%%%%%%%%%%%%%%%%%%%%%%%%%
\definecolor{SteelBlue4}{HTML}{36648B}
\newcommand{\foo}{\color{SteelBlue4}\makebox[0pt]{\textbullet}\hskip-0.5pt\vrule width 1pt\hspace{\labelsep}}
\DeclareCaptionFont{blue}{\color{SteelBlue4}}
\environbodyname\envbody

\NewEnviron{vtimeline}[2][.9]
{
    \captionsetup{singlelinecheck=false, font=blue, labelfont=sc, labelsep=quad} %
    \renewcommand\arraystretch{1.4}\arrayrulecolor{SteelBlue4} % Pone la toprule en el color del caption
    \begin{longtable}{@{\,}r <{\hskip 2pt} !{\foo} p{#1\linewidth}}%
        \caption{#2} \\ %
        \toprule
        \addlinespace[1.5ex]
        \envbody
    \end{longtable} %
}